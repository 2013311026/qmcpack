\part{Introduction to QMCPACK}\label{intro.part}%\hyperlabel{intro.part}%

 This guide is not meant to discuss Quantum Monte Carlo methods and is
written on the assumption that the users are familiar with various QMC
algorithms. There are many excellent tutorials and talks on QMC methods
and numerous published works. A short list includes 

\begin{itemize}
\item{}\href{http://www.physics.uiuc.edu/people/Ceperley/}{Home page of Prof. David M Ceperley}  
\item{}\href{http://cms.mcc.uiuc.edu/wiki/display/ss2007qmc/Home}{
Quantum Monte Carlo from Minerals and Materials to Molecules,
2007 Summer School on Computational Materials Science}
%\item{}\href{http://cdsagenda5.ictp.trieste.it/full_display.php?ida=a0332}
%{Joint DEMOCRITUS-ICTP School on Continuum Quantum Monte CarloMethods (Trieste, Italy, 2004) }  
\item{}\href{http://www.tcm.phy.cam.ac.uk/~mdt26/casino2.html}{QUANTUM MONTE CARLO AND THE CASINO PROGRAM, UK}  
\end{itemize}

\section{Getting and building QMCPACK}\label{start.sec}

Consult \wiki for the updated instructions and helps on advanced topics.

\subsection{Prerequisite}\label{pre.sec}

QMCPACK can be installed on any Unix/Linux distribution with recent gnu
compilers (4.1 and higher). Other tested compilers are Intel compilers and IBM
VisualAge C++.  \boost provides current status of C++ compilers. Please check
if your compiler can support {\tt boost/random} library, which is the minimum
requirement to compile QMCPACK and associated packages.

In order to install QMCPACK, the users have to install \cmake and several
required packages. These packages are included in the standard Linux/cygwin
distributions or can be downloaded by following the links. If these libraries
are installed in standard directories, {\tt /usr}, {\tt /usr/local}, no action
is necessary.  Alternatively, the users can set environment variables \texttt{
XYZ\_HOME} where \texttt{XYZ} stands for the name of package; the build utility
can locate the libraries and use them.

The build utility \cmake will look for {\tt XYZ\_HOME/include} for the header
files and \texttt{XYZ\_HOME/lib} for the library files in addition to the 
standard paths.  When multiple environment variables apply to a library, e.g., {\tt
blas/lapack}, the search is done according to the given  order.  On many HPC
centers, these packages are managed by utilities like \textbf{module} and
\textbf{softenv}. Consult the documentations on each site. 

\subsection{Building using {\tt cmake}}\label{cmake.sec}

For {\tt bash} users,
\begin{enumerate}
\item Download the source code using subversion:
\begin{lstlisting}
  svn co http://cms.mcc.uiuc.edu/svn/qmcpack-dev/trunk qmcpack
\end{lstlisting}
\item Move to a build directory:
\begin{lstlisting}
  cd qmcpack/build
\end{lstlisting}
\item Set compilers and \texttt{XYZ\_HOME} for the required external packages:
\begin{lstlisting}
  export CXX=g++
  export CC=gcc
\end{lstlisting}
\item Run {\tt cmake} to create \nolinkurl{Makefile}s:
\begin{lstlisting}
  cmake ..
\end{lstlisting}
\item Build everything:
\begin{lstlisting}
  make
\end{lstlisting}
\end{enumerate}

If everything goes well, main executable {\tt qmcapp} and other utilites are
located in {\tt bin} directory. 

Customized compiler flags are provided for these compilers:
\begin{itemize}
\item{}GNU compilers 4.1 and higher : GNUCompilers.cmake 
\begin{lstlisting}
  export CXX=g++
  export CC=gcc
\end{lstlisting}
\item{}Intel compilers 9.* and higher : IntelCompilers.cmake 
\begin{lstlisting}
  export CXX=icpc
  export CC=icc
\end{lstlisting}
\item{}IBM Visual Age C++ : IBMCompilers.cmake
\begin{lstlisting}
  export CXX=xlC_r
  export CC=xlc_r
\end{lstlisting}
\end{itemize}
The OpenMP parallelization is automatically enabled with these compilers. To
enable MPI parallelization on distributed-memory computers, set {\tt CXX} to an
appropriate compiler, e.g. {\tt mpicxx}.
  
\subsection{Running QMC}

Once the compilation is completed, \textbf{bin/qmcapp} the main QMCPACK
application will be created in the build directory.  In your shell (assuming
that you are currently in \emph{todir}/build directory), simply type

\begin{verbatim}
  $ bin/qmcapp 
\end{verbatim}
and a message will appear indicating no valid input file is provided. 
 
\begin{lstlisting}[float]
=====================================================
                    qmcpack 0.4                       

(c) Copyright 2003-  qmcpack developers            
=====================================================
ERROR No input file is given.
ERROR usage: qmcapp input-file
Bye \end{lstlisting}
  

However, getting this message is important since it indicates the compilation
is completed and your run-{}time environments are properly set.  Now, QMCPACK
is ready to run.
 
