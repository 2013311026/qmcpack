\documentclass[letterpaper]{article}
\special{papersize=8.5in,11in}
\setlength{\pdfpageheight}{\paperheight}
\setlength{\pdfpagewidth}{\paperwidth}
\usepackage{amsmath}
\title{Notes on Orbital Optimization}
\author{Ken Esler}
\begin{document}
\maketitle
\newcommand{\vr}{\mathbf{r}}
\newcommand{\Aopt}{\tilde{A}}

\section{Trial wave functions in logarithmic form}
Let us consider a trial wave function, $\psi$, which is a product of
components,
\begin{equation}
  \psi = \psi_1 \psi_2 \psi_3 \dots
\end{equation}
To avoid the complicated application of the product rule, we work
rather with the quantity, $\ln(\psi)$.
\begin{equation}
\ln(\psi) = \ln(\psi_1) + \ln(\psi_2) + \ln(\psi_3) \dots
\end{equation}
\subsection{Optimizing trial wave functions}
Each of the components of $\psi$ can be parametrized to allow
optimization of the trial function.  Efficient optimization algorithms
can be employed if we can compute the derivatives of two quantities
with respect to these parameters.  For simplicity, we label one
parameter as $\alpha$.  We then need to compute
\begin{eqnarray}
\partial_\alpha \ln(\psi) & = & \partial_\alpha \ln (\psi_1)
+ \partial_\alpha \ln (\psi_2) + \dots,\ \text{and} \\
\partial_\alpha \left[ \frac{\nabla^2\psi}{\psi}\right] & = &
\partial_\alpha \left\{\nabla^2 \ln(\psi) + \left[\nabla
    \ln(\psi)\right]^2 \right\} \\
& = & \partial_\alpha \nabla^2 \ln(\psi) + 2 \left[\nabla \ln(\psi)
\right]\cdot \left[ \partial_\alpha \nabla \ln(\psi)\right].
\end{eqnarray}
Let us consider the second term in terms of its components.
\begin{eqnarray}
\partial_\alpha \left[ \frac{\nabla^2\psi}{\psi}\right] & = &
 \partial_\alpha \nabla^2\ln(\psi_1) + \partial_\alpha
 \nabla^2\ln(\psi_2)+ \dots \\
& & + 2\left[\nabla \ln(\psi)\right]\cdot
\left[\partial_\alpha \nabla \ln(\psi_1)\right] +
2\left[\nabla \ln(\psi)\right]\cdot
\left[\partial_\alpha \nabla \ln(\psi_2)\right] + \cdots \nonumber
\end{eqnarray}
Thus, the contribution of $\psi1$ to the total is given by
\begin{equation}
\partial_\alpha \left[ \frac{\nabla^2\psi}{\psi}\right]_{\psi_1} =
\partial_\alpha \nabla^2\ln(\psi_1) + 2\left[\nabla \ln(\psi)\right]\cdot
\left[\partial_\alpha \nabla \ln(\psi_1)\right].
\end{equation}


\section{Statement of problem}
We would like to optimize the occupied single-particle orbitals,
$\left\{\phi_i(\vr)\right\}$ in a determinant in the basis of the
single-particle excited states, $\left\{\varphi_j(\vr)\right\}$.
That is, we write the optimized orbitals,
$\left\{\tilde{\phi}_i(\vr)\right\}$, as
\begin{equation}
\tilde{\phi}_i(\vr) = \phi_i(\vr) + \sum_j c_{ij} \varphi_j(\vr).
\end{equation}

In this set of notes, we derive the required algebra for computing the
derivatives of the determinants w.r.t. $c_{ij}$.  First, let
\begin{equation}
A_{mn} \equiv \phi_m(\vr_n),\qquad \Aopt_{mn} \equiv \tilde{\phi}_m(\vr_n)
\end{equation}
If we let $\vr_n \rightarrow \vr'_n$, we can compute the ratio of the
new determinant to the old as
\begin{equation}
\frac{\det(A')}{\det(A)} = \sum_m \left(A^{-1}\right)_{nm} \phi_m(\vr_n').
\end{equation}
Similarly,
\begin{equation}
\frac{\nabla_n\det(A)}{\det(A)} = \sum_m \left(A^{-1}\right)_{nm} \nabla\phi_m(\vr_n).
\end{equation}
Analogously, 
\begin{equation}
\frac{d}{d c_{ij}} \log\left[\det(\Aopt)\right] = \sum_n
\left(A^{-1}\right)_{ni} \varphi_j(\vr_n).
\end{equation}
We need to also compute the derivative of the local kinetic energy
with respect to $c_{ij}$.  These terms are more complicated.  


We first need to compute
\begin{equation}
\frac{d}{d_{c_{ij}}} \left(\Aopt^{-1}\right)_{nm} = 
\end{equation}

\begin{equation}
\left[A + e_k\delta^T\right]^{-1} = A^{-1} - \frac{A^{-1}e_k\delta^T
  A^{-1}}{1 + \delta^T A^{-1}e_k},
\end{equation}

\begin{equation}
\left[A + e_k\delta^T\right]^{-1}_{nm} = \left[A^{-1}\right]_{nm} - \frac{\left[A^{-1}e_k\right]_n\left[\delta^T
  A^{-1}\right]_m}{1 + \delta^T A^{-1}e_k},
\end{equation}

\begin{equation}
\left[A + e_k\delta^T\right]^{-1}_{nm} = \left[A^{-1}\right]_{nm} - \frac{\left[A^{-1}\right]_{nk}\left[\delta^T
  A^{-1}\right]_m}{1 + \delta^T A^{-1}e_k},
\end{equation}

\begin{eqnarray}
\frac{d \left[\Aopt^{-1}\right]_{nm}}{d_{c_{ij}}} & = &
-\left[\Aopt^{-1}\right]_{ni}\left[\varphi_j(\vr_k)\Aopt^{-1}\right]_m  \\
& = & -\left[\Aopt^{-1}\right]_{ni} \sum_k \left(\Aopt^{-1}\right)_{km} \varphi_j(\vr_k) \\
& = & -\left[\Aopt^{-1}\right]_{ni} \frac{d}{d c_{mj}} \log\left[\det(\Aopt)\right]
%-\left[\varphi_j(\vr_m)\Aopt^{-1}\right]_i
\end{eqnarray}
Define
\begin{equation}
\gamma_{mj} \equiv \frac{d}{d c_{mj}} \log\left[\det(\Aopt)\right].
\end{equation}
Now, we can compute
\begin{eqnarray}
\frac{d}{dc_{ij}} \left[\frac{\nabla^2_n
    \det(\Aopt)}{\det(\Aopt)}\right] & = & \sum_m \left\{\rule{0cm}{0.6cm}
-\left[\Aopt^{-1}\right]_{ni}\left[\varphi_j(\vr_i)\Aopt^{-1}\right]_m
\nabla^2 \phi_m(\vr_n) +
\left(\Aopt^{-1}\right)_{nm} \nabla^2 \varphi_j(\vr_n)\delta_{i,m} \right\} \nonumber \\
& = & \sum_m \left\{\rule{0cm}{0.6cm}
-\left[\Aopt^{-1}\right]_{ni} \gamma_{mj}
\nabla^2 \phi_m(\vr_n) +
\left(\Aopt^{-1}\right)_{nm} \nabla^2 \varphi_j(\vr_n) \delta_{i,m}\right\} \\ 
& = & \sum_m \left\{\rule{0cm}{0.6cm}
-\left[\Aopt^{-1}\right]_{ni} \gamma_{mj}
\nabla^2 \phi_m(\vr_n) \right\}+
\left(\Aopt^{-1}\right)_{ni} \nabla^2 \varphi_j(\vr_n)
\end{eqnarray}
Define
\begin{eqnarray}
L_{nm} & \equiv & \nabla^2\phi_m(\vr_n) \\
\mathcal{L}_{nj} & \equiv & \nabla^2\varphi_j(\vr_n)
\end{eqnarray}
\begin{equation}
\frac{d}{dc_{ij}} \left[\frac{\nabla^2_n \det(\Aopt)}{\det(\Aopt)}\right] =
\sum_m  \left\{\rule{0cm}{0.6cm}
-\left[\Aopt^{-1}\right]_{ni} L_{nm} \gamma_{mj} \right\} + \left[\Aopt^{-1}\right]_{ni} \mathcal{L}_{nj}
\end{equation}

\newpage
When we change $c_{ij}$, a column of $\Aopt$ changes.  Using the Sherman-Morrison formula, we can compute the change in $\Aopt^{-1}$.
\begin{equation}
\left[\Aopt^{-1} + \Delta c_{ij} \varphi_j e_i^T\right]^{-1}_{nm} =
\Aopt^{-1} - \frac{\Delta c_{ij} \left[\sum_k \left(\Aopt^{-1}\right)_{nk} \varphi_j(\vr_k)\right]\Aopt^{-1}_{im}}{1 + \lambda}.
\end{equation}
We expand around $\Delta c_{ij}=0$.  Taking the derivative,
\begin{equation}
\frac{d}{dc_{ij}} \left[\Aopt^{-1}\right]_{nm} = \underbrace{\left[-\sum_k \Aopt^{-1}_{nk} \varphi_j(\vr_k)\right]}_{\equiv \gamma_{nj}}\Aopt^{-1}_{im}.
\end{equation}
Now, let us consider how the gradient with respect particle $m$ changes with
$c_{ij}$.  Two terms contribute.  First, a term contributes from the
ratio of 




\end{document}
