\part{Conversion and visualization utilites for QMCPACK}\label{conversion.part}
This part describes external utilities that are used to generate
required input files for QMCPACK

\section{Pseudopotential conversion}
Although psuedopotential file formats are generally fairly simple, and
some standards have been proposed, every major electronic structure
code has adopted its own proprietary format for essentially the same
data.  To overcome this problem, we have a developed a 

\section{Orbital conversion}
For typical materials and molecules simulations, QMCPACK requires
the single-particle orbitals as input.  At present, for orbitals in
Gaussian bases, the bases and coefficients are input directly in
QMCPACK's XML format.  Alternatively, the orbitals may be represented
either on a real-space mesh or a plane-wave expansion.  In these
latter cases, the information is stored in a cross-platform binary
format known as HDF5.  In particular, we have established a set of
conventions for storing the atomic geometry, orbitals, and related
information in and HDF5 in a format called ES-HDF.  In addition, we
have included tools to convert to ES-HDF from the output of a number
of common electronic structure codes:

\begin{table}[H]
\centering
\begin{tabular}{|c|c|c|c|}
\hline
Source format            & source basis & conversion tool & output format \\\hline
ABINIT                   & Plane waves  & wfconv          & ES-HDF \\
CASINO                   & Plane waves  & wfconv          & ES-HDF \\
GAMESS                   & Gaussians    & wfconv          & ES-HDF \\
Quantum Espresso (PWscf) & Plane waves  & pw2qmcpack.x    & ES-HDF \\
Qbox                     & Real-space   & wfconv          & ES-HDF \\
GAUSSIAN                 & Gaussians    & convert4qmc     & XML    \\
\hline
\end{tabular}
\end{table}

\subsection{Using wfconv}
\subsubsection{ABINIT}
\tt{istwfk 1}
\subsubsection{CASINO}
\subsubsection{GAMESS}
\subsubsection{Qbox}
\subsection{Using the mixed-basis representation}

\subsection{Converting from PWscf}


\section{Visualizing orbitals with wfvis}
\subsection{Building}
\subsection{Basic operations}
\subsection{Ray-tracing for publication-quality results}
