\chapter{Configuring Simulations}
%\section{Complete input file reference}
%... maybe a link to the schema once it's complete?
Why XML? We understand the cons of XML and a steep learning curve to overcome but we (at least JK) believe that the pros outweigh the cons. Especially from the developers' point, XML reflects the design of QMCPACK and makes the developers think about how the objects should be written. We recommend you (as a developer) to first write down XML element for a new object and think about how the logic works itself and with other objects. 

The input xml for qmc sections are extremely simple. The major difficulties concern trial wavefunctions. The components of a trial wavefunction can be divided into
\begin{enumerate}
\item{} Jastrow functions: one, two, and three (no four body yet)
\item{} Fermion wavefunction: Slater determinant, Geminal 
\end{enumerate}

Jastrow functions are rather simple but have to be defined by users.

On the other hand, users seldom have to prepare Fermion wavefunctions. They are extracted from outputs of third-party packages by tools (See QMCTools). Users may have to modify the name or id of the wavefunctions but again this can be achieved by command-line options for the tools.

We have not implemented backflow wavefunctions but think we can use Jastrow/Fermion to represent them. 

\section{Specifying the system}
cell and the particles

maybe a subsection on the HDF5 format?
\section{Specifying the wavefunction}
mention B-splines and postpone the discussion to ``Customized wavefunctions''
\subsection{Wavefunctions imported from other codes}
\subsubsection{Pseudopotentials}
mention OPIUM
\subsubsection{Orbitals}
mention Abinit, PWSCF
\section{Observables and estimators}
\section{QMC drivers}
mention wf test runs
\section{Simulation parameters}
\section{Optimization}
\section{GUI frontend}

%Using QMCPack
%  Start to finish walkthrough of a basic system (e.g. Si total energy)
%    Navigating the input file
%    Preparing DFT Orbitals
%    Wavefunction Optimization
%    Your First VMC Calculation
%    Your First DMC Calculation
%    Analyzing output data
%  Files used by QMCPack
%  Complete input file reference
%    Specifying the system
%    Specifying the wavefunction
%      One body jastrow
%      Two body jastrow
%      Three body jastrow
%      Slater Determinant
%      Multideterminants
%      Backflow
%    Optimization options
%      General parameters
%      Conjugate gradient 
%      Others
%    Controlling the calculation process
%      VMC
%      DMC
%    Estimators
%  Obtaining/Creating Pseudopotentials
%    Where to obtain quality pseudopotentials
%    Converting to the native fsatom format using PPconvert
%    Making pseudopotentials with OPIUM
%    Validating pseudopotentials
%      Checking for ghost states
%      Other tests
%  Generating Orbitals
%    From Abinit
%    From PWSCF
%    Using orbital file conversion utilities
%      wfconv
%      pw2qmcpack
%  Performing Optimization Well
%    Jastrow Optimization
%    Orbital Optimization

