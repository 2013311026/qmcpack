\part{Introduction to QMCPACK}\label{intro.part}%\hyperlabel{intro.part}%

This guide is not meant to discuss Quantum Monte Carlo methods and is
written on the assumption that the users are familiar with various QMC
algorithms. There are many excellent tutorials and talks on QMC methods
and numerous published works. A short list includes 

\begin{itemize}
\item{}\href{http://www.physics.uiuc.edu/people/Ceperley/}{Home page of Prof. David M Ceperley}  
\item{}\href{http://cms.mcc.uiuc.edu/wiki/display/ss2007qmc/Home}{
Quantum Monte Carlo from Minerals and Materials to Molecules,
2007 Summer School on Computational Materials Science}
%\item{}\href{http://cdsagenda5.ictp.trieste.it/full_display.php?ida=a0332}
%{Joint DEMOCRITUS-ICTP School on Continuum Quantum Monte CarloMethods (Trieste, Italy, 2004) }  
\item{}\href{http://www.tcm.phy.cam.ac.uk/~mdt26/casino2.html}{QUANTUM MONTE CARLO AND THE CASINO PROGRAM, UK}  
\end{itemize}

\section{Getting and building QMCPACK}\label{start.sec}

\subsection{Prerequisite}

In order to install QMCPACK, the users have to install \cmake and several
required packages. These packages are included in the standard Linux/cygwin
distributions or can be downloaded by following the links. If these libraries
are installed in standard directories, /usr /usr/local, no action is necessary.
Alternatively, the users can set environment variables XYZ\_HOME where XYZ
stands for the name of package; the build utility can locate the libraries and
use them.

With few exceptions, the build utility \cmake will look for XYZ\_HOME/include
for the header files and XYZ\_HOME/lib for the library files.  When multiple
environment variables apply to a library, e.g., blas/lapack, the library is
searched according to the listed order. 
