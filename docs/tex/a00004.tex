{\ttfamily cmake} is a portable build system and is widely used for large-\/scale software development projects such as V\+T\+K. How {\ttfamily cmake} works is similar to {\ttfamily autoconfig/automake}. {\ttfamily cmake} uses {\ttfamily C\+Make\+Lists.\+txt} files ({\ttfamily cmake} scripts) to generate {\ttfamily Makefile}s with complete dependency analysis. {\ttfamily C\+Make\+Lists.\+txt}s are equivalent to {\ttfamily Makefile.\+am}.

The {\ttfamily cmake} command 
\begin{DoxyCode}
cmake ..
\end{DoxyCode}
 starts processing {\ttfamily C\+Make\+Lists.\+txt} in {\ttfamily qmcpack} (the parent directory in this case) and the {\ttfamily C\+Make\+Lists.\+txt}s in {\ttfamily src} directories and below, recursively. In the {\ttfamily build} directory, {\ttfamily cmake} creates a directory tree which mirrors the {\ttfamily qmcpack} and {\ttfamily Makefile}s for each subdirectory\+:


\begin{DoxyCode}
$ls
-rw-r--r--  ...  cmake\_install.cmake
drwxr-xr-x  ...  bin
drwxr-xr-x  ...  lib
drwxr-xr-x  ...  src
-rw-r--r--  ...  Makefile
drwxr-xr-x  ...  CMakeFiles
-rw-r--r--  ...  CMakeCache.txt
\end{DoxyCode}


{\ttfamily qmcpack/\+C\+Make} directory contains customized cmake modules by Q\+M\+C\+P\+A\+C\+K developers to locate libraries and tools that are not fully supported by {\ttfamily cmake}. The list includes
\begin{DoxyItemize}
\item {\ttfamily Find\+Lapack.\+cmake} to locate blas/lapack library
\item {\ttfamily Find\+Einspline.\+cmake} to locate einspline library
\end{DoxyItemize}

These modules function as m4 files for autoconf.

{\ttfamily qmcpack/\+C\+Make\+Lists.\+txt} is a main file (text file) to build Q\+M\+C\+P\+A\+C\+K project. It
\begin{DoxyItemize}
\item defines how to build Q\+M\+C\+P\+A\+C\+K\+: dimensionality, precision, real/complex, mpi, openmp, .....
\item selects compilers
\item enables/disables advanced or developing features
\end{DoxyItemize}\section{Our-\/of-\/source compilation}\label{a00004_ooscomp}
Always use out-\/of-\/source compilation with cmake. The procedure above, using {\ttfamily build} directory (empty) and running {\ttfamily camke} in {\ttfamily build}, is an example. We can further separate the source (development) and build. Let's assume that the Q\+M\+C\+P\+A\+C\+K {\ttfamily topdir} is {\ttfamily /home/foo/src/qmcpack}. Then, one can build multiple executables in different locations by creating new directories and build Q\+M\+C\+P\+A\+C\+K in each directory.


\begin{DoxyCode}
/home/foo/build/gcc-real
/home/foo/build/gcc-complex
/home/foo/build/mpi-real
\end{DoxyCode}


In each directory, e.\+g., {\ttfamily /home/foo/build/gcc-\/real} (after secodeing the environments properly), execute 
\begin{DoxyCode}
cd /home/foo/build/gcc-real
cmake /home/foo/src/qmcpack
make 
\end{DoxyCode}
 There is no need to change sources or cmake files.

If something did not work, simply remove the directory (e.\+g., {\ttfamily rm -\/rf gcc-\/real}) and start again.\section{Building Q\+M\+C\+P\+A\+C\+K}\label{a00004_cmakebuild}
When {\ttfamily cmake} is issued, it generates {\ttfamily Makefile}s to build libraries and executables in the {\ttfamily build} directory. {\ttfamily build/lib} contains several libraries and {\ttfamily build/bin} has the executables including {\ttfamily qmcapp}, the main Q\+M\+C\+P\+A\+C\+K application.

The default build of Q\+M\+C\+P\+A\+C\+K is to perform Q\+M\+C simulations in the three-\/dimensional space with real trial wavefunction in double precision. These are set by the cmake variables in {\ttfamily C\+Make\+Lists.\+txt}\+: 
\begin{DoxyCode}
SET(OHMMS\_DIM 3 CACHE INTEGER \textcolor{stringliteral}{"Select physical dimension"}
SET(OHMMS\_INDEXTYPE \textcolor{keywordtype}{int})
SET(OHMMS\_PRECISION \textcolor{keywordtype}{double})
SET(APP\_PRECISION \textcolor{keywordtype}{double})

SET(PRINT\_DEBUG 0 CACHE BOOL \textcolor{stringliteral}{"Enable/disable debug printing"})
SET(QMC\_COMPLEX 0 CACHE INTEGER \textcolor{stringliteral}{"Build for complex binary"})
SET(QMC\_MPI 1 CACHE BOOL \textcolor{stringliteral}{"Enable/disable MPI"})
SET(QMC\_OMP 1 CACHE BOOL \textcolor{stringliteral}{"Enable/disable OpenMP"})
SET(QMC\_BITS 64 CACHE INTEGER \textcolor{stringliteral}{"Select OS bit"})
\end{DoxyCode}



\begin{DoxyItemize}
\item {\ttfamily O\+H\+M\+M\+S\+\_\+xyz} s are old macros and will be replaced by {\ttfamily A\+P\+P}. {\ttfamily A\+P\+P} stands for A\+P\+Plication so that other application can use it without feeling constrained by the name O\+H\+M\+M\+S. -\/{\ttfamily Q\+M\+C\+\_\+\+C\+O\+M\+P\+L\+E\+X=1} is for complex wavefunctions and fixed-\/phase D\+M\+C/\+R\+M\+C methods.
\item The \char`\"{}cached\char`\"{} parameters can be modified by users manually during a build by editing C\+Make\+Cache.\+txt.
\end{DoxyItemize}

Note that the available variables and their default values are subject to change. cmake files (C\+Make\+Lists.\+txt, C\+Make\+Cache.\+txt and those with cmake extension) are text files; you can read them, make sense out of them and modify them.\section{How to overwrite the default build variables}\label{a00004_cmakeadv1}
The build variables can be overwritten at {\ttfamily cmake} step by passing arguments to {\ttfamily cmake}. A general method to overwrite the build variables is 
\begin{DoxyCode}
cmake -DQMC\_MPI=0 -DQMC\_OMP=0 -DBUILD\_SANDBOX=1 ..
\end{DoxyCode}


Alternatively, {\ttfamily C\+Make\+Lists.\+txt} can be edited before {\ttfamily cmake} step.
\begin{DoxyItemize}
\item This is the only way to change {\ttfamily O\+H\+M\+M\+S\+\_\+\+P\+R\+E\+C\+I\+S\+I\+O\+N} and {\ttfamily O\+H\+M\+M\+S\+\_\+\+I\+N\+D\+E\+X\+T\+Y\+P\+E}
\item single-\/precision has not been debugged (probably compilers will give up).
\item There is N\+O N\+E\+E\+D to use long for {\ttfamily O\+H\+M\+M\+S\+\_\+\+I\+N\+D\+E\+X\+T\+Y\+P\+E}
\end{DoxyItemize}

{\ttfamily cmake} variables and their defaults are

\begin{TabularC}{4}
\hline
\rowcolor{lightgray}{\bf variable}&{\bf type}&{\bf default}&{\bf comment  }\\\cline{1-4}
Q\+M\+C\+\_\+\+B\+U\+I\+L\+D\+\_\+\+L\+E\+V\+E\+L &int &1 &Q\+M\+C Build Level\+: 1=bare, 2=developer, 3=experimental \\\cline{1-4}
O\+H\+M\+M\+S\+\_\+\+D\+I\+M &int &3 &physical dimension of the build \\\cline{1-4}
Q\+M\+C\+\_\+\+M\+P\+I &bool &1 &Eanble/disable M\+P\+I \\\cline{1-4}
Q\+M\+C\+\_\+\+O\+M\+P &bool &1 &Eanble/disable Open\+M\+P \\\cline{1-4}
Q\+M\+C\+\_\+\+C\+O\+M\+P\+L\+E\+X &bool &0 &Eanble/disable complex build \\\cline{1-4}
B\+U\+I\+L\+D\+\_\+\+Q\+M\+C\+T\+O\+O\+L\+S &bool &0 &Build tools for Q\+M\+C\+P\+A\+C\+K \\\cline{1-4}
B\+U\+I\+L\+D\+\_\+\+S\+A\+N\+D\+B\+O\+X &bool &0 &Build sandbox for testing for the developers \\\cline{1-4}
E\+N\+A\+B\+L\+E\+\_\+\+P\+H\+D\+F5 &bool &0 &Enable use of phdf5 \\\cline{1-4}
E\+N\+A\+B\+L\+E\+\_\+\+T\+A\+U\+\_\+\+P\+R\+O\+F\+I\+L\+E &bool &0 &Enable tau for profiling \\\cline{1-4}
\end{TabularC}
In addition to Q\+M\+C\+P\+A\+C\+K-\/defined variables, there are many {\ttfamily cmake} variables that can be manipulated the same way. Check out {\tt cmake wiki}.

During {\ttfamily cmake} step, {\ttfamily C\+Make\+Cache.\+txt} file is created in the {\ttfamily build} directory. As the name implies, it contains cached variables that are used during the build stage. This file can be edited to modify the cached variables above.\section{Building with environment variables}\label{a00004_buildenv}
This method works with G\+N\+U, Intel, and I\+B\+M X\+L\+C compilers. When the libraries are installed in standard locations, e.\+g., {\ttfamily /usr, /usr/local}, there is no need to set the {\ttfamily X\+Y\+Z\+\_\+\+H\+O\+M\+E }for X\+Y\+Z package. In this example of using Intel compilers, we set the environment variables in bash as


\begin{DoxyCode}
export CXX=icpc
export CC=icc
export MKL\_HOME=/usr/local/intel/mkl/10.0.3.020
export LIBXML2\_HOME=/usr/local
export HDF5\_HOME=/usr/local
export BOOST\_HOME=/usr/local/boost
export EINSPLINE\_HOME=/usr/local/einspline
export FFTW\_HOME=/usr/local/fftw
\end{DoxyCode}


{\ttfamily cmake} uses the default compilers on each platform. On most $\ast$\+N\+I\+X systems, G\+N\+U compilers are used when {\ttfamily C\+X\+X} and {\ttfamily C\+C} environment variables are not set.

The compiler options are automatically selected based on the name of compilers. For G\+N\+U, Intel and I\+B\+M compilers, customized {\ttfamily cmake} modules to set compiler options on target C\+P\+U are provided in {\ttfamily qmcpack/\+C\+Make} directory as summarized below.

\begin{TabularC}{4}
\hline
\rowcolor{lightgray}{\bf C\+X\+X }&{\bf C\+C }&{\bf cmake module file }&{\bf comments  }\\\cline{1-4}
g++ &gcc &G\+N\+U\+Compilers.\+cmake &G\+N\+U compilers \\\cline{1-4}
icpc&icc &Intel\+Compilers.\+cmake &Intel compilers \\\cline{1-4}
xl\+C &xlc &I\+B\+M\+Compilers.\+cmake &I\+B\+M Visual\+Age compilers \\\cline{1-4}
\end{TabularC}
Enabling M\+P\+I typically requires special compilers, e.\+g., {\ttfamily mpicxx}. The parallel programming environments on H\+P\+C systems are not standardized and it is hard to make a simple build system work for all the possible M\+P\+I varients. The build system provided with Q\+M\+C\+P\+A\+C\+K will try to figure out the best build options but the users should modify cmake files to meet their needs.

Development and testing are done mostly on L\+I\+N\+U\+X systems using Intel 10.\+x or G\+N\+U 4.\+2 and higher compilers. Older compilers will work but supports for Open\+M\+P both at the compiler and run time may not be good. We strongly encourage people to move on to new compilers whenever possible\+: they are usually getting better with few exceptions, which will be posted on this wiki whenever such cases are encountered.\section{Building with a toolchain file}\label{a00004_toolbuild}
Using a toolchain file can be convenient when the libraries cannot be easily located or cross-\/compilation is needed. This method is recommended on a H\+P\+C system to manage multiple versions of libraries and programming environment (compiler verions etc).

Several toolchain files used by the developers are available in {\ttfamily config} directory. They are used on large-\/scale parallel machines where setting up all the necessary packages can be tricky.
\begin{DoxyItemize}
\item {\ttfamily Titan\+G\+N\+U.\+cmake} for Cray X\+K7 system at O\+L\+C\+F, using only C\+P\+Us
\item {\ttfamily X\+C30\+Intel.\+cmake} for Cray X\+C30 system at N\+E\+R\+S\+C
\item {\ttfamily B\+G\+Q\+Tool\+Chain.\+cmake} for I\+B\+M B\+G\+Q at A\+L\+C\+F
\item {\ttfamily Psi\+Intel\+M\+P\+I.\+cmake} for generic x86 systems with Intel Composer X\+E v13
\end{DoxyItemize}

Once a suitable toolchain file is found, follow these step (example on titan@O\+L\+C\+F)\+: 
\begin{DoxyCode}
cd build
cmake -DCMAKE\_TOOLCHAIN\_FILE=../config/TitanGNU.cmake ..
cmake -DCMAKE\_TOOLCHAIN\_FILE=../config/TitanGNU.cmake ..
make -j16
\end{DoxyCode}
 {\ttfamily build} should be empty. Repeat {\ttfamily cmake} several times, until this message appears 
\begin{DoxyCode}
...
-- Generating done
-- Build files have been written to: your-build-directory
\end{DoxyCode}


For more information on build, consult {\tt Q\+M\+C\+P\+A\+C\+K wiki}. 