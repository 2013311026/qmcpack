\section{Download latest source tarball}\label{a00002_download}
See download section of {\tt http\-://www.\-qmcpack.\-org}\section{Quick build with make}\label{a00002_quicky}
\begin{DoxyNote}{Note}
From this point, we assume that you are working in qmcpack directory.
\end{DoxyNote}
When everything is installed in standard directores, {\ttfamily /usr}, {\ttfamily /usr/local} with default compilers 
\begin{DoxyCode}
cd build
cmake ..
make -j8
\end{DoxyCode}
 This will create a binary with {\bfseries real trial wavefunctions}. Use as many threads as you can with {\ttfamily make}.

On L\-I\-N\-U\-X, G\-N\-U compilers are the default. More on how to build Q\-M\-C\-P\-A\-C\-K can be found at {\tt http\-://qmcpack.\-cmscc.\-org/getting-\/started/cmake-\/101} If everything goes well, then you should see {\itshape qmcpack/build/bin/qmcapp}.

In addition, one has to build a binary with {\bfseries complex trial wavefunctions}. 
\begin{DoxyCode}
mkdir build\_complex
cd build\_complex
cmake -DQMC\_COMPLEX=1 ..
make -j8
\end{DoxyCode}


To build C\-U\-D\-A version, 
\begin{DoxyCode}
mkdir build\_cuda
cd build\_cuda
cmake -DQMC\_CUDA=1 ..
make -j8
\end{DoxyCode}
\section{Required tools and libraries}\label{a00002_pre_sec}

\begin{DoxyItemize}
\item C/\-C++ compilers
\item cmake, build utility, {\tt http\-://www.\-cmake.\-org/}
\item blas/lapack, numerical library, use platform-\/optimized libraries
\item libxml2, X\-M\-L parser, {\tt http\-://xmlsoft.\-org/}
\item hdf5, portable I/\-O library, {\tt http\-://www.\-hdfgroup.\-org/\-H\-D\-F5/}
\item boost, peer-\/reviewed portable C++ source libraries, {\tt http\-://www.\-boost.\-org}
\end{DoxyItemize}

Optional but required for a complete build
\begin{DoxyItemize}
\item fftw, F\-F\-T library, {\tt http\-://www.\-fftw.\-org/}
\end{DoxyItemize}

Einspline is distributed with Q\-M\-C\-P\-A\-C\-K. This is the official U\-R\-L for einspline library.
\begin{DoxyItemize}
\item einspline, 3\-D bspline library, {\tt http\-://einspline.\-sourceforge.\-net/}
\end{DoxyItemize}\section{External libraries}\label{a00002_extlib_sec}
Q\-M\-C\-P\-A\-C\-K uses several third-\/party tools and libraries. The selected packages are widely adopted by open-\/source communities and are generally available on H\-P\-C systems via module. They are included in standard Linux/cygwin distributions or can be installed by standard tools like yum. Installing these libraries with the source codes is straightforward. Because the header files are included by Q\-M\-C\-P\-A\-C\-K, it is important to install developers’ version of each library. If these libraries are installed in standard directories, /usr /usr/local and /sw (Mac), no action is necessary.

Alternatively, environment variables X\-Y\-Z\-\_\-\-H\-O\-M\-E should be set. Here, X\-Y\-Z stands for the name of package; the build utility can locate the libraries and use them. With few exceptions, the build utility cmake will look for {\ttfamily X\-Y\-Z\-\_\-\-H\-O\-M\-E/include} for the header files and {\ttfamily X\-Y\-Z\-\_\-\-H\-O\-M\-E/lib} for the library files. When multiple environment variables apply to a library, e.\-g., blas/lapack, the library is searched according to the listed order.

\begin{TabularC}{3}
\hline
\rowcolor{lightgray}{\bf Name }&{\bf Environment variables }&{\bf Comments  }\\\cline{1-3}
blas/lapack&{\ttfamily M\-K\-L\-\_\-\-H\-O\-M\-E}, {\ttfamily L\-A\-P\-A\-C\-K}, {\ttfamily A\-T\-L\-A\-S}&Alternatives\-: vendor-\/provided blas, e.\-g., E\-S\-S\-L \\\cline{1-3}
hdf5 &{\ttfamily H\-D\-F5\-\_\-\-H\-O\-M\-E}, {\ttfamily H\-D\-F\-\_\-\-H\-O\-M\-E} &phdf5 is not necessary \\\cline{1-3}
libxml2 &{\ttfamily L\-I\-B\-X\-M\-L2\-\_\-\-H\-O\-M\-E} &Configure with {\ttfamily --disable-\/shared --enable-\/static --without-\/python --without-\/http --without-\/ftp} \\\cline{1-3}
boost &{\ttfamily B\-O\-O\-S\-T\-\_\-\-H\-O\-M\-E} &Using only the header files. No need to compile the library. Simply download and unpack the package, if not installed on the system. \\\cline{1-3}
fftw &{\ttfamily F\-F\-T\-W\-\_\-\-H\-O\-M\-E} &double precision only \\\cline{1-3}
\end{TabularC}
