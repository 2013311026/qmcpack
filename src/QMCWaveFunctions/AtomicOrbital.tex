\documentclass{article}
\usepackage{amsmath}

\author{Kenneth P. Esler, Jr.}
\title{Hybrid orbital representation}
\date{\today}

\begin{document}
\newcommand{\vr}{\mathbf{r}}
\maketitle
\begin{equation}
\phi(\vr) = \sum_{\ell=0}^{\ell_\text{max}} \sum_{m=-\ell}^\ell Y_\ell^m (\hat{\Omega})
u_{\ell m}(r),
\end{equation}
where $u_{lm}(r)$ are complex radial functions represented in some
radial basis (e.g. splines).

\section{Real Spherical Harmonics}
If $\phi(\vr)$ can be written as purely real, we can change the
representation so that
\begin{equation}
\phi(\vr) = \sum_{l=0}^{l_\text{max}} \sum_{m=-\ell}^\ell Y_{\ell m}(\hat{\Omega})
\bar{u}_{lm}(r),
\end{equation}
where $\bar{Y}_\ell^m$ are the {\em real} spherical harmonics defined by
\begin{equation}
Y_{\ell m} = \begin{cases}
Y_\ell^0 & \mbox{if } m=0\\
{1\over\sqrt2}\left(Y_\ell^m+(-1)^m \, Y_\ell^{-m}\right) = \sqrt{2} N_{(\ell,m)} P_\ell^m(\cos \theta) \cos m\varphi 
& \mbox{if } m>0 \\
{1\over i\sqrt2}\left(Y_\ell^{-m}-(-1)^{m}\, Y_\ell^{m}\right) = \sqrt{2} N_{(\ell,m)} P_\ell^{-m}(\cos \theta) \sin m\varphi 
&\mbox{if } m<0.
\end{cases}
\end{equation}
We need then to relate $\bar{u}_{\ell m}$ to $u_{\ell m}$.


\end{document}
