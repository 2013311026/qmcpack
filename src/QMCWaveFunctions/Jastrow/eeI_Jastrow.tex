\documentclass{article}
\usepackage{amsmath}
\author{Ken Esler}
\title{Electron-electron-ion Jastrow factor}
\date{\today}
\begin{document}
\maketitle
\newcommand{\riI}{r_{iI}}
\newcommand{\briI}{\mathbf{r}_{iI}}
\newcommand{\rjI}{r_{jI}}
\newcommand{\brjI}{\mathbf{r}_{jI}}
\newcommand{\rij}{r_{ij}}
\newcommand{\brij}{\mathbf{r}_{ij}}
\section{Form of the Jastrow}
The general form of the 3-body Jastrow we describe here depends on the
three inter-particle distances, $(\rij, \riI, \rjI)$.
\begin{equation}
J_3 = \sum_{I\in\text{ions}} \sum_{i,j \in\text{elecs};i\neq j} U(\rij, \riI,
\rjI)
\end{equation}
Note that we constrain the form of $U$ such that
$U(\rij, \riI,\rjI) = U(\rij, \rjI,\riI)$, so as to preserve the
particle symmetry of the wave function.  We then compute the gradient as
\begin{equation}
\nabla_i J_3 =  \sum_{I\in\text{ions}} \sum_{j \neq i}
\left[\frac{\partial U(\rij, \riI,\rjI)}{\partial\rij}
  \frac{\mathbf{r}_i - \mathbf{r}_j}{|\mathbf{r}_i - \mathbf{r}_j|} 
+ \frac{\partial U(\rij, \riI,\rjI)}{\partial\riI}
  \frac{\mathbf{r}_i - \mathbf{I}}{|\mathbf{r}_i - \mathbf{I}|}  \right]
\end{equation}
To compute the laplacian, we take
\begin{eqnarray}
\nabla_i^2 J_3 & = & \nabla_i \cdot \left(\nabla_i J_3\right) \\
& = & \sum_{I\in\text{ions}} \sum_{j\neq i } \left[
\frac{\partial^2 U}{\partial \rij^2} + \frac{2}{\rij} \frac{\partial
  U}{\partial \rij} + 2 \frac{\partial^2 U}{\partial \rij \partial
  \riI}\frac{\brij\cdot\briI}{\rij\riI} +\frac{\partial^2 U}{\partial
  \riI^2}
+ \frac{2}{\riI}\frac{\partial U}{\partial \riI} \nonumber
\right]
\end{eqnarray}
\end{document}
