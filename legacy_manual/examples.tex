\chapter{Examples}
\label{chap:examples}

\textbf{WARNING: THESE EXAMPLES ARE NOT CONVERGED! YOU MUST CONVERGE PARAMETERS (SIMULATION CELL SIZE, JASTROW PARAMETER NUMBER/CUTOFF, TWIST NUMBER, DMC TIME STEP, DFT PLANE WAVE CUTOFF, DFT K-POINT MESH, ETC.) FOR REAL CALCUATIONS!}

The following examples should run in serial on a modern workstation in a few hours.

\section{Using QMCPACK directly}

In \ishell{examples/molecules} are the following examples.
Each directory also contains a \ishell{README} file with more details.

\begin{tabular}{l l}
Directory  & Description \\
\ishell{H2O} &  H2O molecule from GAMESS orbitals \\
\ishell{He} &  Helium atom with simple wavefunctions\\
\end{tabular}




\section{Using Nexus}

For more information about Nexus, see the User Guide in \ishell{nexus/documentation}.

For Python to find the Nexus library, the PYTHONPATH environment variable should be set to \ishell{<QMCPACK source>/nexus/library}.
For these examples to work properly, the executables for QE and QMCPACK either
need to be on the path, or the paths in the script should be adjusted.

These examples can be found under the \ishell{nexus/examples/qmcpack} directory.
%\begin{itemize}
%\item \ishell{diamond} Bulk diamond with VMC
%\item \ishell{graphene} Graphene sheet with DMC
%\item \ishell{c20} C20 cage molecule
%\item \ishell{oxygen\_dimer} Binding curve for O$_2$ molecule
%\item \ishell{H2O} H$_2$O molecule with Quantum ESPRESSO orbitals
%\item \ishell{LiH} LiH crystal with Quantum ESPRESSO orbitals
%\end{itemize}

\begin{tabular}{l l}
Directory  & Description \\
\ishell{diamond} &  Bulk diamond with VMC \\
\ishell{graphene} & Graphene sheet with DMC \\
\ishell{c20} & C20 cage molecule \\
\ishell{oxygen\_dimer} & Binding curve for O$_2$ molecule \\
\ishell{H2O} & H$_2$O molecule with QE orbitals \\
\ishell{LiH} & LiH crystal with QE orbitals \\
\end{tabular}





%\subsection{Bulk diamond}
%The input files are in the directory \ishell{nexus/examples/qmcpack/diamond}.

%\subsection{Graphene Sheet}
%The input files are in the directory \ishell{nexus/examples/qmcpack/graphene}.

%\subsection{C20 cage molecule}
%The input files are in the directory \ishell{nexus/examples/qmcpack/c20}.

%\subsection{Binding curve for O$_2$ molecule}
%The input files are in the directory \ishell{nexus/examples/qmcpack/oxygen\_dimer}.

%\subsection{H$_2$O molecule with Quantum ESPRESSO orbitals}
%
%The input files are in the directory \ishell{nexus/examples/qmcpack/H2O}.

%The Nexus script is in \ishell{H2O.py} and \ishell{H2O.xyz} contains the atomic positions.
%To run the example, the BFD pseudopotentials are needed.  Create a \ishell{pseudopotentials} directory, and copy \ishell{O.BFD.upf}, \ishell{O.BFD.xml}, \ishell{H.BFD.upf}, and \ishell{H.BFD.xml} from \ishell{pseudopotentials/BFD} in the QMCPACK distribution.
%
%The \ishell{H2O.py} script generates the orbitals using Quantum ESPRESSO, runs QMCPACK to optimize the Jastrow and then performs DMC for H$_2$O in a box.
%
%
%
%\subsection{LiH crystal with Quantum ESPRESSO orbitals}
%The input files are in the directory \ishell{nexus/examples/qmcpack/LiH}.
%
%The \ishell{LiH.py} Nexus script uses CASINO-formatted Trail-Needs pseudopotentials (see Section~\ref{subsec:CASINO}) for Li and H in a subdirectory named \ishell{pseudopotentials} (both UPF and CASINO  formats, named \ishell{Li.TN-DF.upf}, \ishell{Li.pp.data}, \ishell{H.TN-DF.upf}, and \ishell{H.pp.data}, respectively), generates the orbitals using Quantum ESPRESSO, then runs QMCPACK to optimize the Jastrow and run DMC for LiH with periodic boundary conditions.
