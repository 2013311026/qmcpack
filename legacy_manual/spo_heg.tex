\subsection{Homogeneous electron gas}
\label{sec:hegbasis}

The interacting Fermi liquid has its own special \ixml{determinantset} for filling up a
Fermi surface.  The shell number can be specified separately for both spin-up and spin-down.
This determines how many electrons to include of each time; only closed shells are currently
implemented.  The shells are filled according to the rules of a square box; if other lattice
vectors are used, the electrons might not fill up a complete shell.

This following example can also be used for Helium simulations by specifying the
proper pair interaction in the Hamiltonian section. 

\begin{lstlisting}[style=QMCPXML,caption=2D Fermi liquid example: particle specification ]
<qmcsystem>
<simulationcell name="global">
<parameter name="rs" pol="0" condition="74">6.5</parameter>
<parameter name="bconds">p p p</parameter>
<parameter name="LR_dim_cutoff">15</parameter>
</simulationcell>
<particleset name="e" random="yes">
<group name="u" size="37">
<parameter name="charge">-1</parameter>
<parameter name="mass">1</parameter>
</group>
<group name="d" size="37">
<parameter name="charge">-1</parameter>
<parameter name="mass">1</parameter>
</group>
</particleset>
</qmcsystem>
\end{lstlisting}

\begin{lstlisting}[style=QMCPXML,caption=2D Fermi liquid example (Slater Jastrow wavefunction) ]
<qmcsystem>
  <wavefunction name="psi0" target="e">
      <determinantset type="electron-gas" shell="7" shell2="7" randomize="true">
  </determinantset>
      <jastrow name="J2" type="Two-Body" function="Bspline" print="no">
        <correlation speciesA="u" speciesB="u" size="8" cusp="0">
          <coefficients id="uu" type="Array" optimize="yes"> 
        </correlation>
        <correlation speciesA="u" speciesB="d" size="8" cusp="0">
          <coefficients id="ud" type="Array" optimize="yes"> 
        </correlation>
      </jastrow>
\end{lstlisting}
