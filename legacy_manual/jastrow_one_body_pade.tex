\subsubsection{Pade form}
\label{sec:onebodyjastrowpade}

Although the spline Jastrow factor is the most flexible and most commonly used form implemented in \qmcpack, 
there are times where its flexibility can make it difficult to optimize.  As an example, a spline Jastrow
with a very large cutoff can be difficult to optimize for isolated systems such as molecules because of the small
number of samples present in the tail of the function.  In such cases, a simpler functional
form might be advantageous.  The second-order Pade Jastrow factor, given in Equation~\ref{padeeqn}, is a good choice 
in such cases.  
\begin{equation}
\label{padeeqn}
u_{ab}(r) = \frac{a*r+c*r^2}{1+b*r} .
\end{equation}
Unlike the spline Jastrow factor, which includes a cutoff, this form has an infinite range and will be applied to every particle
pair (subject to the minimum image convention).  It also is a cuspless Jastrow factor,
so it should be used either in combination with a single particle basis set that contains the proper cusp or
with a smooth pseudopotential.

\paragraph{Input specification}
\begin{table}[h]
\begin{center}
\begin{tabular}{l c c c l }
\hline
\multicolumn{5}{l}{Correlation element} \\
\hline
\bfseries Name & \bfseries Datatype & \bfseries Values & \bfseries Defaults & \bfseries Description \\
\hline
ElementType & Text & Name & See below & Classical particle target  \\
\hline
\multicolumn{5}{l}{Elements}\\ \hline
& Coefficients & & & \\ \hline
\multicolumn{5}{l}{Contents}\\ \hline
& (None)  & & &  \\ \hline
\end{tabular}
%\end{tabular*}
\end{center}
\end{table}

\begin{table}[h]
\begin{center}
\begin{tabular}{l c c c l }
\hline
\multicolumn{5}{l}{Parameter element} \\
\hline
\bfseries Name & \bfseries Datatype & \bfseries Values & \bfseries Defaults & \bfseries Description \\
\hline
Id & String & Name & (Required) & Name for variable \\
Name & String & A or B or C & (Required) & See Equation~\ref{padeeqn}\\
Optimize & Text & Yes or no & Yes & If no, values are fixed in optimizations \\
\hline
\multicolumn{5}{l}{Elements}\\ \hline
(None) & & & \\ \hline
\multicolumn{5}{l}{Contents}\\ \hline
 (No name) & Real & Parameter value & (Required) & Jastrow coefficients \\ \hline
\end{tabular}
%\end{tabular*}
\end{center}
\end{table}

\paragraph{Example use case}
\label{sec:1bjpadeexamples}

Specify a spin-independent function with independent Jastrow factors for two different species (Li and H).
The name of the particleset holding the ionic positions is ``i."
\begin{lstlisting}[style=QMCPXML]
<jastrow name="J1" function="pade2" type="One-Body" print="yes" source="i">
  <correlation elementType="Li">
    <var id="LiA" name="A">  0.34 </var>
    <var id="LiB" name="B"> 12.78 </var>
    <var id="LiC" name="C">  1.62 </var>
  </correlation>
  <correlation elementType="H"">
    <var id="HA" name="A">  0.14 </var>
    <var id="HB" name="B"> 6.88 </var>
    <var id="HC" name="C"> 0.237 </var>
  </correlation>
</jastrow>
\end{lstlisting}
