\section{Specifying the simulation cell}
\label{chap:simulationcell}

The \ixml{simulationcell} block specifies the geometry of the cell, how the boundary conditions should be handled, and how ewald summation should be broken up.

\begin{table}[h]
\begin{center}
\begin{tabularx}{\textwidth}{l l l l l X }
\hline
\multicolumn{6}{l}{\texttt{simulationcell} element} \\
\hline
\multicolumn{2}{l}{parent elements:} & \multicolumn{4}{l}{\texttt{qmcsystem}}\\
\multicolumn{2}{l}{child  elements:} & \multicolumn{4}{l}{None}\\
\multicolumn{2}{l}{attribute      :} & \multicolumn{4}{l}{}\\
   &   \bfseries parameter name            & \bfseries datatype & \bfseries values & \bfseries default   & \bfseries description \\
\hline
   &   \texttt{lattice}  & 9 floats & any float & Must be specified & Specification of \\
   &                     &        &             &                   & lattice vectors. \\
   &   \texttt{bconds}   & string & ``p'' or ``n''  & ``n n n'' & Boundary conditions \\
   &                     &        &             &           & for each axis. \\
   &   \texttt{vacuum} & float & $\ge 1.0$ & 1.0        & Vacuum scale. \\
   &   \texttt{LR\_dim\_cutoff} & float & float & 15        & Ewald breakup distance. \\
   &   \texttt{LR\_tol} & float & float & 3e-4        & Tolerance in Ha for Ewald ion-ion energy per atom. \\
\hline
\end{tabularx}
\end{center}
\end{table}

An example of a \ixml{simulationcell} block is given below:
\begin{lstlisting}[style=QMCPXML]
  <simulationcell>
    <parameter name="lattice">
      3.8       0.0       0.0
      0.0       3.8       0.0
      0.0       0.0       3.8
    </parameter>
    <parameter name="bconds">
       p p p
    </parameter>
    <parameter name="LR_dim_cutoff"> 20 </parameter>
  </simulationcell>
\end{lstlisting}

Here, a cubic cell 3.8 bohr on a side will be used.
This simulation will use periodic boundary conditions, and the maximum
$k$ vector will be $20/r_{wigner-seitz}$ of the cell.


\subsection{Lattice}
The cell is specified using 3 lattice vectors.


\subsection{Boundary conditions}
QMCPACK offers the capability to use a mixture of open and periodic boundary conditions.
The \ixml{bconds} parameter expects a single string of three characters separated by
spaces, \textit{e.g.} ``p p p'' for purely periodic boundary conditions. These characters control
the behavior of the $x$, $y$, and $z$, axes, respectively. Non periodic directions must be placed after the periodic ones.
Examples of valid \ixml{bconds} include:

\begin{description}
\item[``p p p''] Periodic boundary conditions. Corresponds to a 3D crystal.
\item[``p p n''] Slab geometry. Corresponds to a 2D crystal.
\item[``p n n''] Wire geometry. Corresponds to a 1D crystal.
\item[``n n n''] Open boundary conditions. Corresponds to an isolated molecule in a vacuum.
\end{description}

\subsection{Vacuum}
The vacuum option allows adding a vacuum region in slab or wire boundary conditions
(\ixml{bconds= p p n} or \ixml{bconds= p n n}, respectively). The main use is
to save memory with spline or plane-wave basis trial wavefunctions, because no basis
functions are required inside the vacuum region. For example, a large vacuum region
can be added above and below a graphene sheet without having to generate the trial
wavefunction in such a large box or to have as many splines as would otherwise
be required. Note that the trial wavefunction must still be generated in a
large enough box to sufficiently reduce periodic interactions in the underlying
electronic structure calculation.

With the vacuum option, the box used for Ewald summation increases along the axis labeled \ixml{n} by a factor of \ixml{vacuum}.
Note that all the particles remain in the original box without altering their positions. i.e. Bond lengths are not changed by this option.
The default value is 1, no change to the specified axes.

An example of a \ixml{simulationcell} block using \ixml{vacuum} is given below.
The size of the box along the z-axis increases from 12 to 18 by the vacuum scale of 1.5.
\begin{lstlisting}[style=QMCPXML]
  <simulationcell>
    <parameter name="lattice">
      3.8       0.0       0.0
      0.0       3.8       0.0
      0.0       0.0      12.0
    </parameter>
    <parameter name="bconds">
       p p n
    </parameter>
    <parameter name="vacuum"> 1.5 </parameter>
    <parameter name="LR_dim_cutoff"> 20 </parameter>
  </simulationcell>
\end{lstlisting}

\subsection{LR\_dim\_cutoff}
When using periodic boundary conditions direct calculation of the Coulomb energy is
not well behaved. As a result, QMCPACK uses an optimized Ewald summation technique
to compute the Coulomb interaction.\cite{Natoli1995}

In the Ewald summation, the energy is broken into short- and long-ranged terms.
The short-ranged term is computed directly in real space, while the long-ranged term is computed in reciprocal space.
\ixml{LR_dim_cutoff} controls where the short-ranged term ends and the long-ranged term begins.
The real-space cutoff, reciprocal-space cutoff, and \ixml{LR_dim_cutoff} are related via:
\begin{equation}
\mathrm{LR\_dim\_cutoff} = r_{c} \times k_{c}
\end{equation}
where $r_{c}$ is the Wigner-Seitz radius, and $k_{c}$ is the length of the maximum $k$-vector used in the long-ranged term. Larger
values of \ixml{LR_dim_cutoff} increase the accuracy of the evaluation. A value of 15 tends to be conservative.
