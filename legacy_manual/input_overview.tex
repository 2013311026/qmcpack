\chapter{Input file overview}
\label{chap:input_overview}

This chapter introduces XML as it is used in the QMCPACK input file.  The focus is on the XML file format itself and the general structure of the input file rather than an exhaustive discussion of all keywords and structure elements.  

QMCPACK uses XML to represent structured data in its input file.  Instead of text blocks like

\begin{shade}
begin project
  id     = vmc
  series = 0
end project

begin vmc
  move     = pbyp
  blocks   = 200
  steps    =  10
  timestep = 0.4
end vmc
\end{shade} 
QMCPACK input looks like
\begin{lstlisting}[style=QMCPXML]
   <project id="vmc" series="0">
   </project>

   <qmc method="vmc" move="pbyp">
      <parameter name="blocks"  >  200 </parameter>
      <parameter name="steps"   >   10 </parameter>
      <parameter name="timestep">  0.4 </parameter>
   </qmc>
\end{lstlisting}
XML elements start with \ixml{<element\_name>}, end with \ixml{</element\_name>}, and can be nested within each other to denote substructure (the trial wavefunction is composed of a Slater determinant and a Jastrow factor, which are each further composed of \ldots).  \ixml{id} and \ixml{series} are attributes of the \ixml{<project/>} element.  XML attributes are generally used to represent simple values, like names, integers, or real values.  Similar functionality is also commonly provided by \ixml{<parameter/>} elements like those previously shown.

The overall structure of the input file reflects different aspects of the QMC simulation: the simulation cell, particles, trial wavefunction, Hamiltonian, and QMC run parameters.  A condensed version of the actual input file is shown as follows:
\begin{lstlisting}[style=QMCPXML]
<?xml version="1.0"?>
<simulation>

  <project id="vmc" series="0">
    ...
  </project>

  <qmcsystem>

    <simulationcell>
      ...
    </simulationcell>

    <particleset name="e">
      ...
    </particleset>

    <particleset name="ion0">
      ...
    </particleset>

    <wavefunction name="psi0" ... >
      ...
      <determinantset>
        <slaterdeterminant>
          ..
        </slaterdeterminant>
      </determinantset>
      <jastrow type="One-Body" ... >
         ...
      </jastrow>
      <jastrow type="Two-Body" ... >
        ...
      </jastrow>
    </wavefunction>

    <hamiltonian name="h0" ... >
      <pairpot type="coulomb" name="ElecElec" ... />
      <pairpot type="coulomb" name="IonIon"   ... />
      <pairpot type="pseudo" name="PseudoPot" ... >
        ...
      </pairpot>
    </hamiltonian>

   </qmcsystem>

   <qmc method="vmc" move="pbyp">
     <parameter name="warmupSteps">   20 </parameter>
     <parameter name="blocks"     >  200 </parameter>
     <parameter name="steps"      >   10 </parameter>
     <parameter name="timestep"   >  0.4 </parameter>
   </qmc>

</simulation>
\end{lstlisting}
The omitted portions (\texttt{...}) are more fine-grained inputs such as the axes of the simulation cell, the number of up and down electrons, positions of atomic species, external orbital files, starting Jastrow parameters, and external pseudopotential files.  


\section{Project}
The \ixml{<project>} tag uses the \ixml{id} and \ixml{series} attributes.
The value of \ixml{id} is the first part of the prefix for output file names.

Output file names also contain the series number, starting at the value given by the
\ixml{series} tag.  After every \ixml{<qmc>} section, the series value will increment, giving each section a unique prefix.

For the input file shown previously, the output files will start with \ishell{vmc.s000}, for example, \ishell{vmc.s000.scalar.dat}.
If there were another \ixml{<qmc>} section in the input file, the corresponding output files would use the prefix \ishell{vmc.s001}.



\section{Random number initialization}

The random number generator state is initialized from the \ixml{random} element using the \ixml{seed} attribute:
\begin{lstlisting}[style=QMCPXML]
<random seed="1000"/>
\end{lstlisting}

If the random element is not present, or the seed value is negative, the seed will be generated from the current time.

To initialize the many independent random number generators (one per thread and MPI process), the seed value is used (modulo 1024) as a starting index into a list of prime numbers.
Entries in this offset list of prime numbers are then used as the seed for the random generator on each thread and process.

If checkpointing is enabled, the random number state is written to an HDF file at the end of each block (suffix: \ishell{.random.h5}).
This file will be read if the \ixml{mcwalkerset} tag is present to perform a restart.
For more information, see the \ixml{checkpoint} element in the QMC methods Chapter~\ref{chap:qmcmethods} and Section~\ref{sec:checkpoint_files} on checkpoint and restart files.
