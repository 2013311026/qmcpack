\chapter{Some Examples}
\section{Li${}_2$ STO}
\section{Bulk solid}
Solids can be simulated in periodic boundary conditions using orbitals from a plane-wave DFT code such as ABINIT, PWscf (Quantum Espresso), or Qbox.  We will explain the process of generating orbitals, converting them to QMCPACK's HDF5 format, and setting up the input files. As an example, we will use cubic boron nitride, an analog of diamond in the zincblende structure.

\subsection*{Tools for exercises}
\begin{itemize}
  \item{} wfconv
  \begin{itemize}
    \item{} wfconv is a tool for converting orbitals in various plane-wave formats to the real-space 3D mesh format used by QMCPACK. QMCPACK reads files stored in a platform-independent binary file format known as HDF5 (http://hdfgroup.org).
    \item{} A 64-bit static linux binary can be downloaded at wfconv.
    \item{} Contact Ken Esler (esler@uiuc.edu) if you would like the source code. 
  \end{itemize}
  \item{} ppconvert
  \begin{itemize}
    \item{} ppconvert is a tool for converting pseudopotentials between various formats.
    \item{} A 64-bit static linux binary can be downloaded at ppconvert. 
  \end{itemize}
\end{itemize}

After downloading these tools, do \icode{chmod u+x wfconv ppconvert} to mark the files as executable.

\subsection*{Convert the pseudopotentials}
[check this, perhaps also explaining what each line in the file means]

Copy the following pseudopotential in GAMESS format into a file \iterm{B.BFD.gamess}:
\begin{code}
B-QMC GEN 2 1
3
3.00000000 1 5.40423964
16.21271892 3 5.71678458
-11.86640633 2 4.48974455
1
15.49737620 2 3.43781634
\end{code}

Likewise, the following into \iterm{N.BFD.gamess}:
\begin{code}
N-QMC GEN 2 1
3
5.00000000 1 9.23501007
46.17505034 3 7.66830008
-30.18893534 2 7.34486070
1
31.69720409 2 6.99536540
\end{code}

We will now convert the pseudopotentials into the FHI format used by ABINIT and the FSatom XML format used by QMCPACK. Put ppconvert into a directory in your PATH. Then execute
\begin{code}
ppconvert --gamess_pot B.BFD.gamess --s_ref "1s(2)2p(1)" --p_ref "1s(2)2p(1)" \
          --fhi B.BFD.fhi --xml B.BFD.fsatom.xml
ppconvert --gamess_pot N.BFD.gamess --s_ref "1s(2)2p(3)" --p_ref "1s(2)2p(3)" \
          --fhi N.BFD.fhi --xml N.BFD.fsatom.xml
\end{code}
\begin{itemize}
  \item{} The first argument given is the input pseudopotential file.
  \item{} The second and third arguments give the reference state for forming Kleinmann-Bylander projectors.
  \begin{itemize}
    \item{} Note this state specifies the reference for the valence electrons only, e.g. the neutral state for the valence electrons in a boron atom.
    \item{} The last two arguments specify output file formats. 
  \end{itemize}
\end{itemize}

\subsection*{Generating orbitals with ABINIT}
ABINIT (http://www.abinit.org) is a general-purpose plane-wave DFT code which supports pseudopotential and PAW calculations.  It is well-documented, full-featured, and has a vibrant community support forum.

We will begin with a primitive cell of c-BN. Copy the following into \iterm{cBNprim.in}:
\codesrc{cBNprim.in}

Copy the following into \iterm{cBNprim.files}:
\begin{code}
cBNprim.in
cBNprim.out
cBNprim.xi
cBNprim.xo
cBNprim_
B.BFD.fhi
N.BFD.fhi
\end{code}

Now, run
\begin{term}
abinis < cBNprim.files
\end{term}

\subsection*{Converting the orbitals}
With wfconv in your PATH, run
\begin{term}
wfconv --eshdf cBNprim.h5 cBNprim.xo_WFK
\end{term}

This will generated an orbital file in the ESHDF format that QMCPACK reads.

N.B. The GPU version of QMCPACK uses an older format for the orbital file. To generate orbitals for the GPU code, do instead
\begin{term}
wfconv --spline --qmcPACK cBNprim.h5 cBNprim.xo_WFK
\end{term}

The newer CPU code also can read this format, but we are trying to deprecate it. We intend to remerge the GPU and CPU versions in the near future. 

\subsection*{Running QMCPACK}
Copy the following to \iterm{cBNprim.qmc.xml}:
\codesrc{cBNprim.qmc.xml}

More from\\
\url{http://cms.mcc.uiuc.edu/qmcpack/index.php/Bulk_solid_calculations_with_DFT_orbitals}\\
\url{http://qmcpack.cmscc.org/tutorials/bulk-solid}
\section{Liquid helium}
VMC and DMC simulations of liquid helium-4 is a fairly standard benchmark for any QMC code because of its long history~\cite{PhysRev.138.A442,PhysRevA.9.2178}.  This is the same for QMCPACK in that helium atoms are bosonic and have masses greater than 1 (the electron mass in atomic units).  These are features distinct from most of the other simulations done by QMCPACK, which use electrons as fundamental particles.  Since its first success with the Lennard-Jones (LJ) 6-12 pair potential and two-body Jastrow correlations of the form $\exp[-b/r^a]$, there have been many improvements to their functional forms.  There are four types of commonly used pair potentials which are implemented in QMCPACK: LJ, HFD-HE2~\cite{Aziz1979}, HFD-B(HE)~\cite{Aziz1987}, and one based on symmetry-adapted perturbation theory~\cite{Jeziorska2007}.

Shown below is a typical input file which runs VMC first, then DMC.
\codesrc{hfdhe2-sample.xml}

\section{3D homogeneous electron gas}
Reviews on homogeneous electron gas can be found in standard solid-state physics or electronic structure textbooks.  A particularly relevant discussion is in Ref.~\citenum{Martin2003} Ch.~5.  The system is fully specified by only two parameters: $r_s = \left( \cfrac{3}{4 \pi \rho} \right)^{1/3} \equiv \mathtt{rs}$ and the shell number.  For a momentum eigenvalue $\mathbf{k}$, the shell number is defined by mutually exclusive sets of Fermi surfaces\footnote{Note that the Fermi surface is spherical only in the case of noninteracting fermions.  Interaction causes a slight deviation from a perfect sphere.} in $\mathbf{k}$-space which contain the same number of allowed $\mathbf{k}$ states.  The simulation cell in this case is defined with periodic boundary conditions (PBC), resulting in a discrete set of allowed $\mathbf{k}$ values.  The smallest Fermi surface containing more than one $\mathbf{k}$ state contains 7 $\mathbf{k}$ states which include $\mathbf{k} = \mathbf{0}$ and its 6 nearest neighbors.  The next smallest set of Fermi surfaces (``shells'') contain 19 $\mathbf{k}$ and the numbers continue in a sequence of 27, 33, 57, 81, etc.

%Ideally, one only has to set rs and the shell number to set the problem. However, the current implementation requires that the parameters associated with the number of particles have to be set consistently.
%Other parameters to set the number of particles have to be provided explicitly.
QMCPACK currently supports only closed-shell systems, meaning that the allowed numbers of electrons go as $N = 14, 38, 54, 66, 114, 162, \cdots$ since each $\mathbf{k}$ state is occupied by a pair of electrons with opposite spins.  The shells associated with each $N$ are assigned numbers 1, 2, 3, $\cdots$ respectively.  The current implementation also does not recognize the electron gas system as a special case that requires only \icode{rs} and the shell number as the parameters.  Thus, we need to state the number of particles explicitly, taking caution that $N$ corresponds to the correct shell number.  QMCPACK usually will not complain when there is a mismatch unless it causes a memory overflow.

%Key parameters \icode{} sets rs of the problem. condition is used to set the number of particles at the given rs.
The path of the key parameters of the simulation (XPath) are as follows.
\begin{code}
simulationcell/parameter/@name='rs'
particleset/group/@size : number of particles per spin
determinantset/@shell
\end{code}

Shown below is a typical form of the input file.
\codesrc{3dheg.s000.xml}
\section{Spherical systems}
\subsection{Spherical jellium}
%  Specific Example Calculations
%
