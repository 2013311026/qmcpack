\subsection{convert4qmc}
convert4qmc allows conversion of orbitals and wavefunctions from quantum chemistry output files to QMCPACK xml input files.  It is a small C++ executable that is built alongside the qmcpack executable, and can be found in \texttt{trunk/build/bin}.
convert4qmc is a tool that evolves continuously and the number of software supported keeps growing.\\
To this dates, convert4qmc supports the following codes and methods as  implemented in the following codes: Gamess\cite{schmidt93}, Pyscf\cite{Sun2018}, Quantum Package\cite{QP} and Gamess-FMO\cite{Fedorov2004,schmidt93}


\subsubsection{General use}
A general usage of convert4qmc can always be pormpted  by:

\begin{shade}
>convert4qmc

 convert [-gaussian|-casino|-gamesxml|-gamess|-gamessFMO|-VSVB|-QP|-pyscf]
 filename                                                          
[-nojastrow -hdf5 -prefix title -addCusp -production]                                                                                           
[-psi_tag psi0 -ion_tag ion0 -gridtype log|log0|linear -first ri -last rf]
[-size npts -ci file.out -threshold cimin -TargetState state_number
-NaturalOrbitals NumToRead]                                        
Defaults : -gridtype log -first 1e-6 -last 100 -size 1001 -ci required 
-threshold 0.01 -TargetState 0 -prefix sample                                
When the input format is missing, the  extension of filename is used to determine
the format                                                      
 *.Fchk -> gaussian; *.out -> gamess; *.data -> casino; *.xml -> gamesxml
\end{shade}


As an example, to convert from a gamess calculation using a single determinant,the following usage is sufficient:\\
\begin{shade}
convert4qmc -gamess MyGamessOutput.out
\end{shade}

By default, the converter will generate multiple files:\\
\begin{table}[h]
\begin{center}
\begin{tabularx}{\textwidth}{l l l l l }
\hline
\multicolumn{5}{l}{\texttt{convert4qmc} output} \\
\hline
\multicolumn{2}{l}{Outputfiles}  & \multicolumn{3}{l}{}\\
   &   \bfseries output     & \bfseries file type & \bfseries default   & \bfseries description \\
   &   \texttt{*.qmc.in-wfs.xml             } &  XML  & default& Main input file for QMCPACK\\
   &   \texttt{*.qmc.in-wfnoj.xml             } &  XML  & default& Main input file for QMCPACK\\
   &   \texttt{*.structure.xml             } &  XML   &default   & file containing the structure of the system\\
   &   \texttt{*.wfj.xml             } &  XML  & default & Wavefunction file with 1, 2 and 3 body Jastrows\\
   &   \texttt{*.wfnoj.xml             } &  XML   & default & Wavefunction file with no Jastrows. \\
   &   \texttt{*.orbs.h5             } &  HDF5   & with tag -hdf5   & HDF5 file containing all wavefunction data\\
    \hline
    \end{tabularx}
\end{center}
\end{table}

If nothing is specified, the name of the generated files will be the same as the one provided in the input. For instance, if the Gamess output file is \textbf{Mysim}.out the files generated by convert4qmc will use the prefix \textbf{Mysim} and files will be Mysim.qmc.in-wfs.xml, Mysim.structure.xml and so on...

\begin{itemize}
 \item Files \textbf{.in-wfs.xml} and \textbf{.in-wfnoj.xml} \\
 These are the input files for qmcpack. Instead of containing the geometry and the wavefunction, the call 2 external files ``*.structure.xml'' and ``*.wfj.xml`` when .in-wfs.xml is used or ''*.wfnoj.xml'' when *. qmc.in-wfnoj.xml. The Hamiltonian section is also included and the presence or not of an ECP is detected during the convertion. By default, a default ECP name is added (ex: H.qmcpp.xml) and it is the respoinsability of the user to modify the ECP to match the one used in the quantum chemistry run.\\
  \begin{shade}
  <?xml version="1.0"?>
<simulation>
  <!--
 
Example QMCPACK input file produced by convert4qmc
 
It is recommend to start with only the initial VMC block and adjust
parameters based on the measured energies, variance, and statistics.

-->
  <!--Name and Series number of the project.-->
  <project id="gms" series="0"/>
  <!--Link to the location of the Atomic Coordinates and the location of 
      the Wavefunction.-->
  <include href="gms.structure.xml"/>
  <include href="gms.wfnoj.xml"/>
  <!--Hamiltonian of the system. Default ECP filenames are assumed.-->
  <hamiltonian name="h0" type="generic" target="e">
    <pairpot name="ElecElec" type="coulomb" source="e" target="e" 
                                                   physical="true"/>
    <pairpot name="IonIon" type="coulomb" source="ion0" target="ion0"/>
    <pairpot name="PseudoPot" type="pseudo" source="ion0" wavefunction="psi0" 
                                                           format="xml">
      <pseudo elementType="H" href="H.qmcpp.xml"/>
      <pseudo elementType="Li" href="Li.qmcpp.xml"/>
    </pairpot>
  </hamiltonian>

 \end{shade}

 The qmc.in-wfnoj.xml file will have one VMC block with a minimum number of blocks allowing to reproduce the HF/DFT energy used to generate the trial wavefunction.
 
 \begin{shade}
  <qmc method="vmc" move="pbyp" checkpoint="-1">
    <estimator name="LocalEnergy" hdf5="no"/>
    <parameter name="warmupSteps">100</parameter>
    <parameter name="blocks">20</parameter>
    <parameter name="steps">50</parameter>
    <parameter name="substeps">8</parameter>
    <parameter name="timestep">0.5</parameter>
    <parameter name="usedrift">no</parameter>
  </qmc>
</simulation>
 \end{shade}
If the qmc.in-wfj.xml file is used, Jastrow optimization blocks followed by a VMC and DMC block are generated. These blocks contain default value allowing the user to test the accuracy of a system, however, they need to be optimized for each system. 

\begin{shade}
  <loop max="4">
    <qmc method="linear" move="pbyp" checkpoint="-1">
      <estimator name="LocalEnergy" hdf5="no"/>
      <parameter name="warmupSteps">100</parameter>
      <parameter name="blocks">20</parameter>
      <parameter name="timestep">0.5</parameter>
      <parameter name="walkers">1</parameter>
      <parameter name="samples">16000</parameter>
      <parameter name="substeps">4</parameter>
      <parameter name="usedrift">no</parameter>
      <parameter name="MinMethod">OneShiftOnly</parameter>
      <parameter name="minwalkers">0.0001</parameter>
    </qmc>
  </loop>
  <!--
 
Example follow-up VMC optimization using more samples for greater accuracy

-->
  <loop max="10">
    <qmc method="linear" move="pbyp" checkpoint="-1">
      <estimator name="LocalEnergy" hdf5="no"/>
      <parameter name="warmupSteps">100</parameter>
      <parameter name="blocks">20</parameter>
      <parameter name="timestep">0.5</parameter>
      <parameter name="walkers">1</parameter>
      <parameter name="samples">64000</parameter>
      <parameter name="substeps">4</parameter>
      <parameter name="usedrift">no</parameter>
      <parameter name="MinMethod">OneShiftOnly</parameter>
      <parameter name="minwalkers">0.3</parameter>
    </qmc>
  </loop>
  <!--

Production VMC and DMC

Examine the results of the optimization before running these blocks.
e.g. Choose the best optimized jastrow from all obtained, put in 
wavefunction file, do not reoptimize.

-->
  <qmc method="vmc" move="pbyp" checkpoint="-1">
    <estimator name="LocalEnergy" hdf5="no"/>
    <parameter name="warmupSteps">100</parameter>
    <parameter name="blocks">200</parameter>
    <parameter name="steps">50</parameter>
    <parameter name="substeps">8</parameter>
    <parameter name="timestep">0.5</parameter>
    <parameter name="usedrift">no</parameter>
    <!--Sample count should match targetwalker count for 
      DMC. Will be obtained from all nodes.-->
    <parameter name="samples">16000</parameter>
  </qmc>
  <qmc method="dmc" move="pbyp" checkpoint="20">
    <estimator name="LocalEnergy" hdf5="no"/>
    <parameter name="targetwalkers">16000</parameter>
    <parameter name="reconfiguration">no</parameter>
    <parameter name="warmupSteps">100</parameter>
    <parameter name="timestep">0.005</parameter>
    <parameter name="steps">100</parameter>
    <parameter name="blocks">100</parameter>
    <parameter name="nonlocalmoves">yes</parameter>
  </qmc>
</simulation>

\end{shade}

 \item File \textbf{structure.xml} \\
 This file will be called by the main qmcpack input.It contains the geometry of the system, position of the atoms, number of atoms, types, chargeand number of electrons.
 \item structure.xml \\
 
 \item \textbf{.wfj.xml} and file \textbf{.wfnoj.xml}\\
 These file contain the basiset detail, orbital coefficients and the 1,2 3 body Jastrow in the case of (  \textbf{.wfj.xml}). If the wavefunction is multideterminant, the expansion will be at the end of the file. It is recommended to use the tag \textbf{-hdf5} when large molecules are studied to avoid writing the data in xml format and store it instead in an HDF5 file. 
 
 \item File \textbf{.orbs.h5}
 This file is only generated if the tag \textbf{-hdf5} is added as follow:
 \begin{shade}
  convert4qmc -gamess MyGamessOutput.out -hdf5
 \end{shade}
In this case,  \textbf{.wfj.xml} or \textbf{.wfnoj.xml} will point to this H5 file which contains the information about the basiset, orbitals coefficients and the multiderminant expansion. These information will be therefore removed from from the wavefunction files making them lighter. 

\end{itemize}


Convert4qmc supports multiple codes. Some of them are actively maintained (gamess, pyscf, quantum package), and others that require more testing. Detailed usage per code will be detailed bellow. The full list of supported code is listed in the table below.

\begin{table}[h]
\begin{center}
\begin{tabularx}{\textwidth}{l l l l l l }
\hline
\multicolumn{6}{l}{\texttt{convert4qmc} keywords} \\
\hline
\multicolumn{2}{l}{parameters}  & \multicolumn{4}{l}{}\\
   &   \bfseries keyword     & \bfseries datatype & \bfseries values & \bfseries default   & \bfseries description \\
   &   \texttt{gamess             } &  tag  & -gamess & -   & tag to convert from the Gamess code\\
   &   \texttt{pyscf             } &  tag  & -pyscf & -   & tag to convert from the pyscf code\\
   &   \texttt{QP             } &  tag  & -QP & -   & tag to convert from the Quantum Package code\\
   &   \texttt{gamesFMO             } &  tag  & -gamessFMO & -   & tag to convert from the Gamess FMO code\\
   &   \texttt{gaussian             } &  tag  & -gaussian & -   & tag to convert from the Gaussian code (obsolete) \\
   &   \texttt{casino             } &  tag  & -casino & -   & tag to convert from the Casino code (untested)\\
   &   \texttt{gamesxml             } &  tag  & -gamesxml & -   & tag to convert from the Gamess xml format code\\ &&&&&  (obsolete/untested)\\
   &   \texttt{VSVB             } &  tag  & -VSVB & -   & tag to convert from the ArgoVB code for \\ &&&&&variational valence bond  (in hold until newer\\ &&&&& version of the code)\\
    \hline

    \end{tabularx}
\end{center}
\end{table}

\subsubsection{Special tags}

[-nojastrow -hdf5 -prefix title -addCusp -production]                                                                                           
[-psi_tag psi0 -ion_tag ion0 -gridtype log|log0|linear -first ri -last rf]
[-size npts -ci file.out -threshold cimin -TargetState state_number
-NaturalOrbitals NumToRead]                                        
Defaults : -gridtype log -first 1e-6 -last 100 -size 1001 -ci required 
-threshold 0.01 -TargetState 0 -prefix sample                                
When the input format is missing, the  extension of filename is used to determine
the format                                                      

\begin{table}[h]
\begin{center}
\begin{tabularx}{\textwidth}{l l l l  l }
\hline
\multicolumn{5}{l}{\texttt{convert4qmc} keywords} \\
\hline
\multicolumn{2}{l}{parameters}  & \multicolumn{3}{l}{}\\
   &   \bfseries keyword      & \bfseries Value & \bfseries default   & \bfseries description \\
   &   \texttt{-nojastrow            } &  -  & - & Force no jastrow. qmc.in.wfj will not be generated  \\
   &   \texttt{-hdf5             } &  -  & - & Force the wf to be in HDF5 format   \\
   &   \texttt{-prefix             } &  string  & - & all created files will have the name of the string   \\
   &   \texttt{-addCusp             } &  -  & - & Force to add orbital cusp correction (ONLY for all electrons)  \\
   &   \texttt{-production             } &  -  & - & generates specific blocks in the input     \\
   &   \texttt{-psi_tag             } &  string  & psi0 & name of the electrons particles inside QMCPACK   \\
   &   \texttt{-ion_tag             } & string & ion0 & name of the ion particles inside QMCPACK      \\
    \hline

    \end{tabularx}
\end{center}
\end{table}


\subsubsection{Gamess}
