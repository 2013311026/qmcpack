\chapter{Quantum Monte Carlo Methods}
\label{chap:qmcmethods}

\begin{table}[h]
\begin{center}
\begin{tabularx}{\textwidth}{l l l l l l }
\hline
\multicolumn{6}{l}{\texttt{qmc} factory element} \\
\hline
\multicolumn{2}{l}{parent elements:} & \multicolumn{4}{l}{\texttt{simulation, loop}}\\
\multicolumn{2}{l}{type   selector:} & \multicolumn{4}{l}{\texttt{method} attribute}\\
\multicolumn{2}{l}{type   options: } & vmc           & \multicolumn{3}{l}{Variational Monte Carlo}\\
%\multicolumn{2}{l}{                } & opt           & \multicolumn{3}{l}{}\\
\multicolumn{2}{l}{                } & linear        & \multicolumn{3}{l}{Wavefunction optimization with linear method}\\
%\multicolumn{2}{l}{                } & cslinear      & \multicolumn{3}{l}{}\\
\multicolumn{2}{l}{                } & dmc           & \multicolumn{3}{l}{Diffusion Monte Carlo}\\
\multicolumn{2}{l}{                } & rmc           & \multicolumn{3}{l}{Reptation Monte Carlo}\\
%\multicolumn{2}{l}{                } & ptcl          & \multicolumn{3}{l}{}\\
%\multicolumn{2}{l}{                } & mul           & \multicolumn{3}{l}{}\\
%\multicolumn{2}{l}{                } & warp          & \multicolumn{3}{l}{}\\
\multicolumn{2}{l}{shared attributes:} & \multicolumn{4}{l}{}\\
   &   \bfseries name         & \bfseries datatype & \bfseries values & \bfseries default & \bfseries description \\
   &   \texttt{method}        &  text              &   listed above   & invalid           & QMC driver            \\
   &   \texttt{move}          &  text              &   pbyp, alle     & pbyp              & method used to move electrons \\
   &   \texttt{gpu}           &  text              &   yes, no        & dep.              & use the GPU\\
   &   \texttt{trace}         &  text              &                  & no                & ???                      \\
   &   \texttt{checkpoints}   &  integer           &   -1, 0, n       & -1                & checkpoint frequency \\
   &   \texttt{dumpconfig}    &  integer           &   n              & -1                & dump configuration every n step  \\
   &   \texttt{target}        &  text              &                  &                   & ???  \\
   &   \texttt{completed}     &  text              &                  &                   & ???  \\
   &   \texttt{append}        &  text              &   yes, no        & yes               & ???  \\
%   &   \texttt{multiple}      &  text              &   yes, no        & no                & ???  \\
%   &   \texttt{warp}          &  text              &   yes, no        & no                & ???  \\
\hline

\end{tabularx}
\end{center}
\end{table}

Additional information:
\begin{itemize}
\item \texttt{move}. There are two ways implemented to move electrons. The more used method is the particle-by-particle move. In this method, only one electron is moved for acception or rejection. The other method is the all-electron move, namely all the electrons are moved once for testing acception or rejection.

\item \texttt{gpu}. When the executable is compiled with CUDA, the target computing device can be chosen by this switch. With a regular CPU only compilation, this option is not effective.

\item \texttt{checkpoints}. If Checkpoint=``-1'' no checkpoint will be done (default setting). If Checkpoint=``0'' dump after the completion of a qmc section. When dumconfig=``n'' is present with Checkpoint=``0'', the configurations will be dumped at the end of the run (due to Checkpoint=``0''), but also at every n block. If Checkpoint=``n'', configurations will be dump every n block. 

All the dumped data will be written in a *.config.h5. The config.h5 file will contain the state of a population to continue a run including the random number sequences; the list of what is included in the .config.h5 is: number of walkers, status of the run, branch mode, energy dataset, ratio to accepted moves, ratio to proposed moves, variance dataset, vParam\{tau, taueff. E\_trial, E\_ref, Branch\_Max, BranchCutOff, BranchFilter, Sigma, Accepted\_Energy, Accepted\_Samples\}, IParam{warmumSteps, Energy\_Update\_Interval, Counter, targetwalkers, Maxwalkers, MinWalkers, Branching Interval}, Walker coordinates, Random number size, Random number sequence, version of the code.

\begin{lstlisting}[caption=The following is an example of running a simulation that can be restarted . ]
  <qmc method="dmc" move="pbyp"  checkpoint="0" dumpconfig="5">
    <parameter name="timestep">         0.004  </parameter>
    <parameter name="blocks">           100   </parameter>
    <parameter name="steps">            400    </parameter>
  </qmc>
\end{lstlisting}

In this case, the there will be 
The flags checkpoint and dumpconfig instructs qmcpack to output walker configurations.  This also
works in variational Monte Carlo.  This will output an h5 file with the name "projectid"."run-number".config.h5.
Check that this file exists before attempting a restart.

To continue a run, specify the following before your VM/DMC block:
\begin{lstlisting}[caption=Restart (read wakers from previous run) ]
 <mcwalkerset fileroot="BH.s002" version="0 6" collected="yes"/>
  <qmc method="dmc" move="pbyp"  checkpoint="0" dumpconfig="5">
    <parameter name="timestep">         0.004  </parameter>
    <parameter name="blocks">           100   </parameter>
    <parameter name="steps">            400    </parameter>
  </qmc>
\end{lstlisting}
where, BH is the project id and s002 is the calculation number to read in the walkers from the previous run.

In the project id section, make sure that the series number is different than any existing one to avoid rewriting on it. 

\end{itemize}
