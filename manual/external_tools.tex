\chapter{External Tools}
\label{chap:external_tools}
This chapter provides some information on using QMCPACK with external tools.

\section{LLVM Sanitizer Libraries}\label{tool:LLVM-Sanitizer-Libraries}

Using cmake set one of these flags for using the clang sanitizer libraries with or without lldb.

\begin{shade}
-DLLVM_SANITIZE_ADDRESS    link with the %*\href{https://clang.llvm.org/docs/AddressSanitizer.html}{Clang address sanitizer library}*
-DLLVM_SANITIZE_MEMORY     link with the %*\href{https://clang.llvm.org/docs/MemorySanitizer.html}{Clang memory sanitizer library}*
\end{shade}

These set the basic flags required to build with either of these sanitizer libraries. They require a build of clang with dynamic libraries somehow visible, i.e. through \inlinecode{LD_FLAGS=-L/your/path/to/llvm/lib}. You must link through clang, which is generally the default when building with it. Depending on your system and linker this may be incompatible with the ``Release'' build so set \inlinecode{-DCMAKE_BUILD_TYPE=Debug}. They have been tested with the default spack install of llvm@7.0.0 and manually built llvm 7.0.1. See the above links for additional information on use, runtime and build options of the sanitizers.

In general the address sanitizer libraries will catch most pointer based errors. ASAN can also catch memory links but requires additional options be set. MSAN will catch more subtle memory management errors but is difficult to use without a full set of MSAN instrumented libraries.

\section{Intel VTune}

Intel's VTune profiler has an API that allows program control over collection (pause/resume) and can add information to the profile data (e.g. delineating tasks).

\subsection{VTune API}

If the variable \texttt{USE\_VTUNE\_API} is set, QMCPACK will check that the
include file (\texttt{ittnotify.h}) and the library (\texttt{libittnotify.a}) can
be found.
To provide CMake with the VTune paths, add the include path to \texttt{CMAKE\_CXX\_FLAGS} and the the library path to \texttt{CMAKE\_LIBRARY\_PATH}.

An example of options to be passed to CMake
\begin{shade}
 -DCMAKE_CXX_FLAGS=-I/opt/intel/vtune_amplifier_xe/include \
 -DCMAKE_LIBRARY_PATH=/opt/intel/vtune_amplifier_xe/lib64
\end{shade}

\section{NVIDIA Tools Extensions (NVTX)}

NVIDIA's Tools Extensions (NVTX) API enables programmers to annotate their source code when used with the NVIDIA profilers.

\subsection{NVTX API}

If the variable \texttt{USE\_NVTX\_API} is set, QMCPACK will add the library (\texttt{libnvToolsExt.so}) to the qmcpack target. To add NVTX annotations
to a function, it is necessary to include the \texttt{nvToolsExt.h} header file and then make the appropriate calls into the NVTX API. For more information
about the NVTX API, see \url{https://docs.nvidia.com/cuda/profiler-users-guide/index.html#nvtx}. Any additional calls to the NVTX API should be guarded by
the \texttt{USE\_NVTX\_API} compiler define.

\subsection{Timers as Tasks}
To aid in connecting the timers in the code to the profile data, the start/stop of
timers will be recorded as a task if \texttt{USE\_VTUNE\_TASKS} is set.

In addition to compiling with \texttt{USE\_VTUNE\_TASKS}, an option needs to be set at runtime to collect the task API data.
In the GUI, select the checkbox labeled "Analyze user tasks" when setting up the analysis type.
For the command line, set the \texttt{enable-user-tasks} knob to \texttt{true}. For example,
\begin{shade}
amplxe-cl -collect hotspots -knob enable-user-tasks=true ...
\end{shade}

Collection with the timers set at "fine" can generate too much task data in the profile.
Collection with the timers at "medium" collects a more reasonable amount of task data.

\section{Scitools Understand}

Scitools Understand (\url{https://scitools.com/}) is a tool for static
code analysis. The easiest configuration route is to use the JSON output
from CMake which the Understand project importer can read directly:
\begin{enumerate}
\item Configure QMCPACK by running cmake with
  CMAKE\_EXPORT\_COMPILE\_COMMANDS=ON, e.g.
  \texttt{ cmake -DCMAKE\_C\_COMPILER=clang -DCMAKE\_CXX\_COMPILER=clang++ \\
    -DQMC\_MPI=0 -DCMAKE\_EXPORT\_COMPILE\_COMMANDS=ON ../qmcpack/ }
\item Run Understand and create a new C++ project. At the import files
  and settings dialog, import the compile\_commands.json created by
  cmake in the build directory.  
\end{enumerate}
