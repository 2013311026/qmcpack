\subsubsection{Spline form}
\label{sec:twobodyjastrowspline}

The two-body spline Jastrow function is the most commonly used two-body Jastrow for solids. This form 
was first described and used in \cite{EslerKimCeperleyShulenburger2012}.  
Here $u_{ab}$ is an interpolating 1D Bspline (tricublc spline on a linear grid) between 
zero distance and $r_{cut}$. In 3D periodic systems 
the default cutoff distance is the Wigner Seitz cell radius. For other periodicities including isolated 
molecules the $r_{cut}$ must be specified.  $r_i$ and $r_j$ are typically electron positions.  The cusp 
condition as $r_i$ approaches $r_j$ is set by the relative spin of the electrons.

\FloatBarrier
\paragraph{Input Specification}

\FloatBarrier
\begin{table}[h!]
\begin{center}
\begin{tabular}{l c c c l }
\hline
\multicolumn{5}{l}{Correlation element} \\
\hline
\bfseries name & \bfseries datatype & \bfseries values & \bfseries defaults & \bfseries description \\
\hline
speciesA & text & u or d & (required) & Quantum species target \\
speciesB & text & u or d & (required) & Quantum species target \\
size & integer & $> 0$ & (required) & number of coefficients \\
rcut & real & $> 0$ & see below & distance at which the correlation goes to 0 \\
spin & text & yes or no & no & spin dependent jastrow factor \\
\hline
\multicolumn{5}{l}{elements}\\ \hline
& Coefficients & & & \\ \hline
\multicolumn{5}{l}{Contents}\\ \hline
& (None)  & & &  \\ \hline
\end{tabular}
%\end{tabular*}
\end{center}
\end{table}

\FloatBarrier

Additional information:
\begin{itemize}
\item \texttt{speciesA, speciesB} The scale function u(r) is defined for species pairs uu and ud.  
There is no need to define ud or dd since uu=dd and ud=du.  The cusp condition is computed internally 
based on the charge of the quantum particles.
\end{itemize}

\begin{table}[h]
\begin{center}
\begin{tabular}{l c c c l }
\hline
\multicolumn{5}{l}{Coefficients element} \\
\hline
\bfseries name & \bfseries datatype & \bfseries values & \bfseries defaults & \bfseries description \\
\hline
id & text & & (required) & Unique identifier \\
type & text & Array & (required) & \\
optimize & text & yes or no & yes & if no, values are fixed in optimizations \\
\hline
\multicolumn{5}{l}{elements}\\ \hline
(None) & & & \\ \hline
\multicolumn{5}{l}{Contents}\\ \hline
 (no name) & real array & & zeros & Jastrow coefficients \\ \hline
\end{tabular}
%\end{tabular*}
\end{center}
\end{table}

\paragraph{Example use cases}
\label{sec:2bjsplineexamples}

Specify a spin-dependent function with 4 parameters for each channel.  In this case, the cusp is set at 
a radius of 4.0 bohr (rather than to the default of the Wigner Seitz cell radius).  Also, in this example,
the coefficients are set to not be optimized during an optimization step.

\begin{lstlisting}[style=XML]
<jastrow name="J2" type="Two-Body" function="Bspline" print="yes">
  <correlation speciesA="u" speciesB="u" size="8" rcut="4.0">
    <coefficients id="uu" type="Array" optimize="no"> 0.2309049836 0.1312646071 0.05464141356 0.01306231516</coefficients>
  </correlation>
  <correlation speciesA="u" speciesB="d" size="8" rcut="4.0">
    <coefficients id="ud" type="Array" optimize="no"> 0.4351561096 0.2377951747 0.1129144262 0.0356789236</coefficients>
  </correlation>
</jastrow>
\end{lstlisting}
