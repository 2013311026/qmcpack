\chapter{Derivation of twist averaging efficiency}
\label{sec:app_ta_efficiency}
In this appendix we derive the relative statistical efficiency of 
twist averaging with an irreducible (weighted) set of k-points 
versus using uniform weights over an unreduced set of k-points 
(e.g., a full Monkhorst-Pack mesh).

Consider the weighted average of a set of statistical variables 
$\{x_m\}$ with weights $\{w_m\}$:
\begin{align}
  x_{TA} = \frac{\sum_mw_mx_m}{\sum_mw_m}\:.
\end{align} 
If produced by a finite QMC run at a set of 
twist angles/k-points $\{k_m\}$, each variable mean $\mean{x_m}$ 
has a statistical error bar $\sigma_m$, and we can also obtain 
the statistical error bar of the mean of the twist-averaged 
quantity $\mean{x_{TA}}$:
\begin{align}
  \sigma_{TA} = \frac{\left(\sum_mw_m^2\sigma_m^2\right)^{1/2}}{\sum_mw_m}\:.
\end{align}
The error bar of each individual twist $\sigma_m$ is related to the 
autocorrelation time $\kappa_m$,  intrinsic variance $v_m$, and the number 
of postequilibration MC steps $N_{step}$ in the following way:
\begin{align}
  \sigma_m^2=\frac{\kappa_mv_m}{N_{step}}\:.
\end{align}
In the setting of twist averaging, the autocorrelation time and 
variance for different twist angles are often very similar across 
twists, and we have
\begin{align}
  \sigma_m^2=\sigma^2=\frac{\kappa v}{N_{step}}\:.
\end{align} 
If we define the total weight as $W$, that is, $W\equiv\sum_{m=1}^Mw_m$, 
for the weighted case with $M$ irreducible twists, the error bar is
\begin{align}
  \sigma_{TA}^{weighted}=\frac{\left(\sum_{m=1}^Mw_m^2\right)^{1/2}}{W}\sigma\:.
\end{align}
For uniform weighting with $w_m=1$, the number of twists is $W$ and 
we have
\begin{align}
  \sigma_{TA}^{uniform}=\frac{1}{\sqrt{W}}\sigma\:.
\end{align}
We are interested in comparing the efficiency of choosing weights 
uniformly or based on the irreducible multiplicity of each twist angle 
for a given target error bar $\sigma_{target}$.  The number of MC  
steps required to reach this target for uniform weighting is
\begin{align}
  N_{step}^{uniform} = \frac{1}{W}\frac{\kappa v}{\sigma_{target}^2}\:,
\end{align}
while for nonuniform weighting we have
\begin{align}\label{eq:weighted_step}
  N_{step}^{weighted} &= \frac{\sum_{m=1}^Mw_m^2}{W^2}\frac{\kappa v}{\sigma_{target}^2} \nonumber\:,\\
                  &=\frac{\sum_{m=1}^Mw_m^2}{W}N_{step}^{uniform}\:.
\end{align}
The MC efficiency is defined as 
\begin{align}
  \xi = \frac{1}{\sigma^2t}\:,
\end{align}
where $\sigma$ is the error bar and $t$ is the total CPU time required 
for the MC run.  

The main advantage made possible by irreducible twist weighting is to 
reduce the equilibration time overhead by having fewer twists and, 
hence, fewer MC runs to equilibrate.  In the context of twist 
averaging, the total CPU time for a run can be considered to be
\begin{align}
  t=N_{twist}(N_{eq}+N_{step})t_{step}\:,
\end{align}
where $N_{twist}$ is the number of twists, $N_{eq}$ is the number of MC steps required to reach equilibrium, $N_{step}$ is the number 
of MC steps included in the statistical averaging as before, 
and $t_{step}$ is the wall clock time required to complete a single 
MC step. For uniform weighting $N_{twist}=W$; while for irreducible 
weighting $N_{twist}=M$.

We can now calculate the relative efficiency ($\eta$) of irreducible vs. 
uniform twist weighting with the aim of obtaining a target error bar 
$\sigma_{target}$:
\begin{align}
  \eta &= \frac{\xi_{TA}^{weighted}}{\xi_{TA}^{uniform}} \nonumber\:, \\
       &= \frac{\sigma_{target}^2t_{TA}^{uniform}}{\sigma_{target}^2t_{TA}^{weighted}} \nonumber\:, \\
       &= \frac{W(N_{eq}+N_{step}^{uniform})}{M(N_{eq}+N_{step}^{weighted})} \nonumber\:, \\
       &= \frac{W(N_{eq}+N_{step}^{uniform})}{M(N_{eq}+\frac{\sum_{m=1}^Mw_m^2}{W}N_{step}^{uniform})} \nonumber\:, \\
       &= \frac{W}{M}\frac{1+f}{1+\frac{\sum_{m=1}^Mw_m^2}{W}f}\:.
\end{align}
In this last expression, $f$ is the ratio of the number of usable 
MC steps to the number that must be discarded during equilibration 
($f=N_{step}^{uniform}/N_{eq}$); and as before, $W=\sum_mw_m$, which is the number of 
twist angles in the uniform weighting case.  It is important to recall 
that $N_{step}^{uniform}$ in $f$ is defined relative to uniform weighting and is 
the number of MC steps required to reach a target accuracy in the 
case of uniform twist weights.

The formula for $\eta$ in the preceding can be easily changed with the help of 
Equation~\ref{eq:weighted_step} to reflect the 
number of MC steps obtained in an irreducibly weighted run 
instead.  A good exercise is to consider runs that have already completed 
with either uniform or irreducible weighting and calculate the 
expected efficiency change had the opposite type of weighting been used.

The break even point $(\eta=1)$ can be found at a usable step fraction of 
\begin{align}
  f=\frac{W-M}{M\frac{\sum_{m=1}^Mw_m^2}{W}-W}\:.
\end{align}

The relative efficiency $(\eta)$ is useful to consider in view of certain 
scenarios.  An important case is where the number of required sampling 
steps is no larger than the number of equilibration steps (i.e., 
$f\approx 1$).  For a very simple case with eight uniform twists with 
irreducible multiplicities of $w_m\in\{1,3,3,1\}$ ($W=8$, $M=4$), the 
relative efficiency of irreducible vs. uniform weighting is 
$\eta=\frac{8}{4}\frac{2}{1+20/8}\approx 1.14$.  In this case, 
irreducible weighting is about $14$\% more efficient than uniform weighting.

Another interesting case is one in which the number of sampling steps you can 
reach with uniform twists before wall clock time runs out is small 
relative to the number of equilibration steps ($f\rightarrow 0$). 
In this limit, $\eta\approx W/M$.  For our eight-uniform-twist example, this would 
result in a relative efficiency of $\eta=8/4=2$, making irreducible 
weighting twice as efficient.

A final case of interest is one in which the equilibration time is short 
relative to the available sampling time $(f\rightarrow\infty)$, 
giving $\eta\approx W^2/(M\sum_{m=1}^Mw_m^2)$.  Again, for our simple example 
we find $\eta=8^2/(4\times 20)\approx 0.8$, with uniform weighting being 
$25$\% more efficient than irreducible weighting. For this example, the crossover point for irreducible weighting being 
more efficient than uniform weighting is $f<2$, that is, when the 
available sampling period is less than twice the length of the equilibration 
period.  The expected efficiency ratio and crossover point should be checked 
for the particular case under consideration to inform the choice between   
twist averaging methods.



