\section{Variational Monte Carlo}
\label{sec:vmc}

\begin{table}[h]
\begin{tabularx}{\textwidth}{l l l l l X }
\hline
\multicolumn{6}{l}{\texttt{vmc} method} \\
\hline
\multicolumn{2}{l}{parameters}  & \multicolumn{4}{l}{}\\
   &   \bfseries name     & \bfseries datatype & \bfseries values & \bfseries default   & \bfseries description \\
   &   \texttt{walkers             } &  integer  & $> 0$   & dep.& Number of walkers per MPI task  \\
   &   \texttt{blocks              } &  integer  & $\ge 0$ & 1   & Number of blocks            \\
   &   \texttt{steps               } &  integer  & $\ge 0$ & 1   & Number of steps per block   \\
   &   \texttt{warmupsteps         } &  integer  & $\ge 0$ & 0   & Number of steps for warming up\\
   &   \texttt{substeps            } &  integer  & $\ge 0$ & 1   & Number of substeps per step \\
   &   \texttt{usedrift            } &  text     & yes, no & yes  & Use the algorithm with drift\\
   &   \texttt{timestep            } &  real     & $> 0$   & 0.1 & Time step for each electron move \\
   &   \texttt{samples             } &  integer  & $\ge 0$ & 0   & Number of walker samples for DMC/optimization\\
   &   \texttt{stepsbetweensamples } &  integer  & $> 0$   & 1   & Period of sample accumulation\\
   &   \texttt{samplesperthread    } &  integer  & $\ge 0$ & 0   & Number of samples per thread  \\
   &   \texttt{storeconfigs        } &  integer  & all values & 0   & Store configurations o  \\
   &   \texttt{blocks\_between\_recompute} &  integer  & $\ge 0$ & dep.  & Wavefunction recompute frequency  \\
  \hline
\end{tabularx}
\end{table}

Additional information:
\begin{itemize}
\item \ixml{walkers}: The number of walkers per MPI task. The initial default number of \ixml{walkers} is one per OpenMP thread or per MPI task if threading is disabled. The number is rounded down to a multiple of the number of threads with a minimum of one per thread to ensure perfect load balancing. One walker per thread is created in the event fewer \ixml{walkers} than threads are requested. 

\item \ixml{blocks}: This parameter is universal for all the QMC
  methods. The MC processes are divided into a number of
  \ixml{blocks}, each containing a number of steps. At the end of each block,
  the statistics accumulated in the block are dumped into files,
  e.g., \ixml{scalar.dat}. Typically, each block should have a sufficient number of steps that the I/O at the end of each block is negligible
  compared with the computational cost. Each block should not take so
  long that monitoring its progress is difficult. There should be a
  sufficient number of \ixml{blocks} to perform statistical analysis.

\item \ixml{warmupsteps}: \ixml{warmupsteps} are used only for
  equilibration. Property measurements are not performed during
  warm-up steps.

\item \ixml{steps}: \ixml{steps} are the number of energy and other property measurements to perform per block.
  
\item \ixml{substeps}: For each substep, an attempt is made to move each of the electrons once only by either particle-by-particle or an
  all-electron move.  Because the local energy is evaluated only at
  each full step and not each substep, \ixml{substeps} are computationally cheaper
  and can be used to reduce the correlation between property measurements
  at a lower cost.
  
\item \ixml{usedrift}: The VMC is implemented in two algorithms with
  or without drift. In the no-drift algorithm, the move of each
  electron is proposed with a Gaussian distribution. The standard
  deviation is chosen as the time step input. In the drift algorithm,
  electrons are moved by Langevin dynamics.

\item \ixml{timestep}: The meaning of time step depends on whether or not
  the drift is used. In general, larger time steps reduce the
  time correlation but might also reduce the acceptance ratio,
  reducing overall statistical efficiency. For VMC, typically the
  acceptance ratio should be close to 50\% for an efficient
  simulation.

\item \ixml{samples}: Seperate from conventional energy and other
  property measurements, samples refers to storing whole electron
  configurations in memory (``walker samples'') as would be needed by subsequent
  wavefunction optimization or DMC steps. \textit{A standard VMC run to
  measure the energy does not need samples to be set.}

\[
\texttt{samples}=
\frac{\texttt{blocks}\cdot\texttt{steps}\cdot\texttt{walkers}}{\texttt{stepsbetweensamples}}\cdot\texttt{number of MPI tasks}
\]

\item \ixml{samplesperthread}: This is an alternative way to set the target amount of samples and can be useful when preparing a stored population for a subsequent DMC calculation.
\[
\texttt{samplesperthread}=
\frac{\texttt{blocks}\cdot\texttt{steps}}{\texttt{stepsbetweensamples}}
\]

\item \ixml{stepsbetweensamples}: Because samples generated by consecutive steps are correlated, having \ixml{stepsbetweensamples} larger than 1 can be used to reduces that correlation. In practice, using larger substeps is cheaper than using \ixml{stepsbetweensamples} to decorrelate samples. 

\item \ixml{storeconfigs}: If \ixml{storeconfigs} is set to a nonzero value, then electron configurations during the VMC run are saved to files.

\item \ixml{blocks_between_recompute}: Recompute the accuracy critical determinant part of the wavefunction
  from scratch: =1 by default when using mixed precision. =0 (no
  recompute) by default when not using mixed precision. Recomputing
  introduces a performance penalty dependent on system size.
\end{itemize}

An example VMC section for a simple VMC run:
\begin{lstlisting}[style=QMCPXML]
  <qmc method="vmc" move="pbyp">
    <estimator name="LocalEnergy" hdf5="no"/>
    <parameter name="walkers">    256 </parameter>
    <parameter name="warmupSteps">  100 </parameter>
    <parameter name="substeps">  5 </parameter>
    <parameter name="blocks">  20 </parameter>
    <parameter name="steps">  100 </parameter>
    <parameter name="timestep">  1.0 </parameter>
    <parameter name="usedrift">   yes </parameter>
  </qmc>
\end{lstlisting}
Here we set 256 \ixml{walkers} per MPI, have a brief initial equilibration of 100 \ixml{steps}, and then have 20 \ixml{blocks} of 100 \ixml{steps} with 5 \ixml{substeps} each.

The following is an example of VMC section storing configurations (walker samples) for optimization.
\begin{lstlisting}[style=QMCPXML]
  <qmc method="vmc" move="pbyp" gpu="yes">
    <estimator name="LocalEnergy" hdf5="no"/>
    <parameter name="walkers">    256 </parameter>
    <parameter name="samples">    2867200 </parameter>
    <parameter name="stepsbetweensamples">    1 </parameter>
    <parameter name="substeps">  5 </parameter>
    <parameter name="warmupSteps">  5 </parameter>
    <parameter name="blocks">  70 </parameter>
    <parameter name="timestep">  1.0 </parameter>
    <parameter name="usedrift">   no </parameter>
  </qmc>
\end{lstlisting}



