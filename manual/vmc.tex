\section{Variational Monte Carlo}
\label{sec:vmc}

\begin{table}[h]
\begin{center}
\begin{tabularx}{\textwidth}{l l l l l l }
\hline
\multicolumn{6}{l}{\texttt{vmc} method} \\
\hline
\multicolumn{2}{l}{parameters}  & \multicolumn{4}{l}{}\\
   &   \bfseries name     & \bfseries datatype & \bfseries values & \bfseries default   & \bfseries description \\
   &   \texttt{walkers             } &  integer  & $> 0$   & dep.& number of walkers per MPI task  \\
   &   \texttt{blocks              } &  integer  & $\ge 0$ & 1   & number of blocks            \\
   &   \texttt{steps               } &  integer  & $\ge 0$ & 1   & number of steps per block   \\
   &   \texttt{warmupsteps         } &  integer  & $\ge 0$ & 0   & number of steps for warming up\\
   &   \texttt{substeps            } &  integer  & $\ge 0$ & 1   & number of substeps per step \\
   &   \texttt{usedrift            } &  text     & yes, no & no  & use the algorithm with drift\\
   &   \texttt{timestep            } &  real     & $> 0$   & 0.1 & time step for each electron move \\
   &   \texttt{samples             } &  integer  & $\ge 0$ & 0   & number of walker samples for DMC/optimization\\
   &   \texttt{stepsbetweensamples } &  integer  & $> 0$   & 1   & period of sample accumulation\\
   &   \texttt{samplesperthread    } &  integer  & $\ge 0$ & 0   & number of samples per thread  \\
   &   \texttt{storeconfigs        } &  integer  & all values & 0   & store configurations o  \\
   &   \texttt{blocks\_between\_recompute} &  integer  & $\ge 0$ & dep.  & wavefunction recompute frequency  \\
  \hline
\end{tabularx}
\end{center}
\end{table}

Additional information:
\begin{itemize}
\item \ixml{walkers}. The initial default number of walkers is one per OpenMP thread or per MPI task if threading is disabled, with a minimum of one per thread. One walker per thread will be created in the event fewer walkers than threads are requested. 

\item \ixml{blocks}. This parameter is universal for all the QMC
  methods. The Monte Carlo processes are divided into a number of
  blocks each containing a number of steps. At the end of each block,
  all the statistics accumulated in the block is dumped in to files,
  e.g.\ scalar.dat. Typically each block should have sufficient number
  of steps that the I/O at the end of each block is negligible
  compared to the computational cost. Each block should not take so
  long that it is difficult to monitor progress. There should be a
  sufficient number of blocks to perform statistical analysis.

\item \ixml{warmupsteps}. Warm-up steps are steps used only for
  equilibration. Property measurements are not performed during
  warm-up steps.

\item \ixml{steps}. The number of energy and other property measurements to perform per block.
  
\item \ixml{substeps}. For each substep, each of the electrons is
  attempted to be moved only once by either particle-by-particle or
  all-electron move.  Because the local energy is only evaluated at
  each full step and not substep, substeps are computationally cheaper
  and can be used to reduce correlation between property measurements
  at lower cost.
  
\item \ixml{usedrift}. The VMC is implemented in two algorithms with
  or without drift. In the no-drift algorithm, the move of each
  electron is proposed with a Gaussian distribution. The standard
  deviation is chosen as the timestep input. In the drift algorithm,
  electrons are moved by langevin dynamics.

\item \ixml{timestep}. The meaning of timestep depends on whether
  the drift is used or not. In general, larger timesteps reduce the
  time correlation but might also reduce the acceptance ratio,
  reducing overall statistical efficiency. For VMC, typically the
  acceptance ratio should be close to 50\% for an efficient
  simulation.

\item \ixml{samples}. Seperate from conventional energy and other
  property measurements, samples refers to storing whole electron
  configurations in memory (``walker samples'') as would be needed by subsequent
  wavefunction optimization or DMC steps. \textbf{A standard VMC run to
  measure the energy does not need samples to be set.}

\[
\textrm{samples}=
\frac{\textrm{blocks}\cdot\textrm{steps}\cdot\textrm{walkers}}{\textrm{stepsbetweensamples}}\cdot\textrm{number of MPI tasks}
\]

\item \ixml{samplesperthread}. This is an alternative way to set the target amount of samples, and can be useful when preparing a stored population for a subsequent DMC calculation.
\[
\textrm{samplesperthread}=
\frac{\textrm{blocks}\cdot\textrm{steps}}{\textrm{stepsbetweensamples}}
\]

\item \ixml{stepsbetweensamples}. Due to the fact that samples generated by consecutive steps are correlated, having stepsbetweensamples larger than 1 can be used to reduces that correlation. In practice, using larger substeps is cheaper than using stepsbetweensamples to decorrelate samples. 

\item \ixml{storeconfigs}. If storeconfigs is set to a non-zero value, then electron configurations during the VMC run will be saved to files.

\item \ixml{blocks_between_recompute}. Recompute the accuracy critical determinant part of the wavefunction
  from scratch. =1 by default when using mixed precision. =0 (no
  recompute) by default when not using mixed precision. Recomputing
  introduces a performance penalty dependent on system size.
\end{itemize}

An example VMC section for a simple VMC run:
\begin{lstlisting}[style=QMCPXML]
  <qmc method="vmc" move="pbyp">
    <estimator name="LocalEnergy" hdf5="no"/>
    <parameter name="walkers">    256 </parameter>
    <parameter name="warmupSteps">  100 </parameter>
    <parameter name="substeps">  5 </parameter>
    <parameter name="blocks">  20 </parameter>
    <parameter name="steps">  100 </parameter>
    <parameter name="timestep">  1.0 </parameter>
    <parameter name="usedrift">   yes </parameter>
  </qmc>
\end{lstlisting}
Here we set 256 walkers per MPI, have a brief initial equilibration of 100 steps, and then 20 blocks of 100 steps with 5 substeps each.

The following is an example of VMC section storing configurations (walker samples) for optimization.
\begin{lstlisting}[style=QMCPXML]
  <qmc method="vmc" move="pbyp" gpu="yes">
    <estimator name="LocalEnergy" hdf5="no"/>
    <parameter name="walkers">    256 </parameter>
    <parameter name="samples">    2867200 </parameter>
    <parameter name="stepsbetweensamples">    1 </parameter>
    <parameter name="substeps">  5 </parameter>
    <parameter name="warmupSteps">  5 </parameter>
    <parameter name="blocks">  70 </parameter>
    <parameter name="timestep">  1.0 </parameter>
    <parameter name="usedrift">   no </parameter>
  </qmc>
\end{lstlisting}



