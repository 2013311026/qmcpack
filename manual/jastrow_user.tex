\subsection{User defined functional form}
\label{sec:jastrowuserform}

To aid in implementing different forms for $u_{ab}(r)$, there is a script that uses a symbolic expression to generate the appropriate code (with spatial and parameter derivatives).
The script is located in \texttt{src/QMCWaveFunctions/Jastrow/codegen/user\_jastrow.py}.
The script requires Sympy (\url{www.sympy.org}) for symbolic mathematics and code generation.

To use the script, modify it to specify the functional form and a list of variational parameters.
Optionally, there may be fixed parameters - ones that are specified in the input file, but are not part of the variational optimization.
Also one symbol may be specified that accepts a cusp value in order to satisfy the cusp condition.
There are several example forms in the script.  The default form is the simple Pad\'e.

Once the functional form and parameters are specified in the script, run the script from the \texttt{codegen} directory and recompile QMCPACK.
The main output of the script is the file \texttt{src/QMCWaveFunctions/Jastrow/UserFunctor.h}.
The script also prints information to the screen, and one section is a sample XML input block containing all the parameters.

There is a unit test in \texttt{src/QMCWaveFunctions/test/test\_user\_jastrow.cpp} to perform
some minimal testing of the Jastrow factor.   The unit test will need updating to properly test
new functional forms.   Most of the changes relate to the number and name of variational 
parameters.


\begin{table}[h]
\begin{center}
\begin{tabular}{l c c c l }
\hline
\multicolumn{5}{l}{Jastrow element} \\
\hline
\bfseries name & \bfseries datatype & \bfseries values & \bfseries defaults  & \bfseries description \\
\hline
name & text &    & (required) & Unique name for this Jastrow function \\
type & text & One-body & (required) & Define a one-body function \\
     &      & Two-body & (required) & Define a two-body function \\
function & text & user & (required) & User-defined functor \\
\multicolumn{5}{l}{See other parameters as approriate for one or two-body functions} \\
  \hline
\multicolumn{5}{l}{elements}\\ \hline
& Correlation & & & \\ \hline
\multicolumn{5}{l}{Contents}\\ \hline
& (None)  & & &  \\ \hline
\end{tabular}
%\end{tabular*}
\end{center}
\end{table}

