\section{Specifying the particle set}
\label{sec:particleset}


The \ixml{particleset} blocks specify the particles in the QMC simulations: their types, attributes (mass, charge, valence), and positions.   

\subsection{Input specification}
\begin{table}[h]
\begin{center}
\begin{tabularx}{\textwidth}{l l l l l X }
\hline
\multicolumn{6}{l}{\texttt{particleset} element} \\
\hline
\multicolumn{2}{l}{parent elements:} & \multicolumn{4}{l}{\texttt{simulation}}\\
\multicolumn{2}{l}{child  elements:} & \multicolumn{4}{l}{\texttt{group, attrib}}\\
\multicolumn{2}{l}{attribute      :} & \multicolumn{4}{l}{}\\
   &   \bfseries name            & \bfseries datatype & \bfseries values & \bfseries default   & \bfseries description \\
   &   \texttt{name}/\texttt{id}   &  text              &  \textit{any}    &  e                & Name of particle set  \\
   &   \texttt{size}$^o$           &  integer           &  \textit{any}    &  0                & Number of particles in set \\
   &   \texttt{random}$^o$         &  text              &  yes/no          &  no               & Randomize starting positions \\
   &   \texttt{randomsrc}/         &  text     & \texttt{particleset.name} & \textit{none}     & Particle set to randomize  \\
   &   \texttt{random\_source}$^o$ &                    &                  &                   &                       \\
%   &   \texttt{role}     &  text              &  MC/none         &  none               & (obsolete)                       \\
  \hline
\end{tabularx}
\end{center}
\end{table}

\begin{table}[h]
\begin{center}
\begin{tabularx}{\textwidth}{l l l l l X }
\hline
\multicolumn{6}{l}{\texttt{group} element} \\
\hline
\multicolumn{2}{l}{parent elements:} & \multicolumn{4}{l}{\texttt{particleset}}\\
\multicolumn{2}{l}{child  elements:} & \multicolumn{4}{l}{\texttt{parameter, attrib}}\\
\multicolumn{2}{l}{attribute      :} & \multicolumn{4}{l}{}\\
   &   \bfseries name            & \bfseries datatype & \bfseries values & \bfseries default   & \bfseries description \\
   &   \texttt{name}               &  text              &  \textit{any}    &  e                & Name of particle set  \\
   &   \texttt{size}$^o$           &  integer           &  \textit{any}    &  0                & Number of particles in set \\
   &   \texttt{mass}$^o$           &  real              &  \textit{any}    &  1                & Mass of particles in set \\
   &   \texttt{unit}$^o$          &  text              &  au/amu          &  au               & Units for mass of particles \\
\multicolumn{2}{l}{parameters}  & \multicolumn{4}{l}{}\\
   &   \bfseries name     & \bfseries datatype & \bfseries values & \bfseries default   & \bfseries description \\
   &   \texttt{charge}    &  real              &  \textit{any}    &  0                  & Charge of particles in set \\
   &   \texttt{valence}   &  real              &  \textit{any}    &  0                  & Valence charge of particles in set \\
   &   \texttt{atomicnumber} &  integer        &  \textit{any}    &  0                  & Atomic number of particles in set \\
  \hline
  \hline
\end{tabularx}
\end{center}
\end{table}

\begin{table}[h]
\begin{center}
\begin{tabularx}{\textwidth}{l l l l l X }
\hline
\multicolumn{6}{l}{\texttt{attrib} element} \\
\hline
\multicolumn{2}{l}{parent elements:} & \multicolumn{4}{l}{\texttt{particleset,group}}\\
\multicolumn{2}{l}{attribute      :} & \multicolumn{4}{l}{}\\
   &   \bfseries name            & \bfseries datatype & \bfseries values & \bfseries default   & \bfseries description \\
   &   \texttt{name}             &  string            &  \textit{any}    &  \textit{none}    & Name of attrib              \\
   &   \texttt{datatype}         &  string            &  intArray, realArray, &  \textit{none} & Type of data in attrib \\
   &                             &                    &  posArray, stringArray &             &                        \\
   &   \texttt{size}$^o$         &  string            &  \textit{any}    &  \textit{none}    & Size of data in attrib \\
  \hline
  \hline
\end{tabularx}
\end{center}
\end{table}

\subsection{Detailed attribute description}

\subsubsection{particleset required attributes}

\begin{itemize}
\item \ixml{name}/\ixml{id} \\
Unique name for the particle set. Default is ``e" for electrons. ``i" or ``ion0" is typically used for ions. 
\end{itemize}
% Line 192 in ParticleIO/XMLParticleIO.cpp
% Lines 144-145 in QMCApp/ParticleSetPool.cpp

\subsubsection{particleset optional attributes}

\begin{itemize}
\item \ixml{size} \\
Number of particles in set
\end{itemize}
% Line 191 in ParticleIO/XMLParticleIO.cpp

%\begin{itemize}
%\item \ixml{role} \\
%What the particles do in the simulation
%\end{itemize}
% Line 146 in QMCApp/ParticleSetPool.cpp

\begin{itemize}
\item \ixml{random} \\
Randomize starting positions of particles. Each component of each particle's position is randomized independently in the range of the simulation cell in that component's direction. 
\end{itemize}
% Line 190 in ParticleIO/XMLParticleIO.cpp
% Line 147 in QMCApp/ParticleSetPool.cpp

\begin{itemize}
\item \ixml{randomsrc}/\ixml{random_source} \\
Specify source particle set around which to randomize the initial positions of this particle set.
\end{itemize}
% Lines 148-149 in QMCApp/ParticleSetPool.cpp

\subsubsection{name required attributes}

\begin{itemize}
\item \ixml{name}/\ixml{id} \\
Unique name for the particle set group. Typically, element symbols are used for ions and ``u" or ``d" for spin-up and spin-down electron groups, respectively. 
\end{itemize}
% Line 192 in ParticleIO/XMLParticleIO.cpp
% Lines 144-145 in QMCApp/ParticleSetPool.cpp

\subsubsection{group optional attributes}

\begin{itemize}
\item \ixml{mass} \\
Mass of particles in set.
\end{itemize}
% Line 190 in Particle/ParticleSet.cpp

\begin{itemize}
\item \ixml{unit} \\
Units for mass of particles in set (au[$m_e$ = 1] or amu[$\frac{1}{12}m_{\rm ^{12}C}$ = 1]).
\end{itemize}
% Line 66 in ParticleIO/XMLParticleIO.cpp


%condition appears to be future functionality for different unit types on the position array
%condition must be an integer
% Line 407 in ParticleIO/XMLParticleIO.cpp (reads condition in)
% Line 402 in ParticleIO/XMLParticleIO.cpp (declares utype integer)

\subsection{Example use cases}
\begin{minipage}{\linewidth}
\begin{lstlisting}[style=QMCPXML,caption=particleset elements for ions and electrons randomizing electron start positions.]
  <particleset name="i" size="2">
    <group name="Li">
      <parameter name="charge">3.000000</parameter>
      <parameter name="valence">3.000000</parameter>
      <parameter name="atomicnumber">3.000000</parameter>
    </group>
    <group name="H">
      <parameter name="charge">1.000000</parameter>
      <parameter name="valence">1.000000</parameter>
      <parameter name="atomicnumber">1.000000</parameter>
    </group>
    <attrib name="position" datatype="posArray" condition="1">
    0.0   0.0   0.0
    0.5   0.5   0.5
    </attrib>
    <attrib name="ionid" datatype="stringArray">
       Li H
    </attrib>
  </particleset>
  <particleset name="e" random="yes" randomsrc="i">
    <group name="u" size="2">
      <parameter name="charge">-1</parameter>
    </group>
    <group name="d" size="2">
      <parameter name="charge">-1</parameter>
    </group>
  </particleset>                 
\end{lstlisting}
\end{minipage}

\begin{minipage}{\linewidth}
\begin{lstlisting}[style=QMCPXML,caption=particleset elements for ions and electrons specifying electron start positions]
  <particleset name="e">
    <group name="u" size="4">
      <parameter name="charge">-1</parameter>
      <attrib name="position" datatype="posArray">
	2.9151687332e-01 -6.5123272502e-01 -1.2188463918e-01
	5.8423636048e-01  4.2730406357e-01 -4.5964306231e-03
	3.5228575807e-01 -3.5027014639e-01  5.2644808295e-01
       -5.1686250912e-01 -1.6648002292e+00  6.5837023441e-01
      </attrib>
    </group>
    <group name="d" size="4">
      <parameter name="charge">-1</parameter>
      <attrib name="position" datatype="posArray">
	3.1443445436e-01  6.5068682609e-01 -4.0983449009e-02
       -3.8686061749e-01 -9.3744432997e-02 -6.0456005388e-01
	2.4978241724e-02 -3.2862514649e-02 -7.2266047173e-01
       -4.0352404772e-01  1.1927734805e+00  5.5610824921e-01
      </attrib>
    </group>
  </particleset>
  <particleset name="ion0" size="3">
    <group name="O">
      <parameter name="charge">6</parameter>
      <parameter name="valence">4</parameter>
      <parameter name="atomicnumber">8</parameter>
    </group>
    <group name="H">
      <parameter name="charge">1</parameter>
      <parameter name="valence">1</parameter>
      <parameter name="atomicnumber">1</parameter>
    </group>
    <attrib name="position" datatype="posArray">
      0.0000000000e+00  0.0000000000e+00  0.0000000000e+00
      0.0000000000e+00 -1.4308249289e+00  1.1078707576e+00
      0.0000000000e+00  1.4308249289e+00  1.1078707576e+00
    </attrib>
    <attrib name="ionid" datatype="stringArray">
      O H H 
    </attrib>
  </particleset>
\end{lstlisting}
\end{minipage}

\begin{minipage}{\linewidth}
\begin{lstlisting}[style=QMCPXML,caption=particleset elements for ions specifying positions by ion type]
  <particleset name="ion0">
    <group name="O" size="1">
      <parameter name="charge">6</parameter>
      <parameter name="valence">4</parameter>
      <parameter name="atomicnumber">8</parameter>
      <attrib name="position" datatype="posArray">
        0.0000000000e+00  0.0000000000e+00  0.0000000000e+00
      </attrib>
    </group>
    <group name="H" size="2">
      <parameter name="charge">1</parameter>
      <parameter name="valence">1</parameter>
      <parameter name="atomicnumber">1</parameter>
      <attrib name="position" datatype="posArray">
        0.0000000000e+00 -1.4308249289e+00  1.1078707576e+00
        0.0000000000e+00  1.4308249289e+00  1.1078707576e+00
      </attrib>
    </group>
  </particleset>
\end{lstlisting}
\end{minipage}
