\section{One-body Jastrow functions}
\textbf{The text below is rough draft and to be replaced with a
  complete and accurate description! In particular, the table columns
  and entries require consideration. }

The one-body Jastrow is designed to... and is normally used in
conjunction with an additional two-body term. Many different one-body
are implemented...

The jastrow function is specified within a \texttt{wavefunction} element
and must contain one of more \texttt{correlation} elements specifying
additional parameters as well as the actual coefficients. Section
\ref{sec:1bjsplineexamples} gives examples of the typical nesting of
\texttt{jastrow}, \texttt{correlation}, and \texttt{coefficient} elements.

\subsection{Input Specification}

\begin{table}[h]
\begin{center}
\begin{tabular}{l c c c l }
\hline
\multicolumn{5}{l}{Jastrow element} \\
\hline
\bfseries name & \bfseries datatype & \bfseries values & \bfseries defaults  & \bfseries description \\
\hline
name & text & ? & ? & Unique name for this Jastrow function \\
type & text & One-body & (required) & Define a one-body function \\ 
function & text & Bspline & (required) & BSpline Jastrow \\
             &          & Pade & & Pade form \\
             &          & \ldots & & \ldots \\
source & text & ? & ? & \\ 
print & text & ? &  ? & \\
  \hline
\end{tabular}
%\end{tabular*}
\end{center}
\end{table}

\subsection{Spline form}

The one-body spline Jastrow function is the most commonly used one-body Jastrow for solids. This form was first described and used in \cite{spline_jastrow_citation_where_is_it_from}. 
\begin{equation}
J1=\sum_I^{ion0}\sum_i^e u_{ab}(|r_r-R_I|)
\end{equation}
where $u_{ab}$ is an interpolating spline between zero distance and $r_{cut}$. In 3D periodic systems the default cutoff distance is the Wigner Seitz cell radius. For other periodicities including isolated molecules the $r_{cut}$ must be specified. The gradient at zero distance is...  cusp can be set....  $r_i$ and $R_I$ are most commonly the electron and ion positions, but any particlesets that can 
provide the needed centers can be used.

\subsubsection{Input Specification}
\begin{table}[h]
\begin{center}
\begin{tabular}{l c c c l }
\hline
\multicolumn{5}{l}{Correlation element} \\
\hline
\bfseries name & \bfseries datatype & \bfseries values & \bfseries defaults & \bfseries description \\
\hline
elementType & fillmein  & &  & \\
size &  & &  & \\
rcut &  & &  & \\
cusp&   & &  & \\
print & text & ? & (optional) &  \\
\hline
\multicolumn{5}{l}{elements}\\ \hline
& Coefficients & & & \\ \hline
\multicolumn{5}{l}{Contents}\\ \hline
& (None)  & & &  \\ \hline
\end{tabular}
%\end{tabular*}
\end{center}
\end{table}

Additional information:

 \begin{itemize}
 \item \texttt{rcut}. The cutoff distance for the function in atomic units. For 3D fully periodic systems this parameter is optional and a default of the Wigner Seitz cell radius is used. Otherwise this parameter is required.
 \end{itemize}

\begin{table}[h]
\begin{center}
\begin{tabular}{l c c c l }
\hline
\multicolumn{5}{l}{Coefficients element} \\
\hline
\bfseries name & \bfseries datatype & \bfseries values & \bfseries defaults & \bfseries description \\
\hline
id & text & & & Unique identifier \\
type & text & Array & & \\
\hline
\multicolumn{5}{l}{elements}\\ \hline
(None) & & & \\ \hline
\multicolumn{5}{l}{Contents}\\ \hline
 (no name) & real array & & zeros & Jastrow coefficients \\ \hline
\end{tabular}
%\end{tabular*}
\end{center}
\end{table}


\subsubsection{Example use cases}
\label{sec:1bjsplineexamples}

Specify a spin-independent function with four parameters. Because rcut  is not specified, the default cutoff of the Wigner Seitz cell radius is used; this Jastrow must be used with a 3D periodic system such as a bulk solid. The source of the ionic positions is set "i".
\begin{lstlisting}
<jastrow name="J1" type="One-Body" function="Bspline" print="yes" source="i">
 <correlation elementType="C" cusp="0.0" size="4">
   <coefficients id="C" type="Array"> 0  0  0  0  </coefficients>
 </correlation>
</jastrow>
\end{lstlisting}

Specify a spin-dependent function with seven upspin and seven downspin parameters. The cutoff distance is set to 6 atomic units.
\begin{lstlisting}[language=xml]
<jastrow name="J1" type="One-Body" function="Bspline" source="ion0" spin="yes">
  <correlation speciesA="C" speciesB="u" size="7" rcut="6">
    <coefficients id="eCu" type="Array"> 
    0.0 0.0 0.0 0.0 0.0 0.0 0.0
    </coefficients>
  </correlation>
  <correlation speciesA="C" speciesB="d" size="7" rcut="6">
    <coefficients id="eCd" type="Array"> 
    0.0 0.0 0.0 0.0 0.0 0.0 0.0
    </coefficients>
  </correlation>
</jastrow>
\end{lstlisting}
