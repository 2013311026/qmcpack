\subsubsection{Spline form}
\label{sec:onebodyjastrowspline}

The one-body spline Jastrow function is the most commonly used one-body Jastrow for solids. This form 
was first described and used in \cite{EslerKimCeperleyShulenburger2012}.  
Here $u_{ab}$ is an interpolating 1D B-spline (tricublc spline on a linear grid) between zero distance and $r_{cut}$. In 3D periodic systems 
the default cutoff distance is the Wigner Seitz cell radius. For other periodicities, including isolated 
molecules, the $r_{cut}$ must be specified. The cusp can be set.   $r_i$ 
and $R_I$ are most commonly the electron and ion positions, but any particlesets that can provide the 
needed centers can be used.

\paragraph{Input specification}
\begin{table}[h]
\begin{center}
\begin{tabular}{l c c c l }
\hline
\multicolumn{5}{l}{Correlation element} \\
\hline
\bfseries Name & \bfseries Datatype & \bfseries Values & \bfseries Defaults & \bfseries Description \\
\hline
ElementType & Text & Name & See below & Classical particle target  \\
SpeciesA & Text & Name & See below & Classical particle target \\
SpeciesB & Text & Name & See below & Quantum species target \\
Size & Integer & $> 0$ & (Required) & Number of coefficients \\
Rcut & Real & $> 0$ & See below & Distance at which the correlation goes to 0 \\
Cusp & Real & $\ge 0$ & 0 & Value for use in Kato cusp condition \\
Spin & Text & Yes or no & No & Spin-dependent Jastrow factor \\
\hline
\multicolumn{5}{l}{Elements}\\ \hline
& Coefficients & & & \\ \hline
\multicolumn{5}{l}{Contents}\\ \hline
& (None)  & & &  \\ \hline
\end{tabular}
%\end{tabular*}
\end{center}
\end{table}

Additional information:

 \begin{itemize}
 \item \ixml{elementType, speciesA, speciesB, spin}.  For a spin-independent Jastrow factor (spin = ``no''),
elementType should be the name of the group of ions in the classical particleset to which the quantum
particles should be correlated.  For a spin-dependent Jastrow factor (spin = ``yes''), set speciesA to the
group name in the classical particleset and speciesB to the group name in the quantum particleset.
 \item \ixml{rcut}. The cutoff distance for the function in atomic units (bohr). 
For 3D fully periodic systems, this parameter is optional, and a default of the Wigner 
Seitz cell radius is used. Otherwise this parameter is required.
 \item \ixml{cusp}. The one-body Jastrow factor can be used to make the wavefunction
satisfy the electron-ion cusp condition\cite{kato}.  In this case, the derivative of the Jastrow
factor as the electron approaches the nucleus will be given by
\begin{equation}
\left(\frac{\partial J}{\partial r_{iI}}\right)_{r_{iI} = 0} = -Z .
\end{equation}
Note that if the antisymmetric part of the wavefunction satisfies the electron-ion cusp
condition (for instance by using single-particle orbitals that respect the cusp condition)
or if a nondivergent pseudopotential is used, the Jastrow should be cuspless at the 
nucleus and this value should be kept at its default of 0.
 \end{itemize}


\begin{table}[h]
\begin{center}
\begin{tabular}{l c c c l }
\hline
\multicolumn{5}{l}{Coefficients element} \\
\hline
\bfseries Name & \bfseries Datatype & \bfseries Values & \bfseries Defaults & \bfseries Description \\
\hline
Id & Text & & (Required) & Unique identifier \\
Type & Text & Array & (Required) & \\
Optimize & Text & Yes or no & Yes & if no, values are fixed in optimizations \\
\hline
\multicolumn{5}{l}{Elements}\\ \hline
(None) & & & \\ \hline
\multicolumn{5}{l}{Contents}\\ \hline
 (No name) & Real array & & Zeros & Jastrow coefficients \\ \hline
\end{tabular}
%\end{tabular*}
\end{center}
\end{table}


\paragraph{Example use cases}
\label{sec:1bjsplineexamples}

Specify a spin-independent function with four parameters. Because rcut  is not 
specified, the default cutoff of the Wigner Seitz cell radius is used; this 
Jastrow must be used with a 3D periodic system such as a bulk solid. The name of 
the particleset holding the ionic positions is ``i."
\begin{lstlisting}[style=QMCPXML]
<jastrow name="J1" type="One-Body" function="Bspline" print="yes" source="i">
 <correlation elementType="C" cusp="0.0" size="4">
   <coefficients id="C" type="Array"> 0  0  0  0  </coefficients>
 </correlation>
</jastrow>
\end{lstlisting}

Specify a spin-dependent function with seven up-spin and seven down-spin parameters. 
The cutoff distance is set to 6 atomic units.  Note here that the particleset holding
the ions is labeled as ion0 rather than ``i,'' as in the other example.  Also in this case,
the ion is lithium with a coulomb potential, so the cusp condition is satisfied by 
setting cusp=``d."
\begin{lstlisting}[style=QMCPXML]
<jastrow name="J1" type="One-Body" function="Bspline" source="ion0" spin="yes">
  <correlation speciesA="Li" speciesB="u" size="7" rcut="6">
    <coefficients id="eLiu" cusp="3.0" type="Array"> 
    0.0 0.0 0.0 0.0 0.0 0.0 0.0
    </coefficients>
  </correlation>
  <correlation speciesA="C" speciesB="d" size="7" rcut="6">
    <coefficients id="eLid" cusp="3.0" type="Array"> 
    0.0 0.0 0.0 0.0 0.0 0.0 0.0
    </coefficients>
  </correlation>
</jastrow>
\end{lstlisting}

