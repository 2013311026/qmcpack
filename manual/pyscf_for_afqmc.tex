\section{Using PySCF to generate integrals for AFQMC}
\label{sec:pyscf}
These notes describe useful knowledge and tips in the use of pyscf. 
The main purpose of pyscf in the group is in the generation of 2-electron integrals for correlated calculations in solids.

For a detailed description of pyscf, please see the documentation by the authors.

Examples of how to generate integrals are:
\begin{itemize}
\item Run a standard pyscf calculation, e.g. a HF or DFT calculation. Make sure you preserve the chkfile and make sure you store the core hamiltonian on the chkfile. An example of how to do this for a single k-point calculation is found below.
\begin{lstlisting}[style=Python,caption=The following is an example PySCF input file for single k-point calculations.]
import numpy
import h5py
from mpi4py import MPI
from pyscf.pbc import gto, scf, dft, cc

cell = gto.Cell()
cell.a = '''
3.5668 0 0
0 3.5668 0
0 0 3.5668'''
cell.atom = '''C 0. 0. 0. 
C 0.8917 0.8917 0.8917
C 1.7834 1.7834 0. 
C 2.6751 2.6751 0.8917
C 1.7834 0. 1.7834
C 2.6751 0.8917 2.6751
C 0. 1.7834 1.7834
C 0.8917 2.6751 2.6751'''
cell.basis = 'gth-szv'
cell.pseudo = 'gth-pade'
cell.gs = [10]*3 # 10 grids on positive x direction, => 21^3 grids in total
cell.verbose = 4
cell.build()

kpt = cell.get_abs_kpts([0.25, 0.25, 0.25])  

mf = scf.RHF(cell,kpt,exxdiv=0)
mf.chkfile = "scf.dump"                         # store checkpoint file in scf.dump
ehf = mf.kernel()
print("HF energy (per unit cell) = %.17g" % ehf)

hcore = mf.get_hcore(kpt=kpt)            # obtain and store core hamiltonian
with h5py.File(mf.chkfile) as fh5:
  fh5['scf/hcore'] = hcore
\end{lstlisting}

\item {For a calculation with k-points:
Run a standard pyscf calculation, e.g. a HF or DFT calculation. Make sure you preserve the chkfile and make sure you store the core hamiltonian on the chkfile. An example of how to do this for a single k-point calculation is found below.}

\begin{lstlisting}[style=Python,caption=The following is an example PySCF input file for calculations with k-points.]
import numpy
import h5py
from mpi4py import MPI
from pyscf.pbc import gto, scf, dft, cc

cell = gto.Cell()
cell.a = '''
3.5668 0 0
0 3.5668 0
0 0 3.5668'''
cell.atom = '''C 0. 0. 0. 
C 0.8917 0.8917 0.8917
C 1.7834 1.7834 0. 
C 2.6751 2.6751 0.8917
C 1.7834 0. 1.7834
C 2.6751 0.8917 2.6751
C 0. 1.7834 1.7834
C 0.8917 2.6751 2.6751'''
cell.basis = 'gth-szv'
cell.pseudo = 'gth-pade'
cell.gs = [10]*3 # 10 grids on positive x direction, => 21^3 grids in total
cell.verbose = 4
cell.build()

nk = [2,2,2]
kpts = cell.make_kpts(nk) 

mf = scf.KRHF(cell,kpts,exxdiv=0)
mf.chkfile = "scf.dump"                         # store checkpoint file in scf.dump
ehf = mf.kernel()
print("HF energy (per unit cell) = %.17g" % ehf)

hcore = mf.get_hcore(kpts=kpts)          # obtain and store core hamiltonian
fock = (hcore + mf.get_veff(kpts=kpts))  # store fock matrix (required with orthoAO=True)
with h5py.File(mf.chkfile) as fh5:
  fh5['scf/hcore'] = hcore
  fh5['scf/fock'] = fock
\end{lstlisting}
\end{itemize}

Once the checkpoint file has been created, it is now possible to generate the integral file. The recommended approach is:

\begin{lstlisting}[style=Python,caption=The following is an example input file for calculating the integrals.]
from mpi4py import MPI
from qmctools import integrals_from_chkfile

comm = MPI.COMM_WORLD
rank = comm.Get_rank()
nproc = comm.Get_size()

integrals_from_chkfile.eri_to_h5("fcidump", rank, nproc, "scf.dump")    

comm.Barrier()

if rank==0:
    integrals_from_chkfile.combine_eri_h5("fcidump", nproc)
\end{lstlisting}

It is also possible to generated the Cholesky decomposed integrals in pyscf directly. This is typically faster and more appropriate. (the phfmol code expects integrals directly, but these can be generated from qmcpack with the Cholesky decomposed ones.) To generate them for a run with kpoints for example, the recommended approach is: (NOTE: for some reason, mpi4py must be imported first!!!)

For calculations with kpoints (those generated with K...), use integrals\_from\_chkfile.eri\_to\_h5\_kpts(...).
Additional arguments to eri\_to\_h5 are:

\begin{itemize}
\item \textbf{cholesky}. Determines whether 2-electron integrals or their cholesky factorization is calculated.
  Default: False
\item \textbf{orthoAO}. If True, generates the integrals in the orthogonalized AO basis. If False, generates the integrals in the MO basis found on the checkpoint file. For UHF calculations, only orthoAO=True is allowed. If set to False, the fock matrix must be stored in the scf dump file.
  Default: False
\item \textbf{LINDEP\_CUTOFF}.  Cutoff used to define linearly dependent basis functions.
  Default: 1e-9
\item \textbf{gtol}. Cutoff applied during writing for 2-electron integrals. If cholesky=True, then this is the cutoff used in the iterative cholesky factorization. In this case, the resulting factorized hamiltonian will have a residual error smaller than the requested cutoff.
  Default: 1e-6
\item \textbf{wfnName}. Name of the file with the wavefuntion. This is only generated  when orthoAO=True.
  Default: ``wfn.dat''
\item \textbf{wfnPHF}. Name of file with initial guess for phfmol code.
  Default: None (no file is generated)
\item \textbf{MaxIntgs}. Maximum number of integrals  (or terms in cholesky matrix) per block in the hdf5 file. This controls the size of the hdf5 data sets.
  Default: 2000000
\item \textbf{maxvecs}. Represents the maximum number of Cholesky vectors allowed. The actual maximum number of cholesky vectors in the calculation is set to maxvecsnmo. So a value of 10 leads to an actual cutoff in the number of vectors of 10*nmo, where nmo is the total number of molecular orbitals in the calculation. The calculation will stop at the requested number of vectors even if the tolerance is not reached.
  Default: 20
\end{itemize}
