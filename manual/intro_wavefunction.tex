\section{Introduction}
\label{sec:intro_wavefunction}

This section describes the input blocks associated with the specification of the trial wavefunction in a QMCPACK calculation. These sections are contained within the \ixml{<wavefunction> ...  </wavefunction>} xml blocks. \textbf{Users are expected to rely on converters to generate the input blocks described in this section.} The converters and the workflows are designed such that input blocks require minimum modifications from users. Unless the workflow requires modification of wavefunction blocks (e.g., setting the cutoff in a multideterminant calculation), only expert users should directly alter them.
  
The trial wavefunction in QMCPACK has a general product form:
\begin{equation}
\Psi_T(\vec{r}) = \prod_k \Theta_k(\vec{r}) ,
\end{equation}
where each $\Theta_k(\vec{r})$ is a function of the electron coordinates (and possibly ionic coordinates and variational parameters). For problems involving electrons, the overall trial wavefunction must be antisymmetric with respect to electron exchange, so at least one of the functions in the product must be antisymmetric. Notice that, although QMCPACK allows for the construction of arbitrary trial wavefunctions based on the functions implemented in the code (e.g., slater determinants, jastrow functions), the user must make sure that a correct wavefunction is used for the problem at hand. From here on, we assume a standard trial wavefunction for an electronic structure problem 
\begin{equation}
\Psi_T(\vec{r}) =  \textit{A}(\vec{r}) \prod_k \textit{J}_k(\vec{r}),
\end{equation}
where $\textit{A}(\vec{r})$ is one of the antisymmetric functions: (1) slater determinant, (2) multislater determinant, or (3) pfaffian and $\textit{J}_k$ is any of the Jastrow functions (described in Section~\ref{sec:jastrow}).  The antisymmetric functions are built from a set of single particle orbitals (\texttt{sposet}). QMCPACK implements four different types of \texttt{sposet}, described in the following section. Each \texttt{sposet} is designed for a different type of calculation, so their definition and generation varies accordingly. 
