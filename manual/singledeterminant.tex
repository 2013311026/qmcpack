\section{Single determinant wavefunctions}
\label{sec:singledeterminant}
Placing a single determinant for each spin is the most used ansatz for the antisymmetric part of a trial wavefunction.
The input xml block for \texttt{slaterdeterminant} is give in Listing~\ref{listing:singledet}. A list of options is given in
Table~\ref{table:singledet}.

\begin{table}[h]
\begin{center}
\begin{tabularx}{\textwidth}{l l l l l X }
\hline
\multicolumn{6}{l}{\texttt{slaterdeterminant} element} \\
\hline
\multicolumn{2}{l}{Parent elements:} & \multicolumn{4}{l}{\texttt{determinantset}}\\
\multicolumn{2}{l}{Child  elements:} & \multicolumn{4}{l}{\texttt{determinant}}\\
\multicolumn{2}{l}{Attribute:} & \multicolumn{4}{l}{}\\
   &   \bfseries Name       & \bfseries Datatype & \bfseries Values & \bfseries Default & \bfseries Description \\
   &   \texttt{delay\_rank} &  Integer           &  >0              & 1           &  Number of delayed updates. \\
   &   \texttt{optimize}    &  Text              &  Yes/no          & Yes         &  Enable orbital optimization. \\
  \hline
\end{tabularx}
\end{center}
\caption{Options for the \texttt{slaterdeterminant} xml-block.}
\label{table:singledet}
\end{table}

\begin{lstlisting}[style=QMCPXML,caption=Slaterdeterminant set XML element.\label{listing:singledet}]
  <slaterdeterminant delay_rank="32">
    <determinant id="updet" size="208">
      <occupation mode="ground" spindataset="0">
      </occupation>
    </determinant>
    <determinant id="downdet" size="208">
      <occupation mode="ground" spindataset="0">
      </occupation>
    </determinant>
  </slaterdeterminant>
\end{lstlisting}

Additional information:
\begin{itemize}
\item \ixml{delay\_rank}. This option enables delayed updates of the Slater matrix inverse when particle-by-particle move is used.
By default, \ixml{delay\_rank=1} uses the Fahy's variant~\cite{Fahy1990} of the Sherman-Morrison rank-1 update, which is mostly using memory bandwidth-bound BLAS-2 calls.
With \ixml{delay\_rank>1}, the delayed update algorithm~\cite{Luo2018delayedupdate,McDaniel2017} turns most of the computation to compute bound BLAS-3 calls.
Tuning this parameter is highly recommended to gain the best performance on medium-to-large problem sizes ($>200$ electrons).
We have seen up to an order of magnitude speedup on large problem sizes.
When studying the performance of QMCPACK, a scan of this parameter is required and we recommend starting from 32.
The best \ixml{delay\_rank} giving the maximal speedup depends on the problem size.
Usually the larger \ixml{delay\_rank} corresponds to a larger problem size.
On CPUs, \ixml{delay\_rank} must be chosen as a multiple of SIMD vector length for good performance of BLAS libraries.
The best \ixml{delay\_rank} depends on the processor microarchitecture.
GPU support is under development.
\end{itemize}

