\section{Specifying the simulation cell}
\label{chap:simulationcell}

The \texttt{simulationcell} block specifies the geometry of the cell, how the boundary conditions should be handled, and how ewald summation should be broken up.

\begin{table}[h]
\begin{center}
\begin{tabularx}{\textwidth}{l l l l l l }
\hline
\multicolumn{6}{l}{\texttt{simulationcell} element} \\
\hline
\multicolumn{2}{l}{parent elements:} & \multicolumn{4}{l}{\texttt{qmcsystem}}\\
\multicolumn{2}{l}{child  elements:} & \multicolumn{4}{l}{None}\\
\multicolumn{2}{l}{attribute      :} & \multicolumn{4}{l}{}\\
   &   \bfseries parameter name            & \bfseries datatype & \bfseries values & \bfseries default   & \bfseries description \\
\hline
   &   \texttt{lattice}  & 9 floats & any float & Must be specified & Specification of \\
   &                     &        &             &                   & lattice vectors. \\
   &   \texttt{bconds}   & string & ``p'' or ``n''  & ``n n n'' & Boundary conditions \\
   &                     &        &             &           & for each axis. \\
   &   \texttt{vacuum} & float & $>0$ & 1.0        & Vacuum scale. \\
   &   \texttt{LR\_dim\_cutoff} & float & float & 15        & Ewald breakup distance. \\
\hline
\end{tabularx}
\end{center}
\end{table}

An example of a \texttt{simulationcell} block is given below:
\begin{shade}
  <simulationcell>
    <parameter name="lattice">
      3.80000000       0.00000000       0.00000000
      0.00000000       3.80000000       0.00000000
      0.00000000       0.00000000       3.80000000
    </parameter>
    <parameter name="bconds">
       p p p
    </parameter>
    <!--parameter name="vacuum"> 1.0 </parameter-->
    <parameter name="LR_dim_cutoff"> 20 </parameter>
  </simulationcell>
\end{shade}

Here, a cubic cell 3.8 bohr on a side will be used.
This simulation will use periodic boundary conditions, and the maximum
$k$ vector will be $20/r_{wigner-seitz}$ of the cell.


\subsection{Lattice}
The cell is specified using 3 lattice vectors.


\subsection{Boundary conditions}
QMCPACK offers the capability to use a mixture of open and periodic boundary conditions.
The \texttt{bconds} parameter expects a single string of three characters separated by
spaces, \textit{e.g.} ``p p p'' for purely periodic boundary conditions. These characters control
the behavior of the $x$, $y$, and $z$, axes, respectively.
Examples of valid \texttt{bconds} include:

\begin{description}
\item[``p p p''] Periodic boundary conditions. Corresponds to 3d crystal.
\item[``n n n''] Open boundary conditions. Corresponds to isolated molecule in a vacuum.
\item[``p p n''] Slab geometry. Corresponds to 2d crystal.
\item[``p n n''] Wire geometry. Corresponds to 1d crystal.
\end{description}

\subsection{Vacuum}
For slab or wire boundary conditions (\texttt{bconds= p p n} or \texttt{bconds= p n n}, respectively),
the factor \texttt{vacuum} increases the size of the axis labeled n. The default value is 1,
\textit{i.e.} no change to the specified axis.

This option helps reducing the computational cost of DFT and memory requirement in QMC.
For instance, a slab calculation requires a large box with a very long z direction
to minimize the interaction between periodic images in DFT.
Since QMC uses the orbitals from DFT contained in the same box,
a lot of memory is needed when orbitals are stored in B-Spline representation.
With `vacuum' option, the DFT calculation can use a smaller box but still large enough for accuracy
and the QMC calculation rescale the open directions of the box to
have the Coulomb interations calculated in large box.

\subsection{\texttt{LR\_dim\_cutoff}}
When using periodic boundary conditions direct calculation of the Coulomb energy is
not well behaved. As a result, QMCPACK uses an optimized Ewald summation technique
to compute the Coulomb interaction. %% Reference: Natoli & Ceperley 1995 JCP.

In the Ewald summation, the energy is broken into short- and long-ranged terms.
The short-ranged term is computed directly in real space, while the long-ranged term is computed in reciprocal space.
\texttt{LR\_dim\_cutoff} controls where the short-ranged term ends and the long-ranged term begins.
The real-space cutoff, reciprocal-space cutoff, and \texttt{LR\_dim\_cutoff} are related via:
\[
\texttt{LR\_dim\_cutoff} = r_{c} \times k_{c}
\]
where $r_{c}$ is the Wigner-Seitz radius, and $k_{c}$ is the length of the maximum $k$-vector used in the long-ranged term.
