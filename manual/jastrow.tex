\section{Jastrow Factors}
\label{sec:jastrow}

Jastrow factors are among the simplest and most effective ways of including
dynamical correlation in the trial many body wavefunction.  The resulting many body
wavefunction is expressed as the product of an antisymmetric (in the case
of Fermions) or symmetric (for Bosons) part and a correlating jastrow factor
like so:
\begin{equation}
\Psi(\vec{R}) = \mathcal{A}(\vec{R}) \exp\left[J(\vec{R})\right]
\end{equation}

In this section we will detail the types and forms of Jastrow factor used 
in QMCPACK.  Note that each type of Jastrow factor needs to be specified using
its own individual jastrow XML element.  For this reason, we have repeated the
specification of the jastrow tag in each section, with specialization for the
options available for that given type of jastrow.

\subsection{One-body Jastrow functions}
\label{sec:onebodyjastrow}
The one-body Jastrow factor is a form that allows for the direct inclusion
of correlations between particles that are included in the wavefunction with
particles that are not explicitly part of it.  The most common example of
this are correlations between electrons and ions.  

The jastrow function is specified within a \texttt{wavefunction} element
and must contain one or more \texttt{correlation} elements specifying
additional parameters as well as the actual coefficients. Section
\ref{sec:1bjsplineexamples} gives examples of the typical nesting of
\texttt{jastrow}, \texttt{correlation}, and \texttt{coefficient} elements.

\subsubsection{Input Specification}

\begin{table}[h]
\begin{center}
\begin{tabular}{l c c c l }
\hline
\multicolumn{5}{l}{Jastrow element} \\
\hline
\bfseries name & \bfseries datatype & \bfseries values & \bfseries defaults  & \bfseries description \\
\hline
name & text &    & (required) & Unique name for this Jastrow function \\
type & text & One-body & (required) & Define a one-body function \\ 
function & text & Bspline & (required) & BSpline Jastrow \\
             & text & pade2 & & Pade form \\
             & text & \ldots & & \ldots \\
source & text & name & (required) & name of attribute of classical particle set \\ 
print & text & yes / no & yes & jastrow factor printed in external file?\\
  \hline
\multicolumn{5}{l}{elements}\\ \hline
& Correlation & & & \\ \hline
\multicolumn{5}{l}{Contents}\\ \hline
& (None)  & & &  \\ \hline
\end{tabular}
%\end{tabular*}
\end{center}
\end{table}

To be more concrete, the one-body jastrow factors used to describe correlations
between electrons and ions take the form below
\begin{equation}
J1=\sum_I^{ion0}\sum_i^e u_{ab}(|r_i-R_I|)
\end{equation}
where I runs over all of the ions in the calculation, i runs over the electrons
and $u_{ab}$ describes the functional form of the correlation between them.
Many different forms of $u_{ab}$ are implemented in QMCPACK.  We will detail 
two of the most common ones below.
\subsubsection{Spline form}
\label{sec:onebodyjastrowspline}

The one-body spline Jastrow function is the most commonly used one-body Jastrow for solids. This form 
was first described and used in \cite{EslerKimCeperleyShulenburger2012}.  
Here $u_{ab}$ is an interpolating 1D Bspline (tricublc spline on a linear grid) between zero distance and $r_{cut}$. In 3D periodic systems 
the default cutoff distance is the Wigner Seitz cell radius. For other periodicities including isolated 
molecules the $r_{cut}$ must be specified. The cusp can be set.   $r_i$ 
and $R_I$ are most commonly the electron and ion positions, but any particlesets that can provide the 
needed centers can be used.

\paragraph{Input Specification}
\begin{table}[h]
\begin{center}
\begin{tabular}{l c c c l }
\hline
\multicolumn{5}{l}{Correlation element} \\
\hline
\bfseries name & \bfseries datatype & \bfseries values & \bfseries defaults & \bfseries description \\
\hline
elementType & text & name & see below & Classical particle target  \\
speciesA & text & name & see below & Classical particle target \\
speciesB & text & name & see below & Quantum species target \\
size & integer & $> 0$ & (required) & Number of coefficients \\
rcut & real & $> 0$ & see below & Distance at which the correlation goes to 0 \\
cusp & real & $\ge 0$ & 0 & Value for use in Kato cusp condition \\
spin & text & yes or no & no & Spin dependent jastrow factor \\
\hline
\multicolumn{5}{l}{elements}\\ \hline
& Coefficients & & & \\ \hline
\multicolumn{5}{l}{Contents}\\ \hline
& (None)  & & &  \\ \hline
\end{tabular}
%\end{tabular*}
\end{center}
\end{table}

Additional information:

 \begin{itemize}
 \item \texttt{elementType, speciesA, speciesB, spin}.  For a spin independent Jastrow factor (spin = ``no'')
elementType should be the name of the group of ions in the classical particleset to which the quantum
particles should be correlated.  For a spin dependent Jastrow factor (spin = ``yes'') set speciesA to the
group name in the classical particleset and speciesB to the group name in the quantum particleset.
 \item \texttt{rcut}. The cutoff distance for the function in atomic units (bohr). 
For 3D fully periodic systems this parameter is optional and a default of the Wigner 
Seitz cell radius is used. Otherwise this parameter is required.
 \item \texttt{cusp}. The one body jastrow factor can be used to make the wavefunction
satisfy the electron-ion cusp condition\cite{kato}.  In this case, the derivative of the jastrow
factor as the electron approaches the nucleus will be given by:
\begin{equation}
\left(\frac{\partial J}{\partial r_{iI}}\right)_{r_{iI} = 0} = -Z
\end{equation}
Note that if the antisymmetric part of the wavefunction satisfies the electron-ion cusp
condition (for instance by using single particle orbitals that respect the cusp condition)
or if a non-divergent pseudopotential is used that the Jastrow should be cuspless at the 
nucleus and this value should be kept at its default of 0.
 \end{itemize}


\begin{table}[h]
\begin{center}
\begin{tabular}{l c c c l }
\hline
\multicolumn{5}{l}{Coefficients element} \\
\hline
\bfseries name & \bfseries datatype & \bfseries values & \bfseries defaults & \bfseries description \\
\hline
id & text & & (required) & Unique identifier \\
type & text & Array & (required) & \\
optimize & text & yes or no & yes & if no, values are fixed in optimizations \\
\hline
\multicolumn{5}{l}{elements}\\ \hline
(None) & & & \\ \hline
\multicolumn{5}{l}{Contents}\\ \hline
 (no name) & real array & & zeros & Jastrow coefficients \\ \hline
\end{tabular}
%\end{tabular*}
\end{center}
\end{table}


\paragraph{Example use cases}
\label{sec:1bjsplineexamples}

Specify a spin-independent function with four parameters. Because rcut  is not 
specified, the default cutoff of the Wigner Seitz cell radius is used; this 
Jastrow must be used with a 3D periodic system such as a bulk solid. The name of 
the particleset holding the ionic positions is "i".
\begin{lstlisting}[language=xml]
<jastrow name="J1" type="One-Body" function="Bspline" print="yes" source="i">
 <correlation elementType="C" cusp="0.0" size="4">
   <coefficients id="C" type="Array"> 0  0  0  0  </coefficients>
 </correlation>
</jastrow>
\end{lstlisting}

Specify a spin-dependent function with seven upspin and seven downspin parameters. 
The cutoff distance is set to 6 atomic units.  Note here that the particleset holding
the ions is labeled as ion0 rather than ``i'' in the other example.  Also in this case
the ion is Lithium with a coulomb potential, so the cusp condition is satisfied by 
setting cusp="d".
\begin{lstlisting}[language=xml]
<jastrow name="J1" type="One-Body" function="Bspline" source="ion0" spin="yes">
  <correlation speciesA="Li" speciesB="u" size="7" rcut="6">
    <coefficients id="eLiu" cusp="3.0" type="Array"> 
    0.0 0.0 0.0 0.0 0.0 0.0 0.0
    </coefficients>
  </correlation>
  <correlation speciesA="C" speciesB="d" size="7" rcut="6">
    <coefficients id="eLid" cusp="3.0" type="Array"> 
    0.0 0.0 0.0 0.0 0.0 0.0 0.0
    </coefficients>
  </correlation>
</jastrow>
\end{lstlisting}


\subsubsection{Pade form}
\label{sec:onebodyjastrowpade}

While the spline Jastrow factor is the most flexible and most commonly used form implemented in \qmcpack, 
there are times where its flexibility can make it difficult to optimize.  As an example, a spline jastrow
with a very large cutoff may be difficult to optimize for isolated systems like molecules due to the small
number of samples that will be present in the tail of the function.  In such cases, a simpler functional
form may be advantageous.  The second order Pade jastrow factor, given in Eq.\ref{padeeqn} is a good choice 
in such cases.  
\begin{equation}
\label{padeeqn}
u_{ab}(r) = \frac{a*r+c*r^2}{1+b*r}
\end{equation}
Unlike the spline jastrow factor which includes a cutoff, this form has an infinite range and for every particle
pair (subject to the minimum image convention) it will be applied.  It also is a cuspless jastrow factor,
so it should either be used in combination with a single particle basis set that contains the proper cusp or
with a smooth pseudopotential.

\paragraph{Input Specification}
\begin{table}[h]
\begin{center}
\begin{tabular}{l c c c l }
\hline
\multicolumn{5}{l}{Correlation element} \\
\hline
\bfseries name & \bfseries datatype & \bfseries values & \bfseries defaults & \bfseries description \\
\hline
elementType & text & name & see below & Classical particle target  \\
\hline
\multicolumn{5}{l}{elements}\\ \hline
& Coefficients & & & \\ \hline
\multicolumn{5}{l}{Contents}\\ \hline
& (None)  & & &  \\ \hline
\end{tabular}
%\end{tabular*}
\end{center}
\end{table}

\begin{table}[h]
\begin{center}
\begin{tabular}{l c c c l }
\hline
\multicolumn{5}{l}{parameter element} \\
\hline
\bfseries name & \bfseries datatype & \bfseries values & \bfseries defaults & \bfseries description \\
\hline
id & string & name & (required) & name for variable \\
name & string & A or B or C & (required) & see Eq.\ref{padeeqn}\\
optimize & text & yes or no & yes & if no, values are fixed in optimizations \\
\hline
\multicolumn{5}{l}{elements}\\ \hline
(None) & & & \\ \hline
\multicolumn{5}{l}{Contents}\\ \hline
 (no name) & real & parameter value & (required) & Jastrow coefficients \\ \hline
\end{tabular}
%\end{tabular*}
\end{center}
\end{table}

\paragraph{Example use case}
\label{sec:1bjpadeexamples}

Specify a spin independent function with independent jastrow factors for two different species (Li and H).
The name of the particleset holding the ionic positions is "i".
\begin{lstlisting}[style=XML]
<jastrow name="J1" function="pade2" type="One-Body" print="yes" source="i">
  <correlation elementType="Li">
    <var id="LiA" name="A">  0.34 </var>
    <var id="LiB" name="B"> 12.78 </var>
    <var id="LiC" name="C">  1.62 </var>
  </correlation>
  <correlation elementType="H"">
    <var id="HA" name="A">  0.14 </var>
    <var id="HB" name="B"> 6.88 </var>
    <var id="HC" name="C"> 0.237 </var>
  </correlation>
</jastrow>
\end{lstlisting}


\subsection{Two-body Jastrow functions}
The two-body Jastrow factor is a form that allows for the explicit inclusion
of dynamic correlation between two particles included in the wavefunction.  It
is almost always given in a spin dependent form so as to satisfy the Kato cusp
condition between electrons of different spins\cite{kato}.

 The two body jastrow function is specified within a \texttt{wavefunction} element
and must contain one or more correlation elements specifying additional parameters
as well as the actual coefficients.  Section \ref{sec:2bjsplineexamples} gives 
examples of the typical nesting of \texttt{jastrow}, \texttt{correlation} and
\texttt{coefficient} elements.

\subsubsection{Input Specification}

\begin{table}[h]
\begin{center}
\begin{tabular}{l c c c l }
\hline
\multicolumn{5}{l}{Jastrow element} \\
\hline
\bfseries name & \bfseries datatype & \bfseries values & \bfseries defaults  & \bfseries description \\
\hline
name & text &    & (required) & Unique name for this Jastrow function \\
type & text & Two-body & (required) & Define a one-body function \\ 
function & text & Bspline & (required) & BSpline Jastrow \\
print & text & yes / no & yes & jastrow factor printed in external file?\\
  \hline
\multicolumn{5}{l}{elements}\\ \hline
& Correlation & & & \\ \hline
\multicolumn{5}{l}{Contents}\\ \hline
& (None)  & & &  \\ \hline
\end{tabular}
%\end{tabular*}
\end{center}
\end{table}

The two-body jastrow factors used to describe correlations between electrons take the form
\begin{equation}
J2=\sum_i^{e}\sum_{j>i}^{e} u_{ab}(|r_i-r_j|)
\end{equation}

The most commonly used form of two body jastrow factor supported by the code is a splined
jastrow factor, with many similarities to the one body spline jastrow.

\subsubsection{Spline form}
\label{sec:twobodyjastrowspline}

The two-body spline Jastrow function is the most commonly used two-body Jastrow for solids. This form 
was first described and used in \cite{EslerKimCeperleyShulenburger2012}.  
Here $u_{ab}$ is an interpolating 1D Bspline (tricublc spline on a linear grid) between 
zero distance and $r_{cut}$. In 3D periodic systems 
the default cutoff distance is the Wigner Seitz cell radius. For other periodicities including isolated 
molecules the $r_{cut}$ must be specified.  $r_i$ and $r_j$ are typically electron positions.  The cusp 
condition as $r_i$ approaches $r_j$ is set by the relative spin of the electrons.

\FloatBarrier
\paragraph{Input Specification}

\FloatBarrier
\begin{table}[h!]
\begin{center}
\begin{tabular}{l c c c l }
\hline
\multicolumn{5}{l}{Correlation element} \\
\hline
\bfseries name & \bfseries datatype & \bfseries values & \bfseries defaults & \bfseries description \\
\hline
speciesA & text & u or d & (required) & Quantum species target \\
speciesB & text & u or d & (required) & Quantum species target \\
size & integer & $> 0$ & (required) & number of coefficients \\
rcut & real & $> 0$ & see below & distance at which the correlation goes to 0 \\
spin & text & yes or no & no & spin dependent jastrow factor \\
\hline
\multicolumn{5}{l}{elements}\\ \hline
& Coefficients & & & \\ \hline
\multicolumn{5}{l}{Contents}\\ \hline
& (None)  & & &  \\ \hline
\end{tabular}
%\end{tabular*}
\end{center}
\end{table}

\FloatBarrier

Additional information:
\begin{itemize}
\item \texttt{speciesA, speciesB} The scale function u(r) is defined for species pairs uu and ud.  
There is no need to define ud or dd since uu=dd and ud=du.  The cusp condition is computed internally 
based on the charge of the quantum particles.
\end{itemize}

\begin{table}[h]
\begin{center}
\begin{tabular}{l c c c l }
\hline
\multicolumn{5}{l}{Coefficients element} \\
\hline
\bfseries name & \bfseries datatype & \bfseries values & \bfseries defaults & \bfseries description \\
\hline
id & text & & (required) & Unique identifier \\
type & text & Array & (required) & \\
optimize & text & yes or no & yes & if no, values are fixed in optimizations \\
\hline
\multicolumn{5}{l}{elements}\\ \hline
(None) & & & \\ \hline
\multicolumn{5}{l}{Contents}\\ \hline
 (no name) & real array & & zeros & Jastrow coefficients \\ \hline
\end{tabular}
%\end{tabular*}
\end{center}
\end{table}

\paragraph{Example use cases}
\label{sec:2bjsplineexamples}

Specify a spin-dependent function with 4 parameters for each channel.  In this case, the cusp is set at 
a radius of 4.0 bohr (rather than to the default of the Wigner Seitz cell radius).  Also, in this example,
the coefficients are set to not be optimized during an optimization step.

\begin{lstlisting}
<jastrow name="J2" type="Two-Body" function="Bspline" print="yes">
  <correlation speciesA="u" speciesB="u" size="8" rcut="4.0">
    <coefficients id="uu" type="Array" optimize="no"> 0.2309049836 0.1312646071 0.05464141356 0.01306231516</coefficients>
  </correlation>
  <correlation speciesA="u" speciesB="d" size="8" rcut="4.0">
    <coefficients id="ud" type="Array" optimize="no"> 0.4351561096 0.2377951747 0.1129144262 0.0356789236</coefficients>
  </correlation>
</jastrow>
\end{lstlisting}


\subsection{Three-body Jastrow functions}
Explicit three body correlations can be included in the wavefunction via the three-body
jastrow factor.
