\section{Jastrow Factors}
\label{sec:jastrow}

Jastrow factors are among the simplest and most effective ways of including
dynamical correlation in the trial many body wavefunction.  The resulting many body
wavefunction is expressed as the product of an antisymmetric (in the case
of Fermions) or symmetric (for Bosons) part and a correlating jastrow factor
like so:
\begin{equation}
\Psi(\vec{R}) = \mathcal{A}(\vec{R}) \exp\left[J(\vec{R})\right]
\end{equation}

In this section we will detail the types and forms of Jastrow factor used 
in QMCPACK.  Note that each type of Jastrow factor needs to be specified using
its own individual jastrow XML element.  For this reason, we have repeated the
specification of the jastrow tag in each section, with specialization for the
options available for that given type of jastrow.

\subsection{One-body Jastrow functions}
\label{sec:onebodyjastrow}
The one-body Jastrow factor is a form that allows for the direct inclusion
of correlations between particles that are included in the wavefunction with
particles that are not explicitly part of it.  The most common example of
this are correlations between electrons and ions.  

The jastrow function is specified within a \texttt{wavefunction} element
and must contain one or more \texttt{correlation} elements specifying
additional parameters as well as the actual coefficients. Section
\ref{sec:1bjsplineexamples} gives examples of the typical nesting of
\texttt{jastrow}, \texttt{correlation}, and \texttt{coefficient} elements.

\subsubsection{Input Specification}

\begin{table}[h]
\begin{center}
\begin{tabular}{l c c c l }
\hline
\multicolumn{5}{l}{Jastrow element} \\
\hline
\bfseries name & \bfseries datatype & \bfseries values & \bfseries defaults  & \bfseries description \\
\hline
name & text &    & (required) & Unique name for this Jastrow function \\
type & text & One-body & (required) & Define a one-body function \\ 
function & text & Bspline & (required) & BSpline Jastrow \\
             & text & pade2 & & Pade form \\
             & text & \ldots & & \ldots \\
source & text & name & (required) & name of attribute of classical particle set \\ 
print & text & yes / no & yes & jastrow factor printed in external file?\\
  \hline
\multicolumn{5}{l}{elements}\\ \hline
& Correlation & & & \\ \hline
\multicolumn{5}{l}{Contents}\\ \hline
& (None)  & & &  \\ \hline
\end{tabular}
%\end{tabular*}
\end{center}
\end{table}

To be more concrete, the one-body jastrow factors used to describe correlations
between electrons and ions take the form below
\begin{equation}
J1=\sum_I^{ion0}\sum_i^e u_{ab}(|r_i-R_I|)
\end{equation}
where I runs over all of the ions in the calculation, i runs over the electrons
and $u_{ab}$ describes the functional form of the correlation between them.
Many different forms of $u_{ab}$ are implemented in QMCPACK.  We will detail 
two of the most common ones below.
\subsubsection{Spline form}
\label{sec:onebodyjastrowspline}

The one-body spline Jastrow function is the most commonly used one-body Jastrow for solids. This form 
was first described and used in \cite{EslerKimCeperleyShulenburger2012}.  
Here $u_{ab}$ is an interpolating 1D Bspline (tricublc spline on a linear grid) between zero distance and $r_{cut}$. In 3D periodic systems 
the default cutoff distance is the Wigner Seitz cell radius. For other periodicities including isolated 
molecules the $r_{cut}$ must be specified. The cusp can be set.   $r_i$ 
and $R_I$ are most commonly the electron and ion positions, but any particlesets that can provide the 
needed centers can be used.

\paragraph{Input Specification}
\begin{table}[h]
\begin{center}
\begin{tabular}{l c c c l }
\hline
\multicolumn{5}{l}{Correlation element} \\
\hline
\bfseries name & \bfseries datatype & \bfseries values & \bfseries defaults & \bfseries description \\
\hline
elementType & text & name & see below & Classical particle target  \\
speciesA & text & name & see below & Classical particle target \\
speciesB & text & name & see below & Quantum species target \\
size & integer & $> 0$ & (required) & Number of coefficients \\
rcut & real & $> 0$ & see below & Distance at which the correlation goes to 0 \\
cusp & real & $\ge 0$ & 0 & Value for use in Kato cusp condition \\
spin & text & yes or no & no & Spin dependent jastrow factor \\
\hline
\multicolumn{5}{l}{elements}\\ \hline
& Coefficients & & & \\ \hline
\multicolumn{5}{l}{Contents}\\ \hline
& (None)  & & &  \\ \hline
\end{tabular}
%\end{tabular*}
\end{center}
\end{table}

Additional information:

 \begin{itemize}
 \item \texttt{elementType, speciesA, speciesB, spin}.  For a spin independent Jastrow factor (spin = ``no'')
elementType should be the name of the group of ions in the classical particleset to which the quantum
particles should be correlated.  For a spin dependent Jastrow factor (spin = ``yes'') set speciesA to the
group name in the classical particleset and speciesB to the group name in the quantum particleset.
 \item \texttt{rcut}. The cutoff distance for the function in atomic units (bohr). 
For 3D fully periodic systems this parameter is optional and a default of the Wigner 
Seitz cell radius is used. Otherwise this parameter is required.
 \item \texttt{cusp}. The one body jastrow factor can be used to make the wavefunction
satisfy the electron-ion cusp condition\cite{kato}.  In this case, the derivative of the jastrow
factor as the electron approaches the nucleus will be given by:
\begin{equation}
\left(\frac{\partial J}{\partial r_{iI}}\right)_{r_{iI} = 0} = -Z
\end{equation}
Note that if the antisymmetric part of the wavefunction satisfies the electron-ion cusp
condition (for instance by using single particle orbitals that respect the cusp condition)
or if a non-divergent pseudopotential is used that the Jastrow should be cuspless at the 
nucleus and this value should be kept at its default of 0.
 \end{itemize}


\begin{table}[h]
\begin{center}
\begin{tabular}{l c c c l }
\hline
\multicolumn{5}{l}{Coefficients element} \\
\hline
\bfseries name & \bfseries datatype & \bfseries values & \bfseries defaults & \bfseries description \\
\hline
id & text & & (required) & Unique identifier \\
type & text & Array & (required) & \\
optimize & text & yes or no & yes & if no, values are fixed in optimizations \\
\hline
\multicolumn{5}{l}{elements}\\ \hline
(None) & & & \\ \hline
\multicolumn{5}{l}{Contents}\\ \hline
 (no name) & real array & & zeros & Jastrow coefficients \\ \hline
\end{tabular}
%\end{tabular*}
\end{center}
\end{table}


\paragraph{Example use cases}
\label{sec:1bjsplineexamples}

Specify a spin-independent function with four parameters. Because rcut  is not 
specified, the default cutoff of the Wigner Seitz cell radius is used; this 
Jastrow must be used with a 3D periodic system such as a bulk solid. The name of 
the particleset holding the ionic positions is "i".
\begin{lstlisting}[language=xml]
<jastrow name="J1" type="One-Body" function="Bspline" print="yes" source="i">
 <correlation elementType="C" cusp="0.0" size="4">
   <coefficients id="C" type="Array"> 0  0  0  0  </coefficients>
 </correlation>
</jastrow>
\end{lstlisting}

Specify a spin-dependent function with seven upspin and seven downspin parameters. 
The cutoff distance is set to 6 atomic units.  Note here that the particleset holding
the ions is labeled as ion0 rather than ``i'' in the other example.  Also in this case
the ion is Lithium with a coulomb potential, so the cusp condition is satisfied by 
setting cusp="d".
\begin{lstlisting}[language=xml]
<jastrow name="J1" type="One-Body" function="Bspline" source="ion0" spin="yes">
  <correlation speciesA="Li" speciesB="u" size="7" rcut="6">
    <coefficients id="eLiu" cusp="3.0" type="Array"> 
    0.0 0.0 0.0 0.0 0.0 0.0 0.0
    </coefficients>
  </correlation>
  <correlation speciesA="C" speciesB="d" size="7" rcut="6">
    <coefficients id="eLid" cusp="3.0" type="Array"> 
    0.0 0.0 0.0 0.0 0.0 0.0 0.0
    </coefficients>
  </correlation>
</jastrow>
\end{lstlisting}


\subsubsection{Pade form}
\label{sec:onebodyjastrowpade}

While the spline Jastrow factor is the most flexible and most commonly used form implemented in \qmcpack, 
there are times where its flexibility can make it difficult to optimize.  As an example, a spline jastrow
with a very large cutoff may be difficult to optimize for isolated systems like molecules due to the small
number of samples that will be present in the tail of the function.  In such cases, a simpler functional
form may be advantageous.  The second order Pade jastrow factor, given in Eq.\ref{padeeqn} is a good choice 
in such cases.  
\begin{equation}
\label{padeeqn}
u_{ab}(r) = \frac{a*r+c*r^2}{1+b*r}
\end{equation}
Unlike the spline jastrow factor which includes a cutoff, this form has an infinite range and for every particle
pair (subject to the minimum image convention) it will be applied.  It also is a cuspless jastrow factor,
so it should either be used in combination with a single particle basis set that contains the proper cusp or
with a smooth pseudopotential.

\paragraph{Input Specification}
\begin{table}[h]
\begin{center}
\begin{tabular}{l c c c l }
\hline
\multicolumn{5}{l}{Correlation element} \\
\hline
\bfseries name & \bfseries datatype & \bfseries values & \bfseries defaults & \bfseries description \\
\hline
elementType & text & name & see below & Classical particle target  \\
\hline
\multicolumn{5}{l}{elements}\\ \hline
& Coefficients & & & \\ \hline
\multicolumn{5}{l}{Contents}\\ \hline
& (None)  & & &  \\ \hline
\end{tabular}
%\end{tabular*}
\end{center}
\end{table}

\begin{table}[h]
\begin{center}
\begin{tabular}{l c c c l }
\hline
\multicolumn{5}{l}{parameter element} \\
\hline
\bfseries name & \bfseries datatype & \bfseries values & \bfseries defaults & \bfseries description \\
\hline
id & string & name & (required) & name for variable \\
name & string & A or B or C & (required) & see Eq.\ref{padeeqn}\\
optimize & text & yes or no & yes & if no, values are fixed in optimizations \\
\hline
\multicolumn{5}{l}{elements}\\ \hline
(None) & & & \\ \hline
\multicolumn{5}{l}{Contents}\\ \hline
 (no name) & real & parameter value & (required) & Jastrow coefficients \\ \hline
\end{tabular}
%\end{tabular*}
\end{center}
\end{table}

\paragraph{Example use case}
\label{sec:1bjpadeexamples}

Specify a spin independent function with independent jastrow factors for two different species (Li and H).
The name of the particleset holding the ionic positions is "i".
\begin{lstlisting}[style=XML]
<jastrow name="J1" function="pade2" type="One-Body" print="yes" source="i">
  <correlation elementType="Li">
    <var id="LiA" name="A">  0.34 </var>
    <var id="LiB" name="B"> 12.78 </var>
    <var id="LiC" name="C">  1.62 </var>
  </correlation>
  <correlation elementType="H"">
    <var id="HA" name="A">  0.14 </var>
    <var id="HB" name="B"> 6.88 </var>
    <var id="HC" name="C"> 0.237 </var>
  </correlation>
</jastrow>
\end{lstlisting}


\subsection{Two-body Jastrow functions}
The two-body Jastrow factor is a form that allows for the explicit inclusion
of dynamic correlation between two particles included in the wavefunction.  It
is almost always given in a spin dependent form so as to satisfy the Kato cusp
condition between electrons of different spins\cite{kato}.

 The two body jastrow function is specified within a \texttt{wavefunction} element
and must contain one or more correlation elements specifying additional parameters
as well as the actual coefficients.  Section \ref{sec:2bjsplineexamples} gives 
examples of the typical nesting of \texttt{jastrow}, \texttt{correlation} and
\texttt{coefficient} elements.

\subsubsection{Input Specification}

\begin{table}[h]
\begin{center}
\begin{tabular}{l c c c l }
\hline
\multicolumn{5}{l}{Jastrow element} \\
\hline
\bfseries name & \bfseries datatype & \bfseries values & \bfseries defaults  & \bfseries description \\
\hline
name & text &    & (required) & Unique name for this Jastrow function \\
type & text & Two-body & (required) & Define a one-body function \\ 
function & text & Bspline & (required) & BSpline Jastrow \\
print & text & yes / no & yes & jastrow factor printed in external file?\\
  \hline
\multicolumn{5}{l}{elements}\\ \hline
& Correlation & & & \\ \hline
\multicolumn{5}{l}{Contents}\\ \hline
& (None)  & & &  \\ \hline
\end{tabular}
%\end{tabular*}
\end{center}
\end{table}

The two-body jastrow factors used to describe correlations between electrons take the form
\begin{equation}
J2=\sum_i^{e}\sum_{j>i}^{e} u_{ab}(|r_i-r_j|)
\end{equation}

The most commonly used form of two body jastrow factor supported by the code is a splined
jastrow factor, with many similarities to the one body spline jastrow.

\subsubsection{Spline form}
\label{sec:twobodyjastrowspline}

The two-body spline Jastrow function is the most commonly used two-body Jastrow for solids. This form 
was first described and used in \cite{EslerKimCeperleyShulenburger2012}.  
Here $u_{ab}$ is an interpolating 1D Bspline (tricublc spline on a linear grid) between 
zero distance and $r_{cut}$. In 3D periodic systems 
the default cutoff distance is the Wigner Seitz cell radius. For other periodicities including isolated 
molecules the $r_{cut}$ must be specified.  $r_i$ and $r_j$ are typically electron positions.  The cusp 
condition as $r_i$ approaches $r_j$ is set by the relative spin of the electrons.

\FloatBarrier
\paragraph{Input Specification}

\FloatBarrier
\begin{table}[h!]
\begin{center}
\begin{tabular}{l c c c l }
\hline
\multicolumn{5}{l}{Correlation element} \\
\hline
\bfseries name & \bfseries datatype & \bfseries values & \bfseries defaults & \bfseries description \\
\hline
speciesA & text & u or d & (required) & Quantum species target \\
speciesB & text & u or d & (required) & Quantum species target \\
size & integer & $> 0$ & (required) & number of coefficients \\
rcut & real & $> 0$ & see below & distance at which the correlation goes to 0 \\
spin & text & yes or no & no & spin dependent jastrow factor \\
\hline
\multicolumn{5}{l}{elements}\\ \hline
& Coefficients & & & \\ \hline
\multicolumn{5}{l}{Contents}\\ \hline
& (None)  & & &  \\ \hline
\end{tabular}
%\end{tabular*}
\end{center}
\end{table}

\FloatBarrier

Additional information:
\begin{itemize}
\item \texttt{speciesA, speciesB} The scale function u(r) is defined for species pairs uu and ud.  
There is no need to define ud or dd since uu=dd and ud=du.  The cusp condition is computed internally 
based on the charge of the quantum particles.
\end{itemize}

\begin{table}[h]
\begin{center}
\begin{tabular}{l c c c l }
\hline
\multicolumn{5}{l}{Coefficients element} \\
\hline
\bfseries name & \bfseries datatype & \bfseries values & \bfseries defaults & \bfseries description \\
\hline
id & text & & (required) & Unique identifier \\
type & text & Array & (required) & \\
optimize & text & yes or no & yes & if no, values are fixed in optimizations \\
\hline
\multicolumn{5}{l}{elements}\\ \hline
(None) & & & \\ \hline
\multicolumn{5}{l}{Contents}\\ \hline
 (no name) & real array & & zeros & Jastrow coefficients \\ \hline
\end{tabular}
%\end{tabular*}
\end{center}
\end{table}

\paragraph{Example use cases}
\label{sec:2bjsplineexamples}

Specify a spin-dependent function with 4 parameters for each channel.  In this case, the cusp is set at 
a radius of 4.0 bohr (rather than to the default of the Wigner Seitz cell radius).  Also, in this example,
the coefficients are set to not be optimized during an optimization step.

\begin{lstlisting}
<jastrow name="J2" type="Two-Body" function="Bspline" print="yes">
  <correlation speciesA="u" speciesB="u" size="8" rcut="4.0">
    <coefficients id="uu" type="Array" optimize="no"> 0.2309049836 0.1312646071 0.05464141356 0.01306231516</coefficients>
  </correlation>
  <correlation speciesA="u" speciesB="d" size="8" rcut="4.0">
    <coefficients id="ud" type="Array" optimize="no"> 0.4351561096 0.2377951747 0.1129144262 0.0356789236</coefficients>
  </correlation>
</jastrow>
\end{lstlisting}



\subsection{Long-ranged Jastrow factors}
While short-ranged Jastrow factors capture the majority of the benefit 
for minimizing the total energy and the energy variance, long-ranged 
Jastrow factors are important to accurately reproduce the short-ranged 
(long wavelength) behavior of quantities such as the static structure 
factor, and are therefore essential for modern accurate finite size 
corrections in periodic systems.

Below two types of long-ranged Jastrow factors are described.  The 
first (the k-space Jastrow) is simply an expansion of the one and/or 
two body correlation functions in plane waves, with the coefficients 
comprising the optimizable parameters.  The second type have few 
variational parameters and use the optimized breakup method of Natoli 
and Ceperley\cite{Natoli1995} (the Yukawa and Gaskell RPA Jastrows).


\subsubsection{Long-ranged Jastrow: k-space Jastrow}
The k-space Jastrow introduces explicit long-ranged dependence commensurate with the periodic supercell.  This Jastrow is to be used in periodic boundary conditions only.  

The input for the k-space Jastrow fuses both one and two-body forms into a single element and so they are discussed together here.  The one- and two-body terms in the k-Space Jastrow have the form:
\begin{align}
  J_1 &= \sum_{G\ne 0}b_G\rho_G^I\rho_{-G} \\
  J_2 &= \sum_{G\ne 0}a_G\rho_G\rho_{-G}
\end{align}
Here $\rho_G$ is the Fourier transform of the instantaneous electron density:
\begin{align}
  \rho_G=\sum_{n\in electrons}e^{iG\cdot r_n}
\end{align}
and $\rho_G^I$ has the same form, but for the fixed ions. In both cases the coefficients are restricted to be real, though in general the coefficients for the one-body term need not be.  See section \ref{sec:feature_kspace_jastrow} for more detail.

Input for the k-space Jastrow follows the familar nesting of \texttt{jastrow-correlation-coefficients} elements, with attributes unique to the k-space Jastrow at the \texttt{correlation} input level.

\FloatBarrier
\begin{table}[h]
\begin{center}
\begin{tabularx}{\textwidth}{l l l l l l }
\hline
\multicolumn{6}{l}{\texttt{jastrow type=kSpace} element} \\
\hline
\multicolumn{2}{l}{parent elements:} & \multicolumn{4}{l}{\texttt{wavefunction}}\\
\multicolumn{2}{l}{child  elements:} & \multicolumn{4}{l}{\texttt{correlation}}\\
\multicolumn{2}{l}{attributes}  & \multicolumn{4}{l}{}\\
   &   \bfseries name     & \bfseries datatype & \bfseries values          & \bfseries default  & \bfseries description \\
   & \texttt{type}$^r$    &  text              & \textbf{kSpace}           &                    & Must be kSpace           \\
   & \texttt{name}$^r$    &  text              & \textit{anything}         & 0                  & Unique name for Jastrow \\
   & \texttt{source}$^r$  &  text              & \texttt{particleset.name} &                    & Ion particleset name\\
  \hline
\end{tabularx}
\end{center}
\end{table}
\FloatBarrier

\FloatBarrier
\begin{table}[h]
\begin{center}
\begin{tabularx}{\textwidth}{l l l l l l }
\hline
\multicolumn{6}{l}{\texttt{correlation} element} \\
\hline
\multicolumn{2}{l}{parent elements:} & \multicolumn{4}{l}{\texttt{jastrow type=kSpace}}\\
\multicolumn{2}{l}{child  elements:} & \multicolumn{4}{l}{\texttt{coefficients}}\\
\multicolumn{2}{l}{attributes}  & \multicolumn{4}{l}{}\\
   &   \bfseries name           & \bfseries datatype & \bfseries values  & \bfseries default  & \bfseries description \\
   & \texttt{type}$^r$          &  text              & \textbf{One-Body},\textbf{Two-Body}    &                     & Must be One-Body/Two-Body     \\
   & \texttt{kc}$^r$            &  real              & kc$\ge$ 0                                & 0.0                 & k-space cutoff in a.u. \\
   & \texttt{symmetry}$^o$      &  text              & crystal,isotropic,none                 & crystal             & Symmetry of coefficients\\
   & \texttt{spinDependent}$^o$ &  boolean           & yes,no                                 & no                  & \textit{No current function} \\
  \hline
\end{tabularx}
\end{center}
\end{table}
\FloatBarrier

\FloatBarrier
\begin{table}[h]
\begin{center}
\begin{tabularx}{\textwidth}{l l l l l l }
\hline
\multicolumn{6}{l}{\texttt{coefficients} element} \\
\hline
\multicolumn{2}{l}{parent elements:} & \multicolumn{4}{l}{\texttt{correlation}}\\
\multicolumn{2}{l}{child  elements:} & \multicolumn{4}{l}{\textit{None}}\\
\multicolumn{2}{l}{attributes}  & \multicolumn{4}{l}{}\\
   &   \bfseries name     & \bfseries datatype & \bfseries values  & \bfseries default   & \bfseries description \\
   & \texttt{id}$^r$      &  text              & \textit{anything} &     cG1/cG2         & Label for coeffs     \\
   & \texttt{type}$^r$    &  text              & \texttt{Array}    &   0                 & Must be Array \\
\multicolumn{2}{l}{body text}  & \multicolumn{4}{l}{}\\
   &                           & \multicolumn{4}{l}{The body text is a list of real values for the parameters.}     \\
  \hline
\end{tabularx}
\end{center}
\end{table}
\FloatBarrier


Additional information:
\begin{itemize}
  \item{It is normal to provide no coefficients as an initial guess.  The number of coefficients will be automatically calculated according to the k-space cutoff + symmetry and set to zero. }
  \item{Providing an incorrect number of parameters also results in all parameters being set to zero.}
  \item{There is currently no way to turn optimization on/off for the k-space Jastrow.  The coefficients are always optimized.}
  \item{Spin dependence is currently not implemented for this Jastrow.}
  \item{\texttt{kc}: Parameters with G vectors magnitudes less than \texttt{kc} are included in the Jastrow.  If \texttt{kc} is zero, it is the same as excluding the k-space term.}
  \item{\texttt{symmetry=crystal}: Impose crystal symmetry on coefficients according to the structure factor.}
  \item{\texttt{symmetry=isotropic}: Impose spherical symmetry on coefficients according to G-vector magnitude.}
  \item{\texttt{symmetry=none}: Impose no symmetry on the coefficients.}
\end{itemize}


\begin{lstlisting}[caption=k-space Jastrow with one- and two-body terms.]
  <jastrow type="kSpace" name="Jk" source="ion0">
     <correlation kc="4.0" type="One-Body" symmetry="cystal">
        <coefficients id="cG1" type="Array">                  
        </coefficients>
     </correlation>
     <correlation kc="4.0" type="Two-Body" symmetry="crystal">
        <coefficients id="cG2" type="Array">                  
        </coefficients>
     </correlation>
  </jastrow>
\end{lstlisting}

\begin{lstlisting}[caption=k-space Jastrow with one-body term only.]
  <jastrow type="kSpace" name="Jk" source="ion0">
     <correlation kc="4.0" type="One-Body" symmetry="cystal">
        <coefficients id="cG1" type="Array">                  
        </coefficients>
     </correlation>
  </jastrow>
\end{lstlisting}

\begin{lstlisting}[caption=k-space Jastrow with two-body term only.]
  <jastrow type="kSpace" name="Jk" source="ion0">
     <correlation kc="4.0" type="Two-Body" symmetry="crystal">
        <coefficients id="cG2" type="Array">                  
        </coefficients>
     </correlation>
  </jastrow>
\end{lstlisting}




\subsubsection{Long-ranged Jastrows: Gaskell RPA and Yukawa forms}
\label{sec:twobodyjastrowlr}
\textbf{NOTE: The Yukawa and RPA Jastrows do not work at present 
and are currently being revived.  Please contact the developers if 
you are interested in using them.} 

The exact Jastrow correlation functions contain terms which have a 
form similar to the Coulomb pair potential.  In periodic systems 
the Coulomb potential is replaced by an Ewald summation of the 
bare potential over all periodic image cells.  This sum is often 
handled by the optimized breakup method\cite{Natoli1995} and this 
same approach is applied to the long-ranged Jastrow factors in QMCPACK.

There are two main long-ranged Jastrow factors of this type 
implemented in QMCPACK: the Gaskell RPA\cite{Gaskell1961,Gaskell1962} 
form and the Yukawa\cite{Ceperley1978} form.  Both of these forms 
were used by Ceperley in early studies of the electron gas\cite{Ceperley1978}, 
but they are also appropriate starting points for general solids. 

The Yukawa form is defined in real space.  It's long-range form is 
formally defined as
\begin{align}
  u_Y^{PBC}(r) = \sum_{L\ne 0}\sum_{i<j}u_Y(\abs{r_i-r_j+L})
\end{align}
with $u_Y(r)$ given by
\begin{align}
  u_Y(r) = \frac{a}{r}\left(1-e^{-r/b}\right)
\end{align}
In QMCPACK a slightly more restricted form is used:
\begin{align}
  u_Y(r) = \frac{r_s}{r}\left(1-e^{-r/\sqrt{r_s}}\right)
\end{align}
here ``$r_s$'' is understood to be a variational parameter.

The Gaskell RPA form--which contains correct short/long range limits 
and minimizes the total energy of the electron gas within the RPA--is 
defined directly in k-space:
\begin{align}
  u_{RPA}(k) = -\frac{1}{2S_0(k)}+\frac{1}{2}\left(\frac{1}{S_0(k)^2}+\frac{4m_ev_k}{\hbar^2k^2}\right)^{1/2}
\end{align}
where $v_k$ is the Fourier transform of the Coulomb potential and 
$S_0(k)$ is the static structure factor of the non-interacting 
electron gas:
\[
  S_0(k) = \left.
  \begin{cases}
    1 &  k>2k_F \\
    \frac{3k}{4k_F}-\frac{1}{2}\left(\frac{k}{2k_F}\right)^3 & k<2k_F
  \end{cases}
  \right.
\]
When written in atomic units, RPA Jastrow implemented in QMCPACK has the 
form
\begin{align}
  u_{RPA}(k) = \frac{1}{2N_e}\left(-\frac{1}{S_0(k)}+\left(\frac{1}{S_0(k)^2}+\frac{12}{r_s^3k^4}\right)^{1/2}\right)
\end{align}
Here ``$r_s$'' is again a variational parameter and $k_F\equiv(\tfrac{9\pi}{4r_s^3})^{1/3}$.

For both the Yukawa and Gaskell RPA Jastrows, the default value for $r_s$ is $r_s=(\tfrac{3\Omega}{4\pi N_e})^{1/3}$.


\FloatBarrier
\begin{table}[h]
\begin{center}
\begin{tabularx}{\textwidth}{l l l l l l }
\hline
\multicolumn{6}{l}{\texttt{jastrow type=Two-Body function=rpa/yukawa} element} \\
\hline
\multicolumn{2}{l}{parent elements:} & \multicolumn{4}{l}{\texttt{wavefunction}}\\
\multicolumn{2}{l}{child  elements:} & \multicolumn{4}{l}{\texttt{correlation}}\\
\multicolumn{2}{l}{attributes}  & \multicolumn{4}{l}{}\\
   &   \bfseries name        & \bfseries datatype & \bfseries values             & \bfseries default  & \bfseries description   \\
   & \texttt{type}$^r$       &  text              & \textbf{Two-Body}            &                    & Must be Two-Body   \\
   & \texttt{function}$^r$   &  text              & \textbf{rpa}/\textbf{yukawa} &                    & Must be rpa or yukawa   \\
   & \texttt{name}$^r$       &  text              & \textit{anything}            & RPA\_Jee            & Unique name for Jastrow \\
   & \texttt{longrange}$^o$  &  boolean           & yes/no                       & yes                & Use long-range part     \\
   & \texttt{shortrange}$^o$ &  boolean           & yes/no                       & yes                & Use short-range part    \\
\multicolumn{2}{l}{parameters}  & \multicolumn{4}{l}{}\\
   & \texttt{rs}$^o$         &  rs                & $r_s>0$                      & $\tfrac{3\Omega}{4\pi N_e}$ & Avg. elec-elec distance \\
   & \texttt{kc}$^o$         &  kc                & $k_c>0$                      & $2\left(\tfrac{9\pi}{4}\right)^{1/3}\tfrac{4\pi N_e}{3\Omega}$ & K-space cutoff\\
  \hline
\end{tabularx}
\end{center}
\end{table}
\FloatBarrier


\begin{lstlisting}[caption=Two body RPA Jastrow with long- and short-ranged parts.]
<jastrow name=''Jee'' type=''Two-Body'' function=''rpa''>
</jastrow>
\end{lstlisting}



% J1 RPA (intended for electron-proton system)
%   source = particleset.name
%   function = RPA
%   name = anything [Jep]
%   rs = >0 [-1]
%   kc = >0 [-1]
% 
% J2 RPA
%   attributes
%     function = yukawa or rpa
%     name = anything [RPA_Jee]
%     longrange = yes/no [yes]
%     shortrange = yes/no [yes]
%   parameters
%     rs = >0 [-1]  3\Omega/4\pi N_e
%     kc = >0 [-1]  2 (9\pi/4)^1/3 * 4\pi N_e/3\Omega



\subsection{Three-body Jastrow functions}
Explicit three body correlations can be included in the wavefunction via the three-body
jastrow factor.

