\section{Backflow wavefunctions}

\label{sec:backflow}

One can perturb the nodal surface of a single-slater/multi-slater wavefunction through use of a backflow-transformation.  Specifically, if we have an antisymmetric function $D(\mathbf{x}_{0\uparrow},\cdots,\mathbf{x}_{N\uparrow}, \mathbf{x}_{0\downarrow},\cdots,\mathbf{x}_{N\downarrow})$, and if $i_\alpha$ is the $i$-th particle of species type $\alpha$, then the backflow transformation works by making the coordinate transformation $\mathbf{x}_{i_\alpha} \to \mathbf{x}'_{i_\alpha}$, and evaluating $D$ at these new ``quasiparticle" coordinates.  QMCPACK currently supports quasiparticle transformations given by:

\begin{equation}\label{backflowdef}
\mathbf{x}'_{i_\alpha}=\mathbf{x}_{i_\alpha}+\sum_{\alpha \leq \beta} \sum_{i_\alpha \neq j_\beta} \eta^{\alpha\beta}(|\mathbf{x}_{i_\alpha}-\mathbf{x}_{j_\beta}|)(\mathbf{x}_{i_\alpha}-\mathbf{x}_{j_\beta})
\end{equation}

Here, $\eta^{\alpha\beta}(|\mathbf{x}_{i_\alpha}-\mathbf{x}_{j_\beta}|)$ is a radially symmetric back flow transformation between species $\alpha$ and $\beta$.  In QMCPACK, particle $i_\alpha$ is known as the ``target" particle and $j_\beta$ is known as the ``source".  The main types of transformations are so called one-body terms, which are between an electron and an ion $\eta^{eI}(|\mathbf{x}_{i_e}-\mathbf{x}_{j_I}|)$, and two-body terms.  Two body terms are distinguished as those between like and opposite spin electrons:  $\eta^{e(\uparrow)e(\uparrow)}(|\mathbf{x}_{i_e(\uparrow)}-\mathbf{x}_{j_e(\uparrow)}|)$ and  $\eta^{e(\uparrow)e(\downarrow)}(|\mathbf{x}_{i_e(\uparrow)}-\mathbf{x}_{j_e(\downarrow)}|)$.  Henceforth, we will assume that $\eta^{e(\uparrow)e(\uparrow)}=\eta^{e(\downarrow)e(\downarrow)}$.

In the following, we explain how to describe general terms like Eq. \ref{backflowdef} in a QMCPACK XML file.  For specificity, we will consider a particle set consisting of H and He (in that order).  This ordering will be important when we build the XML file, so you can find this out either through your specific declaration of <particleset>, by looking at the hdf5 file in the case of plane waves, or by looking at the qmcpack output file in the section labelled ``Summary of QMC systems".  
\subsection{Input Specifications}
All backflow declarations occur within a single \texttt{<backflow> ... </backflow>} block.  Backflow transformations occur in \texttt{<transformation>} blocks, and have the following input parameters

\begin{table}[h]
\begin{center}
\begin{tabular}{l c c c l }
\hline
\multicolumn{5}{l}{Transformation element} \\
\hline
\bfseries name & \bfseries datatype & \bfseries values & \bfseries defaults  & \bfseries description \\
\hline
name & text &  & (required) & Unique name for this Jastrow function \\
type & text & ``e-I" & (required) & Define a one-body backflow transformation. \\ 
        &          & ``e-e" & & Define a two-body backflow transformation. \\
function & text & Bspline & (required) & B-spline type transformation. (No other types supported) \\
source & text &  &  & ``e" if two-body, ion particle set if one-body.\\ 
  \hline
\end{tabular}
%\end{tabular*}
\end{center}
\end{table}

Just like one and two-body jastrows, parameterization of the backflow transformations are specified within the \texttt{<transformation> } blocks by  \texttt{<correlation>} blocks.  Please refer to \ref{sec:onebodyjastrowspline} for more information.

\subsection{Example Use Case}
Having specified the general form, we present a general example of one-body and two-body backflow transformations in a hydrogen-helium mixture.  The H and He ions have independent backflow transformations, as do the like and unlike-spin two-body terms.  One caveat is in order:  ionic backflow transformations must be listed in the order that they appear in the particle set.  If in our example, He is listed first, and H is listed second, the following example would be correct.  However, switching backflow declaration to H first, then He, will result in an error.  Outside of this, declaration of one-body blocks and two-body blocks aren't sensitive to ordering.  

\begin{lstlisting}[style=XML]
<backflow>
<!--The One-Body term with independent e-He and e-H terms. IN THAT ORDER -->
<transformation name="eIonB" type="e-I" function="Bspline" source="ion0">
    <correlation cusp="0.0" size="8" type="shortrange" init="no" elementType="He" rcut="3.0">
        <coefficients id="eHeC" type="Array" optimize="yes"> 
            0 0 0 0 0 0 0 0
        </coefficients>
    </correlation>
    <correlation cusp="0.0" size="8" type="shortrange" init="no" elementType="H" rcut="3.0">
        <coefficients id="eHC" type="Array" optimize="yes"> 
            0 0 0 0 0 0 0 0
        </coefficients>
    </correlation>
</transformation>

<!--The Two-Body Term with Like and Unlike Spins -->
<transformation name="eeB" type="e-e" function="Bspline" >
    <correlation cusp="0.0" size="7" type="shortrange" init="no" speciesA="u" speciesB="u" rcut="1.2">
        <coefficients id="uuB1" type="Array" optimize="yes"> 
            0 0 0 0 0 0 0
        </coefficients>
    </correlation>
    <correlation cusp="0.0" size="7" type="shortrange" init="no" speciesA="d" speciesB="u" rcut="1.2">
        <coefficients id="udB1" type="Array" optimize="yes"> 
            0 0 0 0 0 0 0
        </coefficients>
    </correlation>
</transformation>
</backflow>
\end{lstlisting}  

Currently, backflow only works with single-slater determinant wavefunctions.  When a backflow transformation has been declared, it should be placed within the \texttt{<determinantset>} block, but outside of the \texttt{<slaterdeterminant>} blocks, like so:

\begin{lstlisting}[style=XML]
<determinantset ... >
    <!--basis set declarations go here, if there are any -->
    
    <backflow>
        <transformation ...>
          <!--Here is where one and two-body terms are defined -->
         </transformation>
     </backflow>
     
     <slaterdeterminant>
         <!--Usual determinant definitions -->
     </slaterdeterminant>
 </determinantset>
\end{lstlisting}

\subsection{Optimization Tips}
Backflow is notoriously difficult to optimize--its extremely nonlinear in the variational parameters and also moves the nodal surface around.  As such, it is likely that a full Jastrow+Backflow optimization with all parameters initialized to zero may not converge in a reasonable time.  If you are experiencing this problem, the following pointers are suggested (in no particular order):

\subsubsection{Get a good starting guess for $\Psi_T$}

\begin{enumerate}
\item Try optimizing the Jastrow first without backflow.
\item Freeze the jastrow parameters, introduce only the e-e terms in the backflow transformation, and optimize these parameters.
\item Freeze the e-e backflow parameters, and now optimize the e-I terms.
  \begin{itemize}
    \item If difficulty is encountered here, try optimizing each species independently.
  \end{itemize}
\item Now, unfreeze all jastrow, e-e backflow, and e-I backflow parameters, and reoptimize.  

\end{enumerate}


\subsubsection{Optimizing Backflow Terms}

It is wholly possible that the above prescription might grind to a halt in steps 2 or 3 with the inability to optimize the e-e or e-I backflow transformation independently, especially if it is initialized to zero.  To get around this, one suggestion is to build a good starting guess for the e-e or e-I backflow terms iteratively as follows.  

\begin{enumerate}
\item Start off with a small number of knots initialized to zero.  Set $r_{cut}$ to be small (much smaller than an interatomic distance).
\item Optimize the backflow function.
\item If this works, slowly increase $r_{cut}$ and/or the number of knots.
\item Repeat steps 2-3 until there is no noticeable change in energy or variance of $\Psi_T$.
\end{enumerate}

\subsubsection{Tweaking the Optimization Run}

The following modifications are worth a try in the optimization block:

\begin{itemize}
\item Try setting ``useDrift" to ``no".  This eliminates the use of wavefunction gradients and force biasing in the VMC algorithm.  This could be an issue for poorly optimized wavefunctions with pathological gradients.  
\item Try increasing ``exp0" in the optimization block.  Larger values of "exp0" cause the search directions to more closely follow those predicted by steepest-descent than those by the linear method.
\end{itemize}

Note that the new adaptive shift optimizer has not yet been tried with backflow wavefunctions. It should perform better than the older optimizers, but a considered optimization process is still recommended.


