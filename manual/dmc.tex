\section{Diffusion Monte Carlo}
\label{sec:dmc}
\pagebreak
\begin{table}[h]
\begin{center}
\begin{tabularx}{\textwidth}{l l l l l l }
\hline
\multicolumn{6}{l}{\texttt{dmc} method} \\
\hline
\multicolumn{2}{l}{parameters}  & \multicolumn{4}{l}{}\\
   &   \bfseries name     & \bfseries datatype & \bfseries values & \bfseries default   & \bfseries description \\
   &   \texttt{targetwalkers             } &  integer  & $> 0$ & dep.   & number of walkers per node  \\
   &   \texttt{blocks              } &  integer  & $\ge 0$ & 1   & number of blocks            \\
   &   \texttt{steps               } &  integer  & $\ge 0$ & 1   & number of steps per block   \\
   &   \texttt{warmupsteps         } &  integer  & $\ge 0$ & 0   & number of steps for warming up\\
   &   \texttt{timestep            } &  real     & $> 0$ & 0.1 & time step for each electron move \\
  % &   \texttt{samples             } &  real  & $\ge 0$ & 0   & total number of samples \\
%   &   \texttt{stepsbetweensamples } &  integer  & $> 0$ & 1   & period of the sample accumulation\\
%   &   \texttt{samplesperthread    } &  real  & $\ge 0$ & 0   & number of samples per thread  \\
%   &   \texttt{rewind              } &  integer  & $\ge 0$ & 0   & number of blocks to roll back   \\
%   &   \texttt{storeconfigs        } &  integer  & $\ge 0$ & 0   & whether to store samples  \\
   &   \texttt{checkproperties     } &  integer  & $\ge 0$ & 100   & number of steps between walker updates  \\
%  &   \texttt{recordwalkers       } &  integer  & $\ge 0$ & 0   & number of steps between saving a sample configuration. (only for VMC)  \\
%   &   \texttt{recordconfigs       } &  integer  & $\ge 0$ & 0   & number of steps between dumping a configuration to h5  \\
%   &   \texttt{current             } &  integer  & $\ge 0$ & 0   & current step (only used in optimization runs)  \\
%   &   \texttt{dmcwalkersperthread } &  real  & $\ge 0$ & 0   & number of samples per thread  \\
   &   \texttt{maxcpusecs          } &  real  & $\ge 0$ & 3.6e5   & maximum allowed walltime in seconds \\
   &   \texttt{energyUpdateInterval} &  integer  & $\ge 0$ & 0   & trial energy update interval \\
   &   \texttt{refEnergy           } &  AU  & all values & dep.   & reference energy  \\
   &   \texttt{feedback            } &  double  & $\ge 0$ & 1.0   & population feedback on the trial energy \\
   &   \texttt{useBareTau          } &  option  & yes,no & 0   & do not use effective time step  \\
   &   \texttt{warmupByReconfiguration} &  option  & yes,no & 0   & warm up with a fixed population  \\
 %  &   \texttt{energyBound         } &  double  & $\ge 0$ & 0   & number of samples per thread  \\
   &   \texttt{sigmaBound          } &  double  & $\ge 0$  & 10   & parameter to cutoff large weights  \\
   &   \texttt{killnode            } &  string  & yes/other & no   & kill or reject walkers that cross nodes  \\
  % &   \texttt{benchmark           } &  string  & $\ge 0$ & 0   & number of sample \\
   &   \texttt{reconfiguration     } &  string  & yes/pure/other & no   & fixed population technique  \\
   &   \texttt{branchInterval      } &  integer  & $\ge 0$ & 1   & branching interval \\
   &   \texttt{substeps            } &  integer  & $\ge 0$ & 1   & branching interval \\
   &   \texttt{nonlocalmoves       } &  string  & yes/other & no   & run with tmoves  \\
   &   \texttt{scaleweight         } &  string  & yes/other & yes   & scale weights (CUDA only)  \\
   &   \texttt{MaxAge              } &  double  & $\ge 0$ & 10   & kill persistent walkers  \\
    &   \texttt{MaxCopy             } &  double  & $\ge 0$ &2   & limit population growth \\
   &   \texttt{fastgrad            } &  text  & yes/other & yes   & fast gradients  \\
 %  &   \texttt{printderivs         } &  text  & $\ge 0$ & 0   & number of samples per  thread  \\
 %  &   \texttt{wlen                } &  integer  & $\ge 0$ & 0   & number of samples per  thread  \\
   &   \texttt{maxDisplSq      } &  real  & all values & -1   & maximum particle move  \\
   &   \texttt{storeconfigs        } &  integer  & all values & 0   & store configurations  \\
   &   \texttt{blocks\_between\_recompute} &  integer  & $\ge 0$ & dep.  & wavefunction recompute frequency  \\
  \hline
\end{tabularx}
\end{center}
\end{table}

Additional information:
\begin{itemize}
\item \texttt{targetwalkers}.  A DMC run can be considered a restart run or a new run.  A restart run is considered to be any method block beyond the first one, such as when a DMC method block that follows a VMC block.  Alternatively,  if the user reads in configurations from disk it is also considered a restart run.  In the case of a restart run, the DMC driver will use the configurations from the previous run, and this variable will not be used.  For a new run, if the number of walkers is less than the number of threads, then the number of walkers will be set equal to the number of threads.  

\item \texttt{blocks}. Number of blocks run during an DMC method block.  A block consists of a number of DMC steps (steps), after which all the statistics accumulated in the block are written to disk.

\item \texttt{steps}. Number of diffusion Monte Carlo steps in a block.

\item \texttt{warmupsteps}. Warm-up steps are steps at the beginning of a DMC run in which the 
instantaneous average energy is used to update the trial energy.  During regular steps, E$_{ref}$ is used.

\item \texttt{timestep}. The timestep determines the accuracy of the imaginary time propagator.  Generally, multiple time steps are used to extrapolate to the infinite time step limit.   A good range of timesteps  in which to perform time step extrapolation will typically have  a minimum of 99\% acceptance probability for each step.

\item \texttt{checkproperties}.  When using particle by particle driver, this variable specifies how often to reset all the variables kept in the buffer.

\item \texttt{maxcpusecs}. The default is 100 hours. Once the specified time has elapsed, the program will finalize the simulation even if not all blocks are completed.

\item \texttt{energyUpdateInterval}. The default is to update the trial energy at every step. Otherwise the trial energy is updated every \texttt{energyUpdateInterval} steps.

\[
E_{\text{trial}}=
\textrm{refEnergy}+\textrm{feedback}\cdot(\ln\textrm{targetWalkers}-\ln N)
\]
where $N$ is the current population.

\item \texttt{refEnergy}. The default reference energy is taken from the VMC run that precedes the DMC run. This value is updated to the current mean whenever branching happens.

\item \texttt{feedback}. Variable used to determine how strong to react to population fluctuations when doing population control.  See the equation in energyUpdateInterval for more details.

\item \texttt{useBareTau}. The same time step is used whether a move is rejected to not. The default is to use an effective time step when a move is rejected.

\item \texttt{warmupByReconfiguration}.  Warmup DMC is done with a fixed population

\item \texttt{sigmaBound}.  Determine the branch cutoff to limit wild weights based on the sigma and sigmaBound

\item \texttt{killnode}.  When running fixed-node, if a walker attempts to cross a node, the move will normally be rejected.  If killnode = "yes", then walkers are destroyed when they cross a node.

%\item \texttt{benchmark}. 

\item \texttt{reconfiguration}.  If reconfiguration is "yes", then run with a fixed walker population using the reconfiguration technique.  

\item \texttt{branchInterval}. Number of steps between branching.  The total number of DMC steps in a block will be BranchInterval*Steps.   

\item \texttt{substeps}.  Same as BranchInterval.


\item \texttt{nonlocalmoves}. DMC driver for running Hamiltonians with non-local moves.  An typical usage is to simulate Hamiltonians with non-local psuedopotentials with T-Moves.  Setting this equal to false will impose the locality approximation.

\item \texttt{scaleweight}. Scaling weight per Umrigar/Nightengale.  CUDA only.

\item \texttt{MaxAge}. Set the weight of a walker to min(currentweight,0.5) after a walker has not moved for MaxAge steps.  Needed if persistent walkers appear during the course of a run.

\item \texttt{MaxCopy}. When determining the number of copies of a walker to branch, set the number of copies equal to min(Multiplicity,MaxCopy).

\item \texttt{fastgrad}. Calculates gradients with either the fast version or the full-ratio version.

\item \texttt{maxDisplSq}.  When running a DMC calculation with particle by particle, this sets the maximum displacement allowed for a single particle move.  All distance displacements larger than the max is rejected.  If initialized to a negative value, it becomes equal to Lattice(LR/rc).

\item \texttt{sigmaBound}.  Determine the branch cutoff to limit wild weights based on the sigma and sigmaBound

%\item \texttt{rewind}. \textit{This input is recorded by QMCDriver.cpp, but is never used anywhere else.}

\item \texttt{storeconfigs}. If storeconfigs is set to a non-zero value, then electron configurations during the DMC run will be saved. This option is disabled for the OpenMP version of DMC.

\item \texttt{blocks\_between\_recompute}. See details in VMC section~\ref{sec:vmc}.

%\item \texttt{recordwalkers}. In VMC this is equivalent for \texttt{stepsbetweensamples}. \textit{This input is not used in DMC.}

%\item \texttt{recordconfigs}. \textit{This input is recorded by QMCDriver.cpp, but is never used anywhere else.}

%\item \texttt{current}. \textit{Only used in QMCLinearOptimize.cpp and QMCOptimize.cpp
%}
%\item \texttt{dmcwalkersperthread}. \textit{This input is only used in VMC.} It is equivalent to \texttt{samplesperthread}.

%\item \texttt{usedrift}. The VMC is implemented in two algorithms with or without drift. In the no-drift algorithm, the move of each electron is proposed with a Gaussian distribution. The standard deviation is chosen as the timestep input. In the drift algorithm, electrons are moved by langevin dynamics.




%\item \texttt{stepsbetweensamples}. Due to the fact that samples generated by consecutive steps might be still correlated. Having stepsbetweensamples larger than 1 reduces that correlation. In practice, using larger substeps is cheaper than using stepsbetweensamples to decorrelate samples.

%\item \texttt{samples}. This is the total amount of samples generated in the current VMC session. This parameter is not important for VMC only calculation but necessary if optimization or DMC follows.
%\[
%\textrm{samples}=
%\frac{\textrm{blocks}\cdot\textrm{steps}\cdot\textrm{walkers}}{\textrm{stepsbetweensamples}}\cdot\textrm{number of MPI tasks}
%\]

%\item \texttt{samplesperthread}. This is an alternative way to set the target amount of samples. More useful in the VMC session preparing the population for the following DMC calculation.
%\[
%\textrm{samplesperthread}=
%\frac{\textrm{blocks}\cdot\textrm{steps}}{\textrm{stepsbetweensamples}}
%\]

\end{itemize}

\begin{lstlisting}[caption=The following is an example of a very simple DMC section. ]
  <qmc method="dmc" move="pbyp" target="e">
    <parameter name="blocks">100</parameter>
    <parameter name="steps">400</parameter>
    <parameter name="timestep">0.010</parameter>
    <parameter name="warmupsteps">100</parameter>
  </qmc>
\end{lstlisting}
The time step should be adjusted for each problem individually.  Please refer to the theory section
on diffusion Monte Carlo.


\begin{lstlisting}[caption=The following is an example of running a simulation that can be restarted . ]
  <qmc method="dmc" move="pbyp"  checkpoint="0">
    <parameter name="timestep">         0.004  </parameter>
    <parameter name="blocks">           100   </parameter>
    <parameter name="steps">            400    </parameter>
  </qmc>
\end{lstlisting}
The checkpoint flag instructs qmcpack to output walker configurations.  This also
works in Variational Monte Carlo.  This will output an h5 file with the name "projectid"."run-number".config.h5.
Check that this file exists before attempting a restart.
To read in this file for a continuation run, specify the following:
\begin{lstlisting}[caption=Restart (read walkers from previous run) ]
 <mcwalkerset fileroot="BH.s002" version="0 6" collected="yes"/>
\end{lstlisting}
BH is the project id and s002 is the calculation number to read in the walkers from the previous run.\\

Combining VMC and DMC in a single run (and wave function optimization can be combined in this way too) is the standard way in which QMCPACK is typical run.   There is no need to run two separate jobs, as method sections can be stacked, and walkers are transferred between them.

\begin{lstlisting}[caption=Combined VMC and DMC run ]
  <qmc method="vmc" move="pbyp" target="e">
    <parameter name="blocks">100</parameter>
    <parameter name="steps">4000</parameter>
    <parameter name="warmupsteps">100</parameter>
    <parameter name="samples">1920</parameter>
    <parameter name="walkers">1</parameter>
    <parameter name="timestep">0.5</parameter>
  </qmc>
  <qmc method="dmc" move="pbyp" target="e">
    <parameter name="blocks">100</parameter>
    <parameter name="steps">400</parameter>
    <parameter name="timestep">0.010</parameter>
    <parameter name="warmupsteps">100</parameter>
  </qmc>
  <qmc method="dmc" move="pbyp" target="e">
    <parameter name="warmupsteps">500</parameter>
    <parameter name="blocks">50</parameter>
    <parameter name="steps">100</parameter>
    <parameter name="timestep">0.005</parameter>
  </qmc>
\end{lstlisting}




