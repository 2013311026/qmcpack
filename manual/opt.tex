\section{Wavefunction Optimization}
\label{sec:optimization}

Optimizing wavefunction is critical in all kinds of real-space quantum Monte Carlo calculations
because it significantly improves both the accuracy and efficiency of computation.
However, it is very difficult to directly adopt deterministic minimization approaches due to the stochastic nature of evaluating quantities with Monte Carlo.
Thanks to the algorithmic breakthrough during the first decade of this century and the tremendous computer power available, 
it becomes feasible to optimize tens of thousands of parameters in a wavefunction for a solid or molecule.
QMCPACK has multiple optimizers implemented based on the state-of-the-art linear method.
We are continually improving our optimizers for the robustness and friendliness and trying to provide a single solution.
Due to the large variation of wavefunction types carrying distinct characteristics, using several optimizer may be needed in some cases.
It is highly suggested to read the recommendation from the experts maintaining these optimizers.

A typical optimization block looks like the following. It starts with method=``linear" and contains three blocks of parameters.
\begin{lstlisting}
 <loop max="10">
  <qmc method="linear" move="pbyp" gpu="yes">
    <!-- Specify the VMC options -->
    <parameter name="walkers">              256 </parameter>
    <parameter name="samples">          2867200 </parameter>
    <parameter name="stepsbetweensamples">    1 </parameter>
    <parameter name="substeps">               5 </parameter>
    <parameter name="warmupSteps">            5 </parameter>
    <parameter name="blocks">                70 </parameter>
    <parameter name="timestep">             1.0 </parameter>
    <parameter name="usedrift">              no </parameter>
    <estimator name="LocalEnergy" hdf5="no"/>
    ...
    <!-- Specify the correlated sampling options and define the cost function -->
    <parameter name="minwalkers">            0.3 </parameter>
         <cost name="energy">               0.95 </cost>
         <cost name="unreweightedvariance"> 0.00 </cost>
         <cost name="reweightedvariance">   0.05 </cost>
    ...
    <!-- Specify the optimizer options -->
    <parameter name="MinMethod">    OneShiftOnly </parameter>
    ...
  </qmc>
 </loop>
\end{lstlisting}
\begin{itemize}
\item loop is helpful to execute identical optimization blocks repeatedly.
\item The first part is highly identical to a regular VMC block.
\item The second part is to specify the correlated sampling options and define the cost function.
\item The last part is used to specify the options of different optimizers. They can be very distinct from one optimizer to another.
\end{itemize}

\subsection{VMC run for the optimization}
The VMC calculation for the wavefunction optimization has a strict requirement 
that \texttt{samples} or \texttt{samplesperthread} must be specified because of the optimizer needs for the stored samples.
The input parameters of this part are identical to the VMC method.

Recommendations:
\begin{itemize}
\item Run the inclusive VMC calculation correctly and efficiently, because the this part takes significant amount of time in optimization.
For example, make sure the derived steps per block is 1 and use larger substeps to control the correlation between samples.
\item A reasonable starting wavefunction is necessary. A lot of optimization fails because of a bad wavefunction starting point.
The sign of a bad initial wavefunction includes but not limited to very long equilibration time, low acceptance ratio and huge variance.
The first thing to do after a failed optimization is to check the information provided by the VMC calculation via *.scalar.dat files.
\end{itemize}

\subsection{Correlated sampling and Cost function}
After generating the samples with VMC, the derivatives of the wavefunction with respect to the parameters are computed for proposing a new set of parameters by optimizers.
And later, a correlated sampling calculation is performed to quickly evaluate values of the cost function on the old set of parameters and the new set for the further decisions.
The input parameters are listed in the following table.
\begin{table}[h]
\begin{center}
\begin{tabularx}{\textwidth}{l l l l l l }
\hline
\multicolumn{6}{l}{\texttt{linear} method} \\
\hline
\multicolumn{2}{l}{parameters}  & \multicolumn{4}{l}{}\\
   &   \bfseries name     & \bfseries datatype & \bfseries values & \bfseries default   & \bfseries description \\
   &   \texttt{nonlocalpp} &  text     & yes, no & no  & include non-local PP energy in the cost function\\
%   &   \texttt{GEVMethod} &  text     & mixed, H2 & mixed  & methods of generalized eigenvalue problem\\
%   &   \texttt{beta} &  real     & any value & 0.0  & a parameter for GEVMethod\\
%   &   \texttt{use\_nonlocalpp\_deriv} &  text     & yes, no & no  & include the derivatives of non-local PP\\
   &   \texttt{minwalkers} &  real     & 0--1   & 0.3 & lower bound of the effective weight\\
   &   \texttt{maxWeight} &  real     & $>1$   & 1e6 & Maximum weight allowed in reweighting\\
  \hline
\end{tabularx}
\end{center}
\end{table}

Additional information:
\begin{itemize}
\item \texttt{maxWeight}. The default should be good.
\item \texttt{nonlocalpp}. Non-local PP contribution to the local energy depends on the wavefunction.
When a new set of parameter is proposed, this contribution needs to be updated if the cost function consists of local energy.
Fortunately, non-local contribution is chosen small when making a PP for small locality error.
We can ignore its change and avoid the expensive computational cost.
GPU code has a implementation issue that large amount of memory is consumed with this option.
\item \texttt{minwalkers}. A CRITICAL parameter. When the ratio of effective samples to actual number of samples in a reweighting step goes lower than \texttt{minwalkers},
the proposed set of parameters is invalid. % The last set of acceptable parameters is kept.
\end{itemize}

The cost function consists of three components: energy, unreweighted variance and reweighted variance.
\begin{lstlisting}
     <cost name="energy">                   0.95 </cost>
     <cost name="unreweightedvariance">     0.00 </cost>
     <cost name="reweightedvariance">       0.05 </cost>
\end{lstlisting}

\subsection{Optimizers}
QMCPACK implements a few optimizers having different preference aiming for different priorities.
They can be switched among  `OneShiftOnly' (default), `adaptive' and `quartic' (old) by the following line in the optimization block.
\begin{lstlisting}
<parameter name="MinMethod"> THE METHOD YOU LIKE </parameter>
\end{lstlisting}

\subsubsection{OneShiftOnly}
The OneShiftOnly optimizer targets a fast optimization by moving parameters more aggressively. It works with OpenMP and GPU and can be considered for large systems.
This method relies on the effective weight of correlated sampling rather than the cost function value to justify a new set of parameters.
If effective weight is larger than \texttt{minwalkers}, the new set is taken no matter the cost function value decreases or not.
If a proposed set is rejected, the standard output prints the measured ratio of effective samples to the total number of samples
and adjustment on \texttt{minwalkers} can be made if needed.

\begin{table}[h]
\begin{center}
\begin{tabularx}{\textwidth}{l l l l l l }
\hline
\multicolumn{6}{l}{\texttt{linear} method} \\
\hline
\multicolumn{2}{l}{parameters}  & \multicolumn{4}{l}{}\\
   &   \bfseries name     & \bfseries datatype & \bfseries values & \bfseries default   & \bfseries description \\
   &   \texttt{shift\_i} &  real     & $>0$ & 0.01 & Direct stabilizer added to the Hamiltonian matrix\\
   &   \texttt{shift\_s} &  real     & $>0$ & 1.00 & Initial stabilizer based on the overlap matrix\\
  \hline
\end{tabularx}
\end{center}
\end{table}

Additional information:
\begin{itemize}
\item \texttt{shift\_i}. This is the direct term added to the diagonal of the Hamiltonian matrix.
                         More stable but slower optimization with a large value.
\item \texttt{shift\_s}. This is the initial value of the stabilizer based on the overlap matrix added to the Hamiltonian matrix.
                         More stable but slower optimization with a large value. The used value is auto-adjusted by the optimizer.
\end{itemize}


Recommendations:
\begin{itemize}
  \item Default \texttt{shift\_i}, \texttt{shift\_s} should be fine.
  \item For hard cases, increasing \texttt{shift\_i} (factor of 5 or 10) can significantly stabilize the optimization by reducing the pace towards the optimal parameter set.
  \item If the VMC energy of the last optimization iterations grows significantly, increase \texttt{minwalkers} closer to 1 and make the optimization stable.
  \item If the first iterations of optimization are rejected on a reasonable initial wavefunction, 
        lower the \texttt{minwalkers} value based on the measured value printed in the standard output to accept the move.
\end{itemize}

It is recommended to use this optimizer in two sections with a very small \texttt{minwalkers} in the first and a large value in the second like the following.
In the very beginning, parameters are far away form optimal values and large changes are proposed by the optimizer.
Having a small \texttt{minwalkers} allows accepting these changes much easier.
When the energy gradually converges, we can have a large \texttt{minwalkers} to avoid risky parameter sets.
\begin{lstlisting}
 <loop max="6">
  <qmc method="linear" move="pbyp" gpu="yes">
    <!-- Specify the VMC options -->
    <parameter name="walkers">                1 </parameter>
    <parameter name="samples">            10000 </parameter>
    <parameter name="stepsbetweensamples">    1 </parameter>
    <parameter name="substeps">               5 </parameter>
    <parameter name="warmupSteps">            5 </parameter>
    <parameter name="blocks">                25 </parameter>
    <parameter name="timestep">             1.0 </parameter>
    <parameter name="usedrift">              no </parameter>
    <estimator name="LocalEnergy" hdf5="no"/>
    <!-- Specify the optimizer options -->
    <parameter name="MinMethod">    OneShiftOnly </parameter>
    <parameter name="minwalkers">           1e-4 </parameter>
  </qmc>
 </loop>
 <loop max="12">
  <qmc method="linear" move="pbyp" gpu="yes">
    <!-- Specify the VMC options -->
    <parameter name="walkers">                1 </parameter>
    <parameter name="samples">            20000 </parameter>
    <parameter name="stepsbetweensamples">    1 </parameter>
    <parameter name="substeps">               5 </parameter>
    <parameter name="warmupSteps">            2 </parameter>
    <parameter name="blocks">                50 </parameter>
    <parameter name="timestep">             1.0 </parameter>
    <parameter name="usedrift">              no </parameter>
    <estimator name="LocalEnergy" hdf5="no"/>
    <!-- Specify the optimizer options -->
    <parameter name="MinMethod">    OneShiftOnly </parameter>
    <parameter name="minwalkers">            0.5 </parameter>
  </qmc>
 </loop>
\end{lstlisting}

For each optimization step, you will see
\begin{lstlisting}
The new set of parameters is valid. Updating the trial wave function!
\end{lstlisting}
or
\begin{lstlisting}
The new set of parameters is not valid. Revert to the old set!
\end{lstlisting}
Occasional rejection is fine. Frequent rejection indicates potential problems and users should inspect the VMC calculation or change optimization strategy.
To track the progress of optimization, using command ``qmca -q ev *.scalar.dat'' to look at the VMC energy and variance for each optimization step.

\subsubsection{adaptive}

The default setting of the adaptive optimizer is to construct the linear method Hamiltonian and overlap matrices explicitly and add different shifts to the Hamiltonian matrix 
as ``stabilizers''.
The generalized eigenvalue problem is solved for each shift to obtain updates to the wave function parameters.
Then a correlated sampling is performed for each shift's updated wave function and the initial trial wave function
using the middle shift's updated wave function as the guiding function.
The cost function for these wave functions is compared, and the update corresponding to the best cost function is selected.
In the next iteration, the median magnitude of the stabilizers is set to that that generated the best update in the current iteration, thus adapting the magnitude of
the stabilizers automatically.

When the trial wave function contains more than ten thousand parameters, constructing and storing the linear method matrices may become a memory bottleneck. 
To avoid explicit construction of these matrices, the adaptive optimizer implements the block linear method (BLM) approach. \cite{Zhao:2017:blocked_lm}
The BLM tries to find an approximate 
solution $\vec{c}_{opt}$ to the standard LM generalized eigenvalue problem by dividing the variable space into a number of blocks
and making intelligent estimates for which directions within those blocks will be most important for constructing $\vec{c}_{opt}$.
which is then obtained by solving a smaller, more memory-efficient 
eigenproblem in the basis of these supposedly important block-wise directions. 

\begin{table}[h]
\begin{center}
\begin{tabularx}{\textwidth}{l l l l l l }
\hline
\multicolumn{6}{l}{\texttt{linear} method} \\
\hline
\multicolumn{2}{l}{parameters}  & \multicolumn{4}{l}{}\\
   &   \bfseries name     & \bfseries datatype & \bfseries values & \bfseries default   & \bfseries description \\
   %&   \texttt{stepsize} &  real     & 0--1 & 0.25  & Step size for moving parameters\\
   &   \texttt{max\_relative\_change} &  real     & $>0$ & 10.0 & Allowed change in cost function\\
   &   \texttt{max\_param\_change} &  real     & $>0$ & 0.3 & Allowed change in wave function parameter\\
   &   \texttt{shift\_i} &  real     & $>0$ & 0.01 & Initial diagonal      stabilizer added to the Hamiltonian matrix\\
   &   \texttt{shift\_s} &  real     & $>0$ & 1.00 & Initial overlap-based stabilizer added to the Hamiltonian matrix\\
   &   \texttt{chase\_lowest} &  text   & yes, no & yes & Chase the lowest eigenvector in iterative solver\\
   &   \texttt{chase\_closest} &  text   & yes, no & no & Chase the eigenvector closest to initial guess\\
   &   \texttt{block\_lm} &  text   & yes, no & no & Use block linear method\\
   &   \texttt{nblocks} &  integer   & $>0$ &  & \# of blocks in BLM\\
   &   \texttt{nolds} &  integer   & $>0$ &  & \# of old update vectors used in BLM\\
   &   \texttt{nkept} &  integer   & $>0$ &  & \# of eigenvectors to keep per block in BLM\\
  \hline
\end{tabularx}
\end{center}
\end{table}

Additional information:
\begin{itemize}
  \item \texttt{shift\_i}.  This is the initial coefficient used to scale the diagonal stabilizer.
                            More stable but slower optimization is expected with a large value.
                            The adaptive method will automatically adjust this value after each linear method iteration.
  \item \texttt{shift\_s}.  This is the initial coefficient used to scale the overlap-based stabilizer.
                            More stable but slower optimization is expected with a large value.
                            The adaptive method will automatically adjust this value after each linear method iteration.
  \item \texttt{nblocks}.   This is the number of blocks used in block LM. The amount of memory required to store LM matrices decreases
                            with increased number of blocks. But the error introduced by BLM would increase with number of blocks.  
  \item \texttt{nolds}.     In BLM, the inter-block correlation is accounted for by including a small number of wave function update vectors
                            outside the block. Larger \texttt{nolds} would include more inter-block correlation and more accurate results, but 
                            also higher memory requirements. 
  \item \texttt{nkept}.     This is the number of update directions retained from each block in the BLM. If all directions are retained in each block, 
                            then the BLM becomes equivalent to the standard LM.  Retaining 5 or fewer directions per block is often sufficient.
\end{itemize}

Recommendations:
\begin{itemize}
  \item Default shift\_i, shift\_s should be fine. 
  \item When there are fewer than about 5,000 variables being optimized, the traditional LM is preferred as it has a lower overhead than the BLM when the number of variables is small.
  \item Initial experience with the BLM suggests that a few hundred blocks and a handful of \texttt{nolds} and \texttt{nkept}
        often provide a good balance between memory use and accuracy.  In general, using fewer blocks should be more accurate but will require more memory.
\end{itemize}

\begin{lstlisting}
 <loop max="15">
  <qmc method="linear" move="pbyp">
    <!-- Specify the VMC options -->
    <parameter name="walkers">                1 </parameter>
    <parameter name="samples">            20000 </parameter>
    <parameter name="stepsbetweensamples">    1 </parameter>
    <parameter name="substeps">               5 </parameter>
    <parameter name="warmupSteps">            5 </parameter>
    <parameter name="blocks">                50 </parameter>
    <parameter name="timestep">             1.0 </parameter>
    <parameter name="usedrift">              no </parameter>
    <estimator name="LocalEnergy" hdf5="no"/>
    <!-- Specify the correlated sampling options and define the cost function -->
         <cost name="energy">               1.00 </cost>
         <cost name="unreweightedvariance"> 0.00 </cost>
         <cost name="reweightedvariance">   0.00 </cost>
    <!-- Specify the optimizer options -->
    <parameter name="MinMethod">adaptive</parameter>
    <parameter name="max_relative_cost_change">10.0</parameter>
    <parameter name="shift_i"> 1.00 </parameter>
    <parameter name="shift_s"> 1.00 </parameter>
    <parameter name="max_param_change"> 0.3 </parameter>
    <parameter name="chase_lowest"> yes </parameter>
    <parameter name="chase_closest"> yes </parameter>
    <parameter name="block_lm"> no </parameter>
    <!-- Specify the BLM specific options if needed
      <parameter name="nblocks"> 100 </parameter>
      <parameter name="nolds"> 5 </parameter>
      <parameter name="nkept"> 3 </parameter>
    -->
  </qmc>
 </loop>
\end{lstlisting}
%To activate this optimizer, add ``-D BUILD\_LMYENGINE\_INTERFACE=1'' in the CMake command line.

The adaptive optimizer is also able to optimize individual excited states directly. \cite{Zhao:2016:dir_tar}
In this case, it tries to minimize the following function: 
\begin{equation*}
\Omega[\Psi]=\frac{\left<\Psi|\omega-H|\Psi\right>}{\left<\Psi|{\left(\omega-H\right)}^2|\Psi\right>}
\end{equation*}
The global minimum of this function corresponds to the state whose energy lies immediately above the shift parameter $\omega$ in the energy spectrum.
For example, if $\omega$ were placed in between the ground state energy and the first excited state energy and the wave function ansatz was capable of a good
description for the first excited state, then the wave function would be optimized for the first excited state.
It is important to note that, if the ansatz is not capable of a good description of the excited state in question, the optimization may converge to a different
state, as is known to occur in some circumstances for traditional ground state optimizations.
Note also that the ground state can be targeted by this method by choosing $\omega$ to be below the ground state energy, although we should stress that this
is not the same thing as a traditional ground state optimization and will in general give a slightly different wave function.
Excited state targeting requires two additional parameters, as shown in this table.

\begin{table}[h]
\begin{center}
\begin{tabularx}{\textwidth}{l l l l l l }
\hline
\multicolumn{6}{l}{Excited State Targeting} \\
\hline
\multicolumn{2}{l}{parameters}  & \multicolumn{4}{l}{}\\
   &   \bfseries name     & \bfseries datatype & \bfseries values & \bfseries default   & \bfseries description \\
   %&   \texttt{stepsize} &  real     & 0--1 & 0.25  & Step size for moving parameters\\
   &   \texttt{targetExcited} &  text   & yes, no      & no   & Whether to use the excited state targeting optimization\\
   &   \texttt{omega}         &  real   & real numbers & none & Energy shift used to target different excited states\\
  \hline
\end{tabularx}
\end{center}
\end{table}

Excited state recommendations:
\begin{itemize}
  \item Due to the finite variance in any approximate wave function, it is recommended to set $\omega=\omega_0-\sigma$, where $\omega_0$ is placed just
        below the energy of the targeted state and $\sigma^2$ is the energy variance.
  \item In order to obtain an unbiased excitation energy, one should optimize the ground state with the excited state variational principle as well by setting
        \texttt{omega} below the ground state energy.  Note that using the ground state variational principle for the ground state and the excited state variational
        principle for the excited state creates a bias in favor of the ground state. 
\end{itemize}

\subsubsection{quartic}
\textbf{This is an older optimizer method retained for compatibility. We recommend starting with the newest OneShiftOnly or adaptive optimizers.}
The quartic optimizer fits a quartic polynomial to 7 values of the cost function obtained using reweighting along chosen direction and determines the optimal move.
This optimizer is very robust but a bit conservative to accept new steps especially when large parameters changes are proposed.
\begin{table}[h]
\begin{center}
\begin{tabularx}{\textwidth}{l l l l l l }
\hline
\multicolumn{6}{l}{\texttt{linear} method} \\
\hline
\multicolumn{2}{l}{parameters}  & \multicolumn{4}{l}{}\\
   &   \bfseries name     & \bfseries datatype & \bfseries values & \bfseries default   & \bfseries description \\
   %&   \texttt{stepsize} &  real     & 0--1 & 0.25  & Step size for moving parameters\\
   &   \texttt{bigchange} &  real     & $>0$ & 50.0  & Largest parameter change allowed\\
   &   \texttt{alloweddifference} &  real     & $>0$ & 1e-4 & Allowed increased in energy\\
   &   \texttt{exp0} &  real     & any value & -16.0 & Initial value for stabilizer\\
   &   \texttt{stabilizerscale} &  real     & $>0$ & 2.0 & Increase in value of exp0 between iterations\\
   &   \texttt{nstabilizers} &  integer     & $>0$ & 3 & Number of stabilizers to try\\
   &   \texttt{max\_its} &  integer   & $>0$ & 1 & Number of inner loops with same samples\\
  \hline
\end{tabularx}
\end{center}
\end{table}

Additional information:
\begin{itemize}
\item \texttt{exp0}. It is the initial value for stabilizer (shift to diagonal of H). The actual value of stabilizer is $10^{\textrm{exp0}}$.
\end{itemize}

Recommendations:
\begin{itemize}
  \item{For hard cases (e.g. simultaneous optimization of long MSD and 3-Body J), set exp0
to 0 and do a single inner iteration (max its=1) per sample of configurations.}
\end{itemize}

\begin{lstlisting}
    <!-- Specify the optimizer options -->
    <parameter name="MinMethod">quartic</parameter>
    <parameter name="exp0">-6</parameter>
    <parameter name="alloweddifference"> 1.0e-4 </parameter>
    <parameter name="nstabilizers"> 1 </parameter>
    <parameter name="bigchange">15.0</parameter>
\end{lstlisting}

\subsection{General recommendations}
Here are a few recommendations to make wavefunction optimization easier.
\begin{itemize}
\item All electron wavefunctions are typically more difficult to optimize than pseudopotential ones due to the importance of the wavefunction near the nucleus.
\item Two body Jastrow contributes the largest portion of correlation energy from bare Slater determinants. For this reason, the recommended order for optimizing wavefunction components is two-body, one-body, three-body Jastrow factors and MSD coefficients.
\item For two-body spline Jastrows, always start from a reasonable one. The lack of physically-motivated constraints in the functional form at large distances can cause slow convergence if starting from zero. 
\item One-body spline Jastrow from old calculations can be a good starting point.
\item Three-body polynomial Jastrow can start from zero. It is beneficial to first optimize one-body and two-body Jastrow factors without adding three-body terms in the calculation and then add the three-body Jastrow and optimize all the three components together.
\end{itemize}


