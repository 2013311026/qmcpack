\chapter{Obtaining, installing and validating QMCPACK}
\label{chap:obtaininginstalling}

This chapter describes how to obtain, build and validate QMCPACK. This process is designed to be as simple as
possible and should be no harder than building a modern plane-wave density
functional theory code such as Quantum Espresso, QBox, or
VASP. Parallel builds enable a complete
compilation in under 2 minutes on a fast multicore sysyem, If you
are unfamiliar with building codes we suggest working with your system
administrator to install QMCPACK.

\section{Installation steps}
To install QMCPACK, follow the steps listed below. Full details of
each step are given in the referenced sections.
\begin{enumerate}
\item Download the source code, Sections \ref{sec:obrelease} or \ref{sec:obdevelopment}.
\item Verify that you have the required compilers, libraries and tools
  installed, Section \ref{sec:prerequisites}.
\item Run the cmake configure step and build with make, Section
  \ref{sec:cmake} and \ref{sec:cmakequick}. Some examples for common
  systems are given in Secton \ref{sec:installexamples}.
\item Run the tests to verify QMCPACK, Section \ref{sec:testing}.
\item Build the ppconvert utility in QMCPACK, Section \ref{sec:buildppconvert}.
\item Download and patch Quantum Espresso. This patch adds the
  pw2qmcpack utility, Section \ref{sec:buildqe}.
\end{enumerate}

Hints for high performance are in Section \ref{sec:buildperformance}. Troubleshooting suggestions are in Section \ref{sec:troubleshoot}.

Note that there are two different QMCPACK executables that can be
produced: the general one, which is the default, and the ``complex''
version which support periodic calculations at arbitrary twist angles and
k-points. This second version is enabled via a cmake configuration
parameter, see Section \ref{sec:cmakeoptions}. The general version
only supports wavefunctions that can be made real. If you run a
calculation that needs the complex version, QMCPACK will stop and inform you.

\section{Obtaining the latest release version}
\label{sec:obrelease} 
Major releases of QMCPACK are distributed from
\url{http://www.qmcpack.org}. These releases undergo the most testing. Unless there are
specific reasons we encourage all production calculations to use the
latest release versions.

Releases are usually compressed tar files indicating the version
number, date, and often the source code revision control number
corresponding to the release.

\begin{itemize}
\item Download the latest QMCPACK distribution from \url{http://www.qmcpack.org}.
\item Untar the archive, e.g., \texttt{tar xvf qmcpack\_v1.3.tar.gz}
\end{itemize}

\section{Obtaining the latest development version}
\label{sec:obdevelopment}
The most recent development version of QMCPACK can be obtained anonymously via 
\begin{verbatim}
svn checkout https://svn.qmcpack.org/svn/trunk 
\end{verbatim}
Once checked-out,
updates can be made via the standard \texttt{svn update}.

The subversion repository contains the day-to-day development source
with the latest updates, bugfixes etc. This may be useful
for updates to the build system to support new machines, for support
of the latest versions of Quantum Espresso, or for updates to the
documentation.  Note that the development version may not be fully
consistent with the online documentation.  We attempt to keep
the development version fully working. However, please be sure to run the tests and
compare with previous release versions before using for any serious
calculations. We try to keep bugs out, but occasionally they crawl
in! Reports of any breakages are appreciated.

\section{Prerequisites}
\label{sec:prerequisites}
The following are required to build QMCPACK. For workstations, these are available via the standard
package manager. On shared supercomputers this software is usually
installed by default and is often
access via a modules environment - check your system
documentation.

\textbf{Use of the latest versions of all compilers and libraries is
strongly encouraged}, but not absolutely essential. Generally newer versions are faster - see
Section \ref{sec:buildperformance} for performance suggestions.

\begin{itemize}
\item C/C++ compilers such as GCC, Intel, IBM XLC. CLANG-based compilers
  are not yet supported by the build system, but the source code is ready.
\item MPI libary such at OpenMPI \url{http://open-mpi.org}
\item BLAS/LAPACK, numerical and linear algebra libraries. Use
  platform-optimized libraries where available, such as Intel MKL.
  ATLAS or other optimized open-source libraries may also be used
  \url{http://math-atlas.sourceforge.net}
\item CMake, build utility, \url{http://www.cmake.org}
\item Libxml2, XML parser, \url{http://xmlsoft.org}
\item HDF5, portable I/O library, \url{http://www.hdfgroup.org/HDF5/}
\item BOOST, peer-reviewed portable C++ source libraries, \url{http://www.boost.org}
\item FFTW, FFT library, \url{http://www.fftw.org/}
\end{itemize}

To build the GPU accelerated version of QMCPACK an installation of
NVIDIA CUDA development tools is required. Ensure that this is
compatible with the C and C++ compiler versions you plan to
use. Supported versions are included in the NVIDIA release notes.

Many of the utilities provided with QMCPACK use python (v2). The numpy
and matplotlib libraries are required for full functionality.

Note that the standalone einspline library used by previous versions of QMCPACK
is no longer required. A more optimized version is included
inside. The standalone version should \emph{not} be on any standard
search paths because conflicts between the old and new include files
can result.

\section{Building with CMake}
\label{sec:cmake}
The build system for QMCPACK is based on CMake.  It will autoconfigure
based on the detected compilers and libraries. The most recent
version of CMake has the best detection for the greatest variety of
systems - at the time of writing this means CMake 3.4.3. The much
older CMake 2.8 is known to work, but might not work optimally on your system.

Previously QMCPACK made extensive use of toolchains, but the build system
has since been updated to eliminate the use of toolchain files for
most cases.  The build system is verified to work with GNU, Intel, and IBM XLC
compilers.  Specific compile options can be specified either through
specific environmental or CMake variables.  When the libraries are
installed in standard locations, e.g., /usr, /usr/local, there is no
need to set environmental or cmake variables for the packages.

\subsection{Quick build instructions (try first)}
\label{sec:cmakequick}

If you are feeling lucky and are on a standard UNIX-like system such
as a Linux workstation, the following might quickly give a
working QMCPACK:

The safest quick build option is to specify the C and C++ compilers
through their MPI wrappers. Here we use Intel MPI and Intel
compilers. Move to the build directory, run cmake and make
\begin{verbatim}
cd build
cmake -DCMAKE_C_COMPILER=mpiicc -DCMAKE_CXX_COMPILER=mpiicpc ..
make -j 8
\end{verbatim}
You can increase the \"8\" to the number of cores on your system for
faster builds. Substitute mpicc and mpicxx or other wrapped compiler names to suit
  your system. e.g. With OpenMPI use
\begin{verbatim}
cd build
cmake -DCMAKE_C_COMPILER=mpicc -DCMAKE_CXX_COMPILER=mpicxx ..
make -j 8
\end{verbatim}

If you are feeling particularly lucky, you can skip the compiler specification:
\begin{verbatim}
cd build
cmake ..
make -j 8
\end{verbatim}

The complexities of modern computer hardware and software systems are
such that you should check that the autoconfiguration system has made
good choices and picked optimized libraries and compiler settings
before doing significant production. i.e. Check the details below. We
give examples for a number of common systems in Section \ref{sec:installexamples}.

\subsection{Environment variables}
A number of enviornmental variables affect the build.  In particular
they can control the default paths for libraries, the default
compilers, etc.  The list of enviornmental variables is given below:
\begin{verbatim}
CXX              C++ compiler
CC               C Compiler
MKL_HOME         Path for MKL
LIBXML2_HOME     Path for libxml2
HDF5_ROOT        Path for HDF5
BOOST_ROOT       Path for Boost
FFTW_HOME        Path for FFTW
\end{verbatim}

\subsection{Configuration options}
\label{sec:cmakeoptions}
In addition to reading the enviornmental variables, CMake provides a
number of optional variables that can be set to control the build and
configure steps.  When passed to CMake, these variables will take
precident over the enviornmental and default variables.  To set them
add -D FLAG=VALUE to the configure line between the cmake command and
the path to the source directory.

\begin{itemize}
\item  Key QMCPACK build options
\begin{verbatim}
QMC_CUDA            Enable CUDA and GPU acceleration (1:yes, 0:no)
QMC_COMPLEX         Build the complex (general twist/k-point) version (1:yes, 0:no)
\end{verbatim}
  \item General build options
\begin{verbatim}
CMAKE_BUILD_TYPE    A variable which controls the type of build (defaults to Release).  Possible values are:
                   None (Do not set debug/optmize flags, use CMAKE_C_FLAGS or CMAKE_CXX_FLAGS)
                   Debug (create a debug build)
                   Release (create a release/optimized build)
                   RelWithDebInfo (create a release/optimized build with debug info)
                   MinSizeRel (create an executable optimized for size)
CMAKE_C_COMPILER    Set the C compiler
CMAKE_CXX_COMPILER  Set the C++ compiler
CMAKE_C_FLAGS       Set the C flags.  Note: to prevent default debug/release flags from being used, set the CMAKE_BUILD_TYPE=None
                   Also supported: CMAKE_C_FLAGS_DEBUG, CMAKE_C_FLAGS_RELEASE, CMAKE_C_FLAGS_RELWITHDEBINFO
CMAKE_CXX_FLAGS     Set the C++ flags.  Note: to prevent default debug/release flags from being used, set the CMAKE_BUILD_TYPE=None
                   Also supported: CMAKE_CXX_FLAGS_DEBUG, CMAKE_CXX_FLAGS_RELEASE, CMAKE_CXX_FLAGS_RELWITHDEBINFO
\end{verbatim}
\item Additional QMCPACK build options
\begin{verbatim}
QMC_DATA            Specify data directory for QMCPACK (currently unused, but likely to be used for performance tests)
QMC_INCLUDE         Add extra include paths
QMC_EXTRA_LIBS      Add extra link libraries
QMC_BUILD_STATIC    Add -static flags to build
\end{verbatim}
\item libxml related
\begin{verbatim}
Libxml2_INCLUDE_DIRS  Specify include directories for libxml2
Libxml2_LIBRARY_DIRS  Specify library directories for libxml2
\end{verbatim}
 \item FFTW related
\begin{verbatim}
FFTW_INCLUDE_DIRS   Specify include directories for FFTW
FFTW_LIBRARY_DIRS   Specify library directories for FFTW
\end{verbatim}
\end{itemize}

\subsection{Configure and build using cmake and make}
To configure and build QMPACK, move to build directory, run cmake and make
\begin{verbatim}
cd build
cmake ..
make -j 8
\end{verbatim}

As you will have gathered, cmake encourages ``out of source'' builds,
where all the files for a specific build configuration reside in their
own directory separate from the source files. This allows multiple
builds to be created from the same source files which is very useful
where the filesystem is shared between different systems. You can also
build versions with different settings (e.g. QMC\_COMPLEX) and
different compiler settings. The build directory does not have to be
called build - use something descriptive such as build\_machinename or
build\_complex. The ``..'' in the cmake line refers to the directory
containing CMakeLists.txt. Update the ``..'' for other build 
directory locations.

\subsection{Example configure and build}
\begin{itemize}
\item Set the environments (the examples below assume bash, Intel compilers and MKL library)
\begin{verbatim}
export CXX=icpc
export CC=icc
export MKL_HOME=/usr/local/intel/mkl/10.0.3.020
export LIBXML2_HOME=/usr/local
export HDF5_ROOT=/usr/local
export BOOST_ROOT=/usr/local/boost
export FFTW_HOME=/usr/local/fftw
\end{verbatim}

\item Move to build directory, run cmake and make
\begin{verbatim}
cd build
cmake -D CMAKE_BUILD_TYPE=Release ..
make -j 8
\end{verbatim}
\end{itemize}

\subsection{Build scripts}
It is recommended to create a helper script that contains the
configure line for CMake.  This is particularly useful when avoiding
enviornmental variables, packages are installed in custom locations,
or if the configure line is long or complex.  In this case it is also
recommended to add "rm -rf CMake*" before the configure line to remove
existing CMake configure files to ensure a fresh configure each time
that the script is called. Deleting all the files in the build
directory is also acceptable. If you do so we recommend to add some sanity
checks in case the script is run from the wrong directory, e.g.,
checking for the existence of some QMCPACK files.

Some build script examples for different systems are given in the
config directory. For example, on Cray systems these scripts might
load the appropriate modules to set the appropriate programming
environment, specific library versions etc.

An example script build.sh is given below. It is overly complex for
the sake of example:

\begin{verbatim}
export CXX=mpic++
export CC=mpicc
export ACML_HOME=/opt/acml-5.3.1/gfortran64
export HDF5_ROOT=/opt/hdf5
export BOOST_ROOT=/opt/boost

rm -rf CMake*

cmake                                                \
  -D CMAKE_BUILD_TYPE=Debug                         \
  -D Libxml2_INCLUDE_DIRS=/usr/include/libxml2      \
  -D Libxml2_LIBRARY_DIRS=/usr/lib/x86_64-linux-gnu \
  -D FFTW_INCLUDE_DIRS=/usr/include                 \
  -D FFTW_LIBRARY_DIRS=/usr/lib/x86_64-linux-gnu    \
  -D QMC_EXTRA_LIBS="-ldl ${ACML_HOME}/lib/libacml.a -lgfortran" \
  -D QMC_DATA=/projects/QMCPACK/qmc-data            \
  ..
\end{verbatim}

\section{Installation instructions for common workstations and
  supercomputers}
\label{sec:installexamples}

This section describes how to build QMCPACK on various common systems
including multiple Linux distributions, Apple OS X, and various
supercomputers. The examples should serve as good starting point for
building QMCPACK on similar machines. For example, the software
environment on modern Crays is very consistent. Note that updates to
operating systems and system software may require small modifications
to these recipes. See Section \label{sec:buildperformance} for key
points to check to obtain highest performance and
Section \label{sec:troubleshoot} for troubleshooting hints.

\subsection{Installing on Ubuntu Linux or other apt-get based distributions}

The following is designed to obtain a working QMCPACK build on e.g. a
student laptop, starting from a basic Linux installation with none of
the developer tools installed. Fortunately, all the required packages
are available in the default repositories making for a quick
installation. Note that for convenience we use a generic BLAS. For
production a platform optimized BLAS should be used.

\begin{verbatim}
apt-get subversion cmake g++ openmpi-bin libopenmpi-dev libboost-dev
apt-get libatlas-base-dev liblapack-dev libhdf5-dev libxml2-dev fftw3-dev
export CXX=mpiCC
cd build
cmake ..
make -j 8
ls -l bin/qmcapp
\end{verbatim}

For qmca and other tools to function, we install some python libraries:
\begin{verbatim}
sudo apt-get install python-numpy python-matplotlib
\end{verbatim}

\subsection{Installing on CentOS Linux or other yum based distributions}

The following is designed to obtain a working QMCPACK build on e.g. a
student laptop, starting from a basic Linux installation with none of
the developer tools installed. CentOS 7 (Red Hat compatible) is using
gcc 4.8.2. The installation is only complicated by the need to install
another repository to obtain HDF5 packages which are not available by
default. Note that for convenience we use a generic BLAS. For
production a platform optimized BLAS should be used.

\begin{verbatim}
sudo yum install make cmake gcc gcc-c++ subversion openmpi  openmpi-devel fftw fftw-devel boost boost-devel libxml2 libxml2-devel
sudo yum install blas-devel lapack-devel atlas-devel
module load mpi 

\end{verbatim}

To setup repoforge as a source for the HDF5 package, go to
\url{http://repoforge.org/use} . Install the appropriate up to date
release package for your OS. By default the CentOS Firefox will offer
to run the installer. The CentOS 6.5 settings were usable for HDF5 on
CentOS 7 in July 2014, but use CentOS 7 versions when they become
available.

\begin{verbatim}
sudo yum install hdf5 hdf5-devel 
\end{verbatim}

To build QMCPACK
\begin{verbatim}
module load mpi/openmpi-x86_64
which mpirun
# Sanity check; should print something like   /usr/lib64/openmpi/bin/mpirun
export CXX=mpiCC
cd build
cmake ..
make -j 8
ls -l bin/qmcapp
\end{verbatim}

\subsection{Installing on Mac OS X using Macports}
These instructions assume a fresh installation of macports
and for consistency with current Linux distributions, use the gcc 4.8.2
compiler. More recent versions are fine, but it is vital to ensure matching compilers/options for all
packages and to force use of what is installed in /opt/local. As with
the Linux examples above, this build is very good if not optimal, and
is easily good enough to learn QMCPACK or experiment on a travel laptop.

Note that we utilize the Apple provided Accelerate framework for optimized BLAS.

Follow the Macports install instructions \url{https://www.macports.org/}

\begin{itemize}
\item Install Xcode and the Xcode Command Line Tools
\item Agree to Xcode license in Terminal: sudo xcodebuild -license
\item Install MacPorts for your version of OS X
\end{itemize}


Install the required tools:

\begin{verbatim} 
sudo port install gcc48
sudo port select gcc mp-gcc48  # Set default

sudo port install openmpi-devel-gcc48
sudo port select —set mpi openmpi-devel-gcc48-fortran  # Set default

# Sanity check
mpiCXX -v 
#should return … “gcc version 4.8.2 (MacPorts gcc48 4.8.2_2)” or similar.

sudo port install fftw-3 +gcc48
sudo port install cmake    # already cmake 3 as of 2014/7/29

sudo port install boost +gcc48
sudo port install libxml2
sudo port install hdf5-18 +gcc48

sudo port select —set python python27
sudo port install py27-matplotlib  # For qmca
\end{verbatim}

QMCPACK build:
\begin{verbatim}
export CXX=mpiCXX
export CC=/opt/local/bin/gcc
export LIBXML2_HOME=/opt/local/
export HDF5_HOME=/opt/local
export BOOST_HOME=/opt/local
export FFTW_HOME=/opt/local
cd build
cmake ..
make -j 6 # Adjust for available core count
ls -l bin/qmcapp 
\end{verbatim}

\subsection{Installing on ANL ALCF Mira IBM BGQ}

\subsection{Installing on ORNL OLCF Titan Cray XK7 (NVIDIA GPU
  accelerated)}
\label{sec:titanbuildgpu}
Titan is a GPU accelerated supercomputer at Oak Ridge National
Laboratory's  Oak Ridge Leadship Computing Facility  (ORNL OLCF). Each
compute node has a 16 core AMD 2.2GHz Opteron 6274 (Interlagos) and an
NVIDIA Kepler accelerator. The standard Cray software environment is
available, with libraries accessed via modules. The only extra
settings required to build the GPU version are the cudatoolkit module
and specifiying -DQMC\_CUDA=1 on the cmake configure line.

Note that on Crays the compiler wrappers ``CC'' and ``cc'' are 
used. The build system checks for these and does not (should not) use
the compilers directly.

\begin{verbatim}
module swap PrgEnv-pgi PrgEnv-gnu # Use gnu compilers
module load cudatoolkit           # CUDA for GPU build
module load cray-hdf5
module load cmake
module load fftw
export FFTW_HOME=$FFTW_DIR/..
module load boost
mkdir build_titan_gpu
cd build_titan_gpu
cmake -DQMC_CUDA=1 ..             # Must enable CUDA capabilities
ls -l bin/qmcapp 
\end{verbatim}

\subsection{Installing on ORNL OLCF Titan Cray XK7 (CPU version)}
As noted in Section\ref{sec:titanbuildgpu} for the GPU, building on
Crays requires only loading the appropriate library modules.

\begin{verbatim}
module swap PrgEnv-pgi PrgEnv-gnu # Use gnu compilers
module unload cudatoolkit         # No CUDA for CPU build
module load cray-hdf5
module load cmake
module load fftw
export FFTW_HOME=$FFTW_DIR/..
module load boost
mkdir build_titan_cpu
cd build_titan_cpu
cmake ..
ls -l bin/qmcapp 
\end{verbatim}

\subsection{Installing on ORNL OLCF Eos Cray XC30}
Eos is Cray XC30 with 16 core Intel Xeon E5-2670 processors connected
by the Aries interconnect. The build process is identical to Titan,
except that we use the default Intel programming environment. This is
usually preferred to GNU.
\begin{verbatim}
module load cray-hdf5
module load cmake
module load fftw
export FFTW_HOME=$FFTW_DIR/..
module load boost
mkdir build_eos
cd build_eos
cmake ..
ls -l bin/qmcapp 
\end{verbatim}

\subsection{Installing on NERSC Edison Cray XC30}

Edison is a Cray XC30 with dual 12-core Intel "Ivy Bridge" nodes 
installed at NERSC. The build settings are identical to eos. 

\begin{verbatim}
module load cray-hdf5
module load cmake
module load fftw
export FFTW_HOME=$FFTW_DIR/..
module load boost
mkdir build_edison
cd build_edison
cmake ..
ls -l bin/qmcapp 
\end{verbatim}
When the above was tested on 1 February 2016, the following module and
software versions were present:
\begin{verbatim}
qmcpack@edison04:trunk> module list
Currently Loaded Modulefiles:
  1) modules/3.2.10.3                       9) udreg/2.3.2-1.0502.9889.2.20.ari      17) rca/1.0.0-2.0502.57212.2.56.ari       25) darshan/2.3.0
  2) nsg/1.2.0                             10) ugni/6.0-1.0502.10245.9.9.ari         18) atp/1.8.3                             26) subversion/1.7.9
  3) eswrap/1.1.0-1.020200.1130.0          11) pmi/5.0.10-1.0000.11050.0.0.ari       19) PrgEnv-intel/5.2.56                   27) cray-hdf5/1.8.14
  4) switch/1.0-1.0502.57058.1.58.ari      12) dmapp/7.0.1-1.0502.10246.8.47.ari     20) craype-ivybridge                      28) cmake/2.8.11.2
  5) craype-network-aries                  13) gni-headers/4.0-1.0502.10317.9.2.ari  21) cray-shmem/7.3.0                      29) fftw/3.3.4.6
  6) craype/2.5.0                          14) xpmem/0.1-2.0502.57015.1.15.ari       22) cray-mpich/7.3.0                      30) boost/1.54
  7) intel/15.0.1.133                      15) dvs/2.5_0.9.0-1.0502.1958.2.55.ari    23) slurm/edison
  8) cray-libsci/13.3.0                    16) alps/5.2.3-2.0502.9295.14.14.ari      24) altd/2.0
\end{verbatim}

\subsection{Installing on NERSC Cori (Phase 1) Cray XC40}
Cori is a Cray XC40 with 16-core Intel "Haswell" nodes 
installed at NERSC. The build settings are identical to eos. 

\begin{verbatim}
module load cray-hdf5
module load cmake
module load fftw
export FFTW_HOME=$FFTW_DIR/..
module load boost
mkdir build_cori
cd build_cori
cmake ..
ls -l bin/qmcapp 
\end{verbatim}

When the above was tested on 1 February 2016, the following module and
software versions were present:

\begin{verbatim}
qmcpack@cori05:trunk> module list
Currently Loaded Modulefiles:
  1) nsg/1.2.0                             11) pmi/5.0.9-1.0000.10911.0.0.ari        21) cray-shmem/7.2.5
  2) modules/3.2.10.3                      12) dmapp/7.0.1-1.0502.11080.8.76.ari     22) cray-mpich/7.2.5
  3) eswrap/1.1.0-1.020200.1231.0          13) gni-headers/4.0-1.0502.10859.7.8.ari  23) slurm/cori
  4) switch/1.0-1.0502.60522.1.61.ari      14) xpmem/0.1-2.0502.64982.5.3.ari        24) cray-hdf5/1.8.14
  5) intel/16.0.0.109                      15) dvs/2.5_0.9.0-1.0502.2188.1.116.ari   25) gcc/5.1.0
  6) craype-network-aries                  16) alps/5.2.4-2.0502.9774.31.11.ari      26) cmake/3.3.2
  7) craype/2.4.2                          17) rca/1.0.0-2.0502.60530.1.62.ari       27) fftw/3.3.4.5
  8) cray-libsci/13.2.0                    18) atp/1.8.3                             28) boost/1.59
  9) udreg/2.3.2-1.0502.10518.2.17.ari     19) PrgEnv-intel/5.2.82
 10) ugni/6.0-1.0502.10863.8.29.ari        20) craype-haswell
\end{verbatim}

\section{Testing and validation of QMCPACK}
\label{sec:testing}
We \textbf{strongly encourage} running the included tests each time
QMCPACK is built. These compare the results from the executable with
known-good mean-field, quantum chemical, and other QMC results. 

The tests included with QMCPACK currently test only the VMC code with
single determinant wavefunction and simple spline Jastrow
wavefunctions, and for gaussian and periodic spline basis
sets. Although not yet comprehensive, it is extremely unlikely that,
e.g., DMC will be correct if the VMC tests do not pass.  We check that the known mean
field results are obtained with no Jastrow. When Jastrow functions are
included we test against previous QMC data. The tests are statistical
with a generous 3 $\sigma$ tolerance, however the system sizes are
small, typically $<10$ electrons, so the error bars are typically
small.

 The ``short'' tests only take a few minutes on a 16
core machine. You can run these tests using the command below in the
build directory:

\begin{verbatim}
ctest -R short   # Run the tests with "short" in their name
\end{verbatim}

The  full set of tests consist of longer versions of the short
test. They require several hours each to run yielding much more
stringent tests. To run all the tests simply run ctest in the build
directory:

\begin{verbatim}
ctest            # Run all the tests. This will take several hours.
\end{verbatim}

You can also run verbose tests which direct the QMCPACK
output to the standard output:
\begin{verbatim}
ctest -V -R short   # Verbose short tests
\end{verbatim}

The test system includes specific tests for the complex version of the code.

The data files for the tests are located in the tests directory. The
runs occur in build/src/QMCApp/test/test\_name. The numerical
comparisons and test definitions are in
src/QMCApp/test/CMakeLists.txt. If \textit{all} the QMC tests fail it is likely
that the appropriate mpiexec (or aprun, srun) is not being
called or found. If the QMC runs appear to work but all the other
tests fail it is possible that python is not working on your system -
we suggest checking some of the test outputs in build/src/QMCApp/test/test\_name.

Note that because the tests are very small, consisting of only a few
electrons, the performance is not representative of larger
calculations. For example, while the calculations might fit in cache,
there will be essentially no vectorization due to the small electron
counts. \textbf{The tests should not be used for any benchmarking or
performance analysis}. Dedicated larger runs are required.

\subsection{Automatic tests of QMCPACK}

The QMCPACK developers run automatic tests of QMCPACK on several
different computer systems and on a continous basis. We currently test
the following combinations:

\begin{itemize}
\item On a Red Hat Linux workstation:
  \begin{itemize}
  \item GCC 4.8.2 with OpenMPI and CUDA 7.0 (GPU build, run on NVIDIA K40s) 
  \item GCC 4.8.2 with OpenMPI 
  \item Intel 2016 with Intel MPI and MKL
  \item Intel 2015 with Intel MPI and MKL and CUDA 7.0 (GPU build, run on NVIDIA K40s) 
  \item Intel 2015 with Intel MPI  and MKL
  \end{itemize}
\item On Eos, a Cray XC30 Intel machine:
  \begin{itemize}
\item The default Intel programming environment and compiler with Cray MPI and Intel MKL
  \end{itemize}

\item On Titan, a Cray XK7 CPU+GPU machine:
  \begin{itemize}
  \item The GCC programming environment and compiler with Cray MPI and CUDA 
  \item The GCC programming environment and compiler with Cray MPI 
  \end{itemize}
\end{itemize}

\begin{figure}
  \centering
  \includegraphics[width=10cm]{figures/QMCPACK_CDash_Ctest_Results_20160129.png}
  \caption{Example test results for QMCPACK, showing data for a
    workstation (Intel, GCC, both CPU and GPU builds) and for two ORNL
    supercomputers. In this example, 4 errors were found.}
  \label{fig:cdash}
\end{figure}

\section{Building ppconvert, a pseudopotential format converter}
\label{sec:buildppconvert}
QMCPACK includes a utility, ppconvert, to convert between different
pseudopotential formats. Examples include effective core potential
formats (in gaussians), the UPF format used by Quantum Espresso, and
the XML format used by QMCPACK itself. The utility also enables the
atomic orbitals to recomputed via a numerical density functional
calculation if they need to be reconstructured for use in an
electronic structure calculation.

To build ppconvert follow the instructions in
src/QMCTools/ppconvert/README. Currently ppcovert is not built
automatically although we expect to automate it soon. The makefile
must be updated to refer to suitable C++ compiler and link in
BLAS. Due to the small size of the calculations, optimal settings are
not essential.

\section{Installing and patching Quantum Espresso}
\label{sec:buildqe}
For trial wavefunctions obtained in a plane-wave basis we mainly
support Quantum Espresso. Note that ABINIT and QBox were supported historically
and could be reactivated.

Quantum-Espresso current stores wavefunctions in a non-standard internal
``save'' format. To convert these to a conventional HDF5 format file
we have developed a converter, pw2qmcpack. This is an add on to the
Quantum Espresso distribution.

To simplify the process of patching Quantum Espresso we have developed
a script that will automatically download and patch the source
code. The patches are specific to each version. e.g. To download and
patch QE v5.3.0:
\begin{verbatim}
cd external_codes/quantum_espresso
./download_and_patch_qe5.3.0.sh
\end{verbatim}
After running the patch, be sure to configure quantum espresso with
the HDF5 capability enabled, e.g.
\begin{verbatim}
cd espresso-5.3.0
./configure --with-hdf5 HDF5_DIR=/opt/local   # Specify HDF5 base directory
\end{verbatim}

The complete process is described in external\_codes/quantum\_espresso/README.


\section{How to build the fastest executable version of QMCPACK}
\label{sec:buildperformance}
To build the fastest version of QMCPACK we recommend the following:
\begin{itemize}
\item Use the latest C++ compilers available for your
  system. Substantial gains have been made optimizing C++ in recent
  years.
\item Use a vendor optimized BLAS library such as Intel MKL. Although
  QMC does not make extensive use of linear algebra, it is used in the
  VMC wavefunction optimizer and also to appy the orbital coefficients in local basis
  calculations.
\item Use a vector math library such as Intel VML.  For periodic
  calculations, the calculation of the structure factor and Ewald
  potential benefit from vectorized evaluation of sin and
  cos. Currently we only autodetect Intel VML, as provided with MKL,
  but support for MASSV and AMD ACML is included via \#defines. See,
  e.g. src/Numerics/e2iphi.h. For
  large supercells, this optimization can gain 10\% in performance.
\end{itemize}

Note that greater speedups of QMC calculations can usually be obtained by
carefully choosing the required statistics for each
investigation. e.g. By not computing smaller error bars than necessary.

\section{Troubleshooting the installation}
\label{sec:troubleshoot}
Some tips to help troubleshoot installations of QMCPACK:
\begin{itemize}
\item First, build QMCPACK on a workstation that you control, or any
  system that is likely to have a simple up-to-date set of development
  tools. You can compare the results of cmake and QMCPACK on this
  system with any more difficult systems you encounter.
\item To monitor the compiler settings, use a verbose build, ``make
  VERBOSE=1''. If an individual source file is failing to compile you
  can experiment by hand using the output of the verbose build to
  reconstruct the full compilation line.
\item Use up to date development software, particularly a recent
  CMake. 
\item Verify that the compilers and libraries that you expect are
  being configured. It is common to have multiple versions
  installed. The configure system will stop at the first version it
  finds which might not be the most recent. If this occurs, specify the appropriate
  directories directly (Section \ref{sec:cmakeoptions}).
\end{itemize}

If you still have problems please post to the QMCPACK Google group with full
details, or contact a developer.
