\chapter{Examples}
\label{chap:examples}

\textbf{WARNING: THESE EXAMPLES ARE NOT CONVERGED! YOU MUST CONVERGE PARAMETERS (SIMULATION CELL SIZE, JASTROW PARAMETER NUMBER/CUTOFF, TWIST NUMBER, DMC TIME STEP, DFT PLANE WAVE CUTOFF, DFT K-POINT MESH, ETC.) FOR REAL CALCUATIONS!}

The following examples should run in serial on a modern workstation in a few hours.

\section{Using Nexus}

\subsection{H$_2$O Molecule with Quantum ESPRESSO Orbitals}

The input files are in the directory \texttt{nexus/examples/qmcpack/H2O}.

The Nexus script is in \texttt{H2O.py} and \texttt{H2O.xyz} contains the atomic positions.
To run the example, the BFD pseudopotentials are needed.  Create a \texttt{pseudopotentials} directory, and copy \texttt{O.BFD.upf}, \texttt{O.BFD.xml}, \texttt{H.BFD.upf}, and \texttt{H.BFD.xml} from \texttt{pseudopotentials/BFD} in the QMCPACK distribution.

The \texttt{H2O.py} script generates the orbitals using Quantum ESPRESSO, runs QMCPACK to optimize the Jastrow and then performs DMC for H$_2$O in a box.
The executables for Quantum ESPRESSO and QMCPACK either need to be on the path, or the paths in the script can be adjusted as needed.

For python to find the Nexus library, the PYTHONPATH environment variable should be set to \texttt{/nexus/libraries}



\subsection{LiH Crystal with Quantum ESPRESSO Orbitals}
The input files are in the directory \texttt{nexus/examples/qmcpack/LiH}.

The \texttt{LiH.py} Nexus script uses CASINO-formatted Trail-Needs pseudopotentials (see Section \ref{subsec:CASINO}) for Li and H in a subdirectory named \texttt{pseudopotentials} (both UPF and CASINO  formats, named \texttt{Li.TN-DF.upf}, \texttt{Li.pp.data}, \texttt{H.TN-DF.upf}, and \texttt{H.pp.data}, respectively), generates the orbitals using Quantum ESPRESSO, then runs QMCPACK to optimize the Jastrow and run DMC for LiH with periodic boundary conditions.
