\chapter{Auxiliary-Field Quantum Monte Carlo}\label{chap:afqmc}
The AFQMC method is an orbital-space formulation of the imaginary-time propagation algorithm. We refer the reader to one of the review articles on the method \cite{AFQMC_review,PhysRevLett.90.136401,PhysRevE.70.056702} for a detailed description of the algorithm. It uses the Hubbard-Stratonovich transformation to express the imaginary-time propagator, which is inherently a 2-body operator, as an integral over 1-body propagators, which can be efficiently applied to an arbitrary Slater determinant. This transformation allows us to represent the interacting many-body system as an average over a noninteracting system (e.g., Slater determinants) in a time-dependent fluctuating external field (the Auxiliary fields). The walkers in this case represent nonorthogonal Slater determinants, whose time average represents the desired quantum state. QMCPACK currently implements the phaseless AFQMC algorithm of Zhang and Krakauer \cite{PhysRevLett.90.136401}, where a trial wavefunction is used to project the simulation to the real axis, controlling the fermionic sign problem at the expense of a bias. This approximation is similar in spirit to the fixed-node approximation in real-space DMC but applied in the Hilbert space where the AFQMC random walk occurs.     

\section{Theoretical Background}
... Coming Soon ...

\section{Input}

The input for an AFQMC calculation is fundamentally different to the input for other real-space algorithms in QMCPACK. The main source of input comes from the Hamiltonian matrix elements in an appropriate single particle basis. This must be evaluated by an external code and saved in a format that QMCPACK can read. More details about file formats follow. The input file has six basic xml-blocks: \texttt{AFQMCInfo}, \texttt{Hamiltonian}, \texttt{Wavefunction}, \texttt{WalkerSet}, \texttt{Propagator}, and \texttt{execute}. The first five define input structures required for various types of calculations. The \texttt{execute} block represents actual calculations and takes as input the other blocks. 
Nonexecution blocks are parsed first, followed by a second pass where execution blocks are parsed (and executed) in order. Listing 15.1 shows an example of a minimal input file for an AFQMC calculation. Table~\ref{table:afqmc_basic} shows a brief description of the most important parameters in the calculation. All xml sections contain a ``name'' argument used to identify the resulting object within QMCPACK. For example, in the example, multiple Hamiltonian objects with different names can be defined. The one actually used in the calculation is the one passed to ``execute'' as ham.

\begin{lstlisting}[style=QMCPXML,caption=Sample input file for AFQMC.]
<?xml version="1.0"?>
<simulation method="afqmc">
  <project id="Carbon" series="0"/>

  <AFQMCInfo name="info0">
    <parameter name="NMO">32</parameter>
    <parameter name="NAEA">16</parameter>
    <parameter name="NAEB">16</parameter>
  </AFQMCInfo>

  <Hamiltonian name="ham0" info="info0">
    <parameter name="filename">../fcidump.h5</parameter>
  </Hamiltonian>

  <Wavefunction name="wfn0" type="MSD" info="info0">
    <parameter name="filetype">ascii</parameter>
    <parameter name="filename">wfn.dat</parameter>
  </Wavefunction>

  <WalkerSet name="wset0">
    <parameter name="walker_type">closed</parameter> 
  </WalkerSet>

  <Propagator name="prop0" info="info0">
  </Propagator>

  <execute wset="wset0" ham="ham0" wfn="wfn0" prop="prop0" info="info0">
    <parameter name="timestep">0.005</parameter>
    <parameter name="blocks">10000</parameter>
    <parameter name="nWalkers">20</parameter>
  </execute>

</simulation>
\end{lstlisting}

%The following table lists some of the most practical parameters in the \texttt{execute} block
%The following table lists some of the practical parameters
\begin{table}[h]
\begin{center}
\caption{Input options for AFQMC in QMCPACK.\label{table:afqmc_basic}}
\begin{tabularx}{\textwidth}{l l l l l X }
\hline
\multicolumn{6}{l}{\texttt{afqmc} method} \\
\hline
\multicolumn{6}{l}{parameters in \texttt{AFQMCInfo}} \\
   &   \bfseries name     & \bfseries datatype & \bfseries values & \bfseries default   & \bfseries description \\
   &   \texttt{NMO             } &  integer     & $\ge 0$ & no & Number of molecular orbitals \\
   &   \texttt{NAEA            } &  integer     & $\ge 0$ & no & Number of active electrons of spin-up \\
   &   \texttt{NAEB            } &  integer     & $\ge 0$ & no & Number of active electrons of spin-down \\
\multicolumn{6}{l}{parameters in \texttt{Hamiltonian}}  \\
   &   \texttt{info            } &  argument   &               &      & Name of \texttt{AFQMCInfo} block \\\\
   &   \texttt{filename        } &  string     &               & no   & Name of file with the hamiltonian \\
   &   \texttt{filetype        } &  string     & hdf5          & yes  & Native HDF5-based format of QMCPACK  \\ 
\multicolumn{6}{l}{parameters in \texttt{Wavefunction}}\\
   &   \texttt{info            } &  argument   &             &      & Name of \texttt{AFQMCInfo} block \\
   &   \texttt{type            } &  argument & MSD      & no   & Linear combination of (assumed non-orthogonal) Slater determinants \\
   &   \texttt{                } &           & PHMSD &    & CI-type multi-determinant wave function  \\
   &   \texttt{filetype        } &  string  & ascii       & no   & ASCII data file type \\
   &   \texttt{                } &          & hdf5        &      & HDF5 data file type \\
\multicolumn{6}{l}{parameters in \texttt{WalkerSet}} \\
   &   \texttt{walker$\_$type       } &  string    & collinear  & yes  & Request a collinear walker set. \\ 
   &   \texttt{       } &     & closed  & no  & Request a closed shell (doubly-occupied) walker set. \\ 
\multicolumn{6}{l}{parameters in \texttt{Propagator}} \\
   &   \texttt{type            } &  argument   & afqmc & afqmc & Type of propagator \\
   &   \texttt{info            } &  argument   &       &       & Name of \texttt{AFQMCInfo} block \\
   &   \texttt{hybrid   } &  string   & yes  & yes  & Use hybrid propagation algorithm. \\ 
   &   \texttt{   } &     & no  &  & Use local energy based propagation algorithm. \\ 
\multicolumn{6}{l}{parameters in \texttt{execute}} \\
   &   \texttt{wset            } &  argument    &         &      &  \\
   &   \texttt{ham             } & argument     &         &      &  \\
   &   \texttt{wfn             } & argument     &         &      &  \\
   &   \texttt{prop            } & argument     &         &      &  \\
   &   \texttt{info            } &  argument    &         &      & Name of \texttt{AFQMCInfo} block \\
   &   \texttt{nWalkers        } &  integer     & $\ge 0$ & 5    & Initial number of walkers per task group   \\
   &   \texttt{timestep        } &  real        & $> 0$   & 0.01 & Time step in 1/a.u. \\
   &   \texttt{blocks          } &  integer     & $\ge 0$ & 100  & Number of blocks            \\
   &   \texttt{step            } &  integer     & $> 0$   & 1    & Number of steps within a block \\
   &   \texttt{substep         } &  integer     & $> 0$   & 1    & Number of substeps within a step \\
   &   \texttt{ortho           } &  integer     & $> 0$   & 1    & Number of steps between walker orthogonalization. \\ 
  \hline
\end{tabularx}
\end{center}
\end{table}

The following list includes all input sections for AFQMC calculations, along with a detailed explanation of accepted parameters. Since the code is under active development, the list of parameters and their interpretation might change in the future.\\

\texttt{AFQMCInfo}: Input block that defines basic information about the calculation. It is passed to all other input blocks to propagate the basic information:
\texttt{<AFQMCInfo name="info0">}
\begin{itemize}
\item \textbf{NMO}. Number of molecular orbitals, i.e., number of states in the single particle basis. 
\item \textbf{NAEA}. Number of active electrons-alpha, i.e., number of spin-up electrons.
\item \textbf{NAEB}. Number of active electrons-beta, i.e., number of spin-down electrons.
\end{itemize}

\texttt{Hamiltonian}: Controls the object that reads, stores, and manages the \texttt{hamiltonian}. 
  \texttt{<Hamiltonian name="ham0" type="SparseGeneral" info="info0">}
\begin{itemize}
\item \textbf{filename}. Name of file with the \texttt{Hamiltonian}. This is a required parameter.
\item \textbf{cutoff\_1bar}. Cutoff applied to integrals during reading. Any term in the Hamiltonian smaller than this value is set to zero. (For filetype=``hdf5'', the cutoff is applied only to the 2-electron integrals). Default: 1e-8
\item \textbf{cutoff\_decomposition}. Cutoff used to stop the iterative cycle in the generation of the Cholesky decomposition of the 2-electron integrals. The generation of Cholesky vectors is stopped when the maximum error in the diagonal reaches this value. In case of an eigenvalue factorization, this becomes the cutoff applied to the eigenvalues. Only eigenvalues above this value are kept. Default: 1e-6
\item \textbf{nblocks}. This parameter controls the distribution of the 2-electron integrals among processors. In the default behavior (nblocks=1), all nodes contain the entire list of integrals. If nblocks $>$ 1, the of nodes in the calculation will be split in nblocks groups. Each node in a given group contains the same subset of integrals and subsequently operates on this subset during  any further operation that requires the hamiltonian. The maximum number of groups is NMO. Currently only works for filetype=``hdf5'' and the file must contain integrals.  Not yet implemented for input hamiltonians in the form of Cholesky vectors or for ASCII input. Coming soon!
    Default: No distribution
\item \textbf{printEig}. If ``yes'', prints additional information during the Cholesky decomposition.
    Default: no
\item \textbf{fix\_2eint}.  If this is set to ``yes'', orbital pairs that are found not to be positive definite are ignored in the generation of the Cholesky factorization. This is necessary if the 2-electron integrals are not positive definite because of round-off errors in their generation.
    Default: no \\
\end{itemize}

\texttt{Wavefunction}: controls the object that manages the trial wavefunctions. This block expects a list of xml-blocks defining actual trial wavefunctions for various roles. 
\texttt{<Wavefunction name="wfn0" type="MSD/PHMSD" info="info0">}
\begin{itemize}
\item \textbf{filename}. Name of file with wavefunction information.
\item \textbf{cutoff}. cutoff applied to the terms in the calculation of the local energy. Only terms in the Hamiltonian above this cutoff are included in the evaluation of the energy.
      Default: 1e-6
\item \textbf{nnodes}. Defines the parallelization of the local energy evaluation and the distribution of the \texttt{Hamiltonian} matrix (not to be confused with the list of 2-electron integrals managed by \texttt{Hamiltonian}. These are not the same.) If nnodes $>$ 1, the nodes in the simulation are split into groups of nnodes, each group works collectively in the evaluation of the local energy of their walkers. This helps distribute the effort involved in the evaluation of the local energy among the nodes in the group, but also distributes the memory associated with the wavefunction among the nodes in the group.
      Default: No distribution
\item \textbf{ndet}. Number of determinants to read from file.
      Default: Read all determinants. 
\item \textbf{cutoff}. For sparse hamiltoniants, this defines the cutoff applied to the half-rotated 2-electron integrals. 
      Default: 0.0
\item \textbf{nbatch}. This turns on(>=1)/off(==0) batched calculation of density matrices and overlaps. -1 means all the walkers in the batch. 
      Default: 0 (CPU) / -1 (GPU) 
\item \textbf{nbatch\_qr}. This turns on(>=1)/off(==0) batched QR calculation. -1 means all the walkers in the batch.
      Default: 0 (CPU) / -1 (GPU) 
\end{itemize}

\texttt{WalkerSet}: Controls the object that handles the set of walkers.
\texttt{<WalkerSet name="wset0">}
\begin{itemize}
\item \textbf{walker\_type}. Type of walker set: closed or collinear. 
      Default: collinear
\item \textbf{pop\_control}. Population control algorithm. Options: ``simple'': Uses a simple branching scheme with a fluctuating population. Walkers with weight above max\_weight are split into multiple walkers of weight reset\_weight. Walkers with weight below min\_weight are killed with probability (weight/min\_weight); ``pair'': Fixed-population branching algorithm, based on QWalk's branching algorithm. Pairs of walkers with weight above/below max\_weight/min\_weight are combined into 2 walkers with weights equal to $(w_1+w_2)/2$. The probability of replicating walker w1 (larger weight) occurs with probability $w_1/(w_1+w_2)$, otherwise walker w2 (lower weight) is replicated; ``comb'': Fixed-population branching algorithm based on the Comb method. Will be available in the next release. Default: ``pair''
\item \textbf{min\_weight}. Weight at which walkers are possibly killed (with probability weight/min\_weight). Default: 0.05
\item \textbf{max\_weight}. Weight at which walkers are replicated. Default: 4.0
\item \textbf{reset\_weight}. Weight to which replicated walkers are reset to. Default: 1.0
\end{itemize}

\texttt{Propagator}: Controls the object that manages the propagators.
\texttt{<Propagator name="prop0" info="info0">}
\begin{itemize}
\item \textbf{cutoff}. Cutoff applied to Cholesky vectors. Elements of the Cholesky vectors below this value are set to zero. Only meaningful with sparse hamiltonians.
    Default: 1e-6
\item \textbf{substractMF}. If ``yes'', apply mean-field subtraction based on the ImpSamp trial wavefunction. Must set to ``no'' to turn it off.
    Default: yes
\item \textbf{vbias\_bound}. Upper bound applied to the vias potential. Components of the vias potential above this value are truncated there. The bound is currently applied to $\sqrt{\tau} v_{bias}$, so a larger value must be used as either the time step or the fluctuations increase (e.g. from running a larger system or using a poor trial wavefunction).
    Default: 3.0
\item \textbf{apply\_constrain}. If ``yes'', apply the phaseless constrain to the walker propagation. Currently, setting this to ``no'' produces unknown behavior, since free propagation algorithm has not been tested.
    Default: yes
\item \textbf{hybrid}. If ``yes'', use hybrid propagation algorithm. This propagation scheme doesn't use the local energy during propagation, leading to significant speed ups when its evaluation  cost is high. The local energy of the ImpSamp trial wavefunction is never evaluated. To obtain energy estimates in this case, you must define an Estimator xml-block with the \texttt{Wavefunction} block. The local energy of this trial wavefunction is evaluated and printed. It is possible to use a previously defined trial wavefunction in the Estimator block, just set its ``name'' argument to the name of a previously defined wavefunction. In this case, the same object is used for both roles.
    Default: no
\item \textbf{nnodes}. Controls the parallel propagation algorithm. If nnodes $>$ 1, the nodes in the simulation are split into groups of nnodes nodes, each group working collectively to propagate their walkers.
    Default: 1 (Serial algorithm)
\item \textbf{nbatch}. This turns on(>=1)/off(==0) batched calculation of density matrices and overlaps. -1 means all the walkers in the batch.
      Default: 0 (CPU) / -1 (GPU)
\item \textbf{nbatch$\_$qr}. This turns on(>=1)/off(==0) batched QR calculation. -1 means all the walkers in the batch.
      Default: 0 (CPU) / -1 (GPU) 
\end{itemize}

\texttt{execute}: Defines an execution region. 
\texttt{<execute wset="wset0" ham="ham0" wfn="wfn0" prop="prop0" info="info0">}
\begin{itemize}
\item \textbf{nWalkers}. Initial number of walkers per core group (see ncores). This sets the number of walkers for a given gorup of ``ncores" on a node; the total number of walkers in the simulation depends on the total number of nodes and on the total number of cores on a node in the following way: $ \#_walkers_total = nWalkers * \#_nodes * \#_cores_total / ncores $. \\ 
    Default: 5
\item \textbf{timestep}. Time step in 1/a.u. \\
    Default: 0.01
\item \textbf{blocks}. Number of blocks. Slow operations occur once per block (e.g., write to file, slow observables, checkpoints), \\
    Default: 100
\item \textbf{step}. Number of steps within a block. Operations that occur at the step level include load balance, orthogonalization, branching, etc. \\
    Default: 1
\item \textbf{substep}. Number of substeps within a step. Only walker propagation occurs in a substep. \\
    Default: 1
\item \textbf{ortho}. Number of steps between orthogonalization.
    Default: 1
\item \textbf{ncores}. Number of nodes in a task group. This number defines the number of cores on a node that share the parallel work associated with a distributed task. This number is used in the \texttt{Wavefunction} and \texttt{Propagator} task groups. The walker sets are shares by the ncores on a given node in the task group.
\item \textbf{checkpoint}. Number of blocks between checkpoint files are generated. If a value smaller than 1 is given, no file is generated. If \textbf{hdf\_write\_file} is not set, a default name is used. \textbf{Default: 0} 
%\item \textbf{samplePeriod}. Number of blocks between sample collection. \textbf{Default: 0}
\item \textbf{hdf\_write\_file}. If set (and checkpoint>0), a checkpoint file with this name will be written.
\item \textbf{hdf\_read\_file}. If set, the simulation will be restarted from the given file.\\
\end{itemize}

Within the \texttt{Estimators} xml block has an argument \textbf{name}: the type of estimator we want to measure. Currently available estimators include: ``basic'', ``energy'', ``mixed\_one\_rdm'', and ``back\_propagation''.  

The basic estimator has the following optional parameters:
\begin{itemize}
\item \textbf{timers}. print timing information. Default: true
\end{itemize}

The back\_propagation estimator has the following parameters:
\begin{itemize}
\item \textbf{ortho}. Number of back-propagation steps between orthogonalization. 
    Default: 10
\item \textbf{nsteps}. Maximum number of back-propagation steps. 
    Default: 10
\item \textbf{naverages}. Number of back propagation calculations to perform. The number of steps will be chosed equally distributed in the range {0,nsteps}. 
    Default: 1
\item \textbf{block\_size}. Number of blocks to use in the internal average of the back propagated estimator. This is used to block data and reduce the size of the output. 
    Default: 1 
\item \textbf{nskip}. Number of blocks to skip at the start of the calculation for equilibration purposes. 
    Default: 0
\end{itemize}

\section{File formats}
QMCPACK offers three factorization approaches which are appropriate in different settings. The most generic approach implemented
is based on the modified-Cholesky
factorization\cite{BeebeCholesky1977,KochCholesky2003,AquilanteMOLCAS2009,PurwantoCa2011,PurwantoDownfolding2013} of the ERI
tensor:
\begin{equation}
    v_{pqrs} = V_{(pr),(sq)} \approx \sum_n^{N_\mathrm{chol}} L_{pr,n} L^{*}_{sq,n},
\end{equation}
where the sum is truncated at $N_{\mathrm{chol}} = x_c M$, $x_c$ is typically between $5$ and $10$, $M$ is the number of basis
functions and we have assumed that the single-particle orbitals are in general complex.
The storage requirement is thus naively $\mathcal{O}(M^3)$.
Note we follow the usual definition of $v_{pqrs} = \langle pq | rs \rangle = (pr|qs)$.
With this form of factorization QMCPACK allows for the integrals to be stored in either dense or sparse format.

The dense case is the simplest and is only implemented for Hamiltonians with \emph{real} integrals (and basis functions, i.e. not the homegeneous electron gas which has complex orbitals but real integrals).
The file format is given as follows:
\begin{lstlisting}[style=SHELL,caption=Sample Dense Cholesky QMCPACK Hamtiltonian.]
$ h5dump -n afqmc.h5
HDF5 "afqmc.h5" {
    FILE_CONTENTS {
        group      /
        group      /Hamiltonian
        group      /Hamiltonian/DenseFactorized
        dataset    /Hamiltonian/DenseFactorized/L
        dataset    /Hamiltonian/dims
        dataset    /Hamiltonian/hcore
        dataset    /Hamiltonian/Energies
    }
}
\end{lstlisting}
where the datasets are given by the following
\begin{itemize}
    \item \ishell{/Hamiltonian/DenseFactorized/L}: Contains the $[M^2,N_\mathrm{nchol}]$ dimensional matrix representatation of $L_{pr,n}$.
    \item \ishell{/Hamiltonian/dims}: Descriptor array of length 8 containing $[0,0,0,M,N_\alpha,N_\beta,0,N_\mathrm{nchol}]$. Note that $N_\alpha$ and $N_\beta$ are somewhat redundant and will be read from the input file and wavefunction. This allows for the Hamiltonian to be used with different (potentially spin polarized) wavefunctions.
    \item \ishell{/Hamiltonian/hcore}: Contains the $[M,M]$ dimensional one-body Hamiltonian matrix elements $h_{pq}$.
    \item \ishell{/Hamiltonian/Energies}: Array containing $[E_{II}, E_{\mathrm{core}}]$. $E_{II}$ should contain ion-ion repulsion energy and any additional constant terms which have to be added to the total energy. $E_{\mathrm{core}}$ is deprecated and not used.
\end{itemize}

Typically the Cholesky matrix is sparse, particularly if written in the non-orthogonal AO basis (not currently supported in QMCPACK). In this case only a small number of non-zero elements (denoted $nnz$ below) need to be stored which can reduce the memory overhead considerably.
Internally QMCPACK stores this matrix in the CSR format, and the HDF5 file format is reflective of this.
For large systems and, more generally when running in parallel, it is convenient to chunk the writing/reading of the Cholesky matrix into blocks of size $[M^2,\frac{N_{\mathrm{chol}}}{N_{\mathrm{blocks}}}]$ (if interpreted as a dense array).
This is achieved by writing these blocks to different data sets in the file.
For the sparse case the Hamtiltonian file format is given as follows:
\begin{lstlisting}[style=SHELL,caption=Sample Sparse Cholesky QMCPACK Hamtiltonian.]
$ h5dump -n afqmc.h5
HDF5 "afqmc.h5" {
    FILE_CONTENTS {
        group      /
        group      /Hamiltonian
        group      /Hamiltonian/Factorized
        dataset    /Hamiltonian/Factorized/block_sizes
        dataset    /Hamiltonian/Factorized/index_0
        dataset    /Hamiltonian/Factorized/vals_0
        dataset    /Hamiltonian/ComplexIntegrals
        dataset    /Hamiltonian/dims
        dataset    /Hamiltonian/hcore
        dataset    /Hamiltonian/Energies
    }
}
\end{lstlisting}
\begin{itemize}
    \item \ishell{/Hamiltonian/Factorized/block_sizes}: Contains the number of elements in each block of the sparse representation of the Cholesky matrix $L_{pr,n}$. In this case there is 1 block.
    \item \ishell{/Hamiltonian/Factorized/index_0}: $[2\times nnz]$ dimensional array, containing the indices of the non-zero values of $L_{ik,n}$. The row indices are stored in the even entries, and the column indices in the odd entries.
    \item \ishell{/Hamiltonian/Factorized/vals_0}: $[nnz]$ length array containing non-zero values of $L_{pr,n}$ for chunk 0.
    \item \ishell{/Hamiltonian/dims}: Descriptor array of length 8 containing $[0,nnz,N_{\mathrm{block}},M,N_\alpha,N_\beta,0,N_\mathrm{nchol}]$.
    \item \ishell{/Hamiltonian/ComplexIntegrals}: Length 1 array that specifies if integrals are complex valued. 1 for complex integrals, 0 for real integrals.
    \item \ishell{/Hamiltonian/hcore}: Contains the $[M,M]$ dimensional one-body Hamiltonian matrix elements $h_{pq}$. Due to its small size this is written as a dense 2D-array.
    \item \ishell{/Hamiltonian/Energies}: Array containing $[E_{II}, E_{\mathrm{core}}]$. $E_{II}$ should contain ion-ion repulsion energy and any additional constant terms which have to be added to the total energy. $E_{\mathrm{core}}$ is deprecated and not used.
\end{itemize}

To reduce the memory overhead of storing the three-index tensor we recently adapted the
tensor-hypercontraction\cite{HohensteinTHCI2012,ParrishTHCII2012,HohensteinTHCIII2012} (THC) approach for use in AFQMC\cite{MaloneISDF2019}. Within the THC approach we
can approximate the orbital products entering the ERIs as
\begin{equation}
    \varphi^{*}_p(\mathbf{r})\varphi_r(\mathbf{r}) \approx \sum_\mu^{N_\mu} \zeta_\mu(\mathbf{r}) \varphi^*_p(\mathbf{r}_\mu)\varphi_r(\mathbf{r}_\mu),\label{eq:orb_prod}
\end{equation}
where $\varphi_p(\mathbf{r})$ are the one-electron orbitals and $\mathbf{r}_\mu$ are a set of specially selected interpolating
points, $\zeta_\mu(\mathbf{r})$ are a set of interpolating vectors and $N_\mu = x_\mu M$. We can then write the ERI tensor as a
product of rank-2 tensors
\begin{equation}
    v_{pqrs} \approx \sum_{\mu\nu} \varphi^{*}_p(\mathbf{r}_\mu)\varphi_r(\mathbf{r}_\mu) M_{\mu\nu} \varphi^{*}_q(\mathbf{r}_\nu)\varphi_s(\mathbf{r}_\nu)\label{eq:4ix_thc},
\end{equation}
where
\begin{equation}
    M_{\mu\nu} = \int d\mathbf{r}d\mathbf{r}' \zeta_\mu(\mathbf{r})\frac{1}{|\mathbf{r}-\mathbf{r}'|}\zeta^{*}_\nu(\mathbf{r}')\label{eq:mmat}.
\end{equation}
We also require the half-rotated versions of these quantities which live on a different set of $\tilde{N}_\mu$ interpolating points $\tilde{\mathbf{r}}_\mu$ (see Ref.~\cite{MaloneISDF2019}).
The file format for THC factorization is as follows:
\begin{lstlisting}[style=SHELL,caption=Sample Sparse Cholesky QMCPACK Hamtiltonian.]
$ h5dump -n afqmc.h5
HDF5 "afqmc.h5" {
    FILE_CONTENTS {
        group      /
        group      /Hamiltonian
        group      /Hamiltonian/THC
        dataset    /Hamiltonian/THC/Luv
        dataset    /Hamiltonian/THC/Orbitals
        dataset    /Hamiltonian/THC/HalfTransformedMuv
        dataset    /Hamiltonian/THC/HalfTransformedFullOrbitals
        dataset    /Hamiltonian/THC/HalfTransformedOccOrbitals
        dataset    /Hamiltonian/THC/dims
        dataset    /Hamiltonian/ComplexIntegrals
        dataset    /Hamiltonian/dims
        dataset    /Hamiltonian/hcore
        dataset    /Hamiltonian/Energies
    }
}
\end{lstlisting}
\begin{itemize}
    \item \ishell{/Hamiltonian/THC/Luv}: Cholesky factorization of the $M_{\mu\nu}$ matrix given in Eq.~\ref{eq:mmat}.
    \item \ishell{/Hamiltonian/THC/Orbitals}: $[M,N_\mu]$ dimensional array of orbitals evaluated at chosen interpolating points $\varphi_i(\mathbf{r}_\mu)$.
    \item \ishell{/Hamiltonian/THC/HalfTransformedMuv}: $[\tilde{N}_\mu,\tilde{N}_\mu]$ dimensional array containing half-transformed $\tilde{M}_{\mu\nu}$.
    \item \ishell{/Hamiltonian/THC/HalfTransformedFullOrbitals}: $[M,\tilde{N}_\mu]$ dimensional array containing orbital set computed at half-transformed interpolating points $\varphi_i(\tilde{\mathbf{r}}_\mu)$.
    \item \ishell{/Hamiltonian/THC/HalfTransformedOccOrbitals}: $[N_\alpha+N_\beta,\tilde{N}_\mu]$ dimensional array containing half-rotated orbital set computed at half-transformed interpolating points $\varphi_a(\tilde{\mathbf{r}}_\mu) = \sum_{p} A_{pa}^* \varphi_{p}(\tilde{\mathbf{r}}_\mu)$, where $\mathbf{A}$ is the Slater-Matrix of the (currently single-determinant) trial wavefunction.
    \item \ishell{/Hamiltonian/THC/dims}: Descriptor array containing $[M, N_\mu, \tilde{N}_\mu]$.
    \item \ishell{/Hamiltonian/ComplexIntegrals}: Length 1 array that specifies if integrals are complex valued. 1 for complex integrals, 0 for real integrals.
    \item \ishell{/Hamiltonian/dims}: Descriptor array of length 8 containing $[0,0,0,M,N_\alpha,N_\beta,0,0]$.
    \item \ishell{/Hamiltonian/hcore}: Contains the $[M,M]$ dimensional one-body Hamiltonian matrix elements $h_{ij}$.
    \item \ishell{/Hamiltonian/Energies}: Array containing $[E_{II}, E_{\mathrm{core}}]$. $E_{II}$ should contain ion-ion repulsion energy and any additional constant terms which have to be added to the total energy (such as the electron-electron interaction Madelung contribution of $\frac{1}{2} N \xi )$. $E_{\mathrm{core}}$ is deprecated and not used.
\end{itemize}

Finally, we have implemented an explicitly $k$-point dependent factorization for periodic systems\cite{MottaKPoint2019,MaloneGPU2020}
\begin{equation}
    v_{pqrs} = \sum_{\substack{n\textbf{Q}\textbf{k}\textbf{k}' \\ pqrs\sigma\sigma'}} L^{\textbf{Q},\textbf{k}}_{pr,n} {L^{\textbf{Q},\textbf{k}'}_{sq,n}}^{*}\label{eq:kp_h2}
\end{equation}
where $\textbf{k}$, $\textbf{k}'$ and $\textbf{Q}$ are vectors in the first Brillouin zone.
The one-body Hamiltonian is block diagonal in $\textbf{k}$ and in Eq. \ref{eq:kp_h2} we have used momentum conservation $(\textbf{k}_p - \textbf{k}_r + \textbf{k}_q - \textbf{k}_s) = \textbf{G}$ with $\textbf{G}$ being some vector in the reciprocal lattice of the simulation cell.
The convention for the Cholesky matrix $L^{\textbf{Q},\textbf{k}}_{pr,\gamma}$ is as follows: $\textbf{k}_r = \textbf{k}_p - \textbf{Q}$, so the vector $\textbf{k}$ labels the \textit{k}-point of the first band index, $\textit{p}$, while the \textit{k}-point vector of the second band index, $\textit{r}$, is given by $\textbf{k} - \textbf{Q}$.  Electron repulsion integrals at different $\textbf{Q}$ vectors are zero by symmetry, resulting in a reduction in the number of required $\mathbf{Q}$ vectors.
For certain $\textbf{Q}$ vectors that satisfy $\textbf{Q} \ne -\textbf{Q}$ (this is not satisfied at the origin and at high symmetry points on the edge of the 1BZ), we have ${L^{\textbf{Q},\textbf{k}}_{sq,\gamma}}^{*} = {L^{-\textbf{Q},\textbf{k}-\textbf{Q}}_{qs,\gamma}}$, which requires us to store Cholesky vectors for either one of the $(\textbf{Q},-\textbf{Q})$ pair, but not both.

In what follows let $m_{\mathbf{k}}$ denote the number of basis functions for basis functions of a given $k$-point (these can in principle differ for different $k$-points due to linear dependencies), $n^{\alpha}_{\mathbf{k}}$ the number of $\alpha$ electrons in a given $k$-point and $n_{\mathrm{chol}}^{\mathbf{Q}_n}$ the number of Cholesky vectors for momentum transfer $\mathbf{Q}_n$.
The file format for this factorization is as follows (for a $2\times2\times2$ $k$-point mesh, for denser meshes generally there will be far fewer symmetry inequivalent momentum transfer vectors than there are $k$-points):
\begin{lstlisting}[style=SHELL,caption=Sample Dense $k$-point dependent Cholesky QMCPACK Hamtiltonian.]
$ h5dump -n afqmc.h5
HDF5 "afqmc.h5" {
    FILE_CONTENTS {
        group      /
        group      /Hamiltonian
        group      /Hamiltonian/KPFactorized
        dataset    /Hamiltonian/KPFactorized/L0
        dataset    /Hamiltonian/KPFactorized/L1
        dataset    /Hamiltonian/KPFactorized/L2
        dataset    /Hamiltonian/KPFactorized/L3
        dataset    /Hamiltonian/KPFactorized/L4
        dataset    /Hamiltonian/KPFactorized/L5
        dataset    /Hamiltonian/KPFactorized/L6
        dataset    /Hamiltonian/KPFactorized/L7
        dataset    /Hamiltonian/NCholPerKP
        dataset    /Hamiltonian/MinusK
        dataset    /Hamiltonian/NMOPerKP
        dataset    /Hamiltonian/QKTok2
        dataset    /Hamiltonian/H1_kp0
        dataset    /Hamiltonian/H1_kp1
        dataset    /Hamiltonian/H1_kp2
        dataset    /Hamiltonian/H1_kp3
        dataset    /Hamiltonian/H1_kp4
        dataset    /Hamiltonian/H1_kp5
        dataset    /Hamiltonian/H1_kp6
        dataset    /Hamiltonian/H1_kp7
        dataset    /Hamiltonian/ComplexIntegrals
        dataset    /Hamiltonian/KPoints
        dataset    /Hamiltonian/dims
        dataset    /Hamiltonian/Energies
    }
}
\end{lstlisting}
\begin{itemize}
    \item \ishell{/Hamiltonian/KPFactorized/L[n]}: This series of datasets store elements of the Cholesky tensors $L[\mathbf{Q}_n,\mathbf{k},pr,n]$. Each data set is of dimension $[N_k,m_{\mathbf{k}}\times m_{\mathbf{k}'},n^{\mathbf{Q}_n}_\mathrm{chol}]$, where, again, $k$ is the $k$-point associated with basis function $p$, the $k$-point of basis function $r$ is defined via the mapping \ishell{QKtok2}.
    \item \ishell{/Hamiltonian/NCholPerKP}: $N_k$ length array giving number of Cholesky vectors per $k$-point.
    \item \ishell{/Hamiltonian/MinusK}: $N_k$ length array mapping a $k$-point to its inverse: $\mathbf{k}_i+$\ishell{MinusK[i]} $= \mathbf{0} \mod \mathbf{G}$.
    \item \ishell{/Hamiltonian/NMOPerKP}: $N_k$ length array listing number of basis functions per $k$-point.
    \item \ishell{/Hamiltonian/QKTok2}: $[N_k,N_k]$ dimensional array. \ishell{QKtok2[i,j]} yields the $k$ point index satisfying $\mathbf{k}=\mathbf{Q}_i-\mathbf{k}_j+\mathbf{G}$.
    \item \ishell{/Hamiltonian/dims}: Descriptor array of length 8 containing $[0,0,0,M,N_\alpha,N_\beta,0,0]$.
    \item \ishell{/Hamiltonian/H1_kp[n]}: Contains the $[m_{\mathbf{k}_n},m_{\mathbf{k}_n}]$ dimensional one-body Hamiltonian matrix elements $h_{(\mathbf{k}_{n}p)(\mathbf{k}_{n}q)}$.
    \item \ishell{/Hamiltonian/ComplexIntegrals}: Length 1 array that specifies if integrals are complex valued. 1 for complex integrals, 0 for real integrals.
    \item \ishell{/Hamiltonian/KPoints} $[N_k,3]$ Dimensional array containing $k$-points used to sample Brillouin zone.
    \item \ishell{/Hamiltonian/dims}: Descriptor array of length 8 containing $[0,0,N_k,M,N_\alpha,N_\beta,0,N_\mathrm{nchol}]$. Note that $M$ is the total number of basis functions, i.e. $M=\sum_\mathbf{k} m_\mathbf{k}$, and likewise for the number of electrons.
    \item \ishell{/Hamiltonian/Energies}: Array containing $[E_{II}, E_{\mathrm{core}}]$. $E_{II}$ should contain ion-ion repulsion energy and any additional constant terms which have to be added to the total energy (such as the electron-electron interaction Madelung contribution of $\frac{1}{2} N \xi )$. $E_{\mathrm{core}}$ is deprecated and not used.
\end{itemize}

Complex integrals should be written as an array with an additional dimension, e.g., a 1D array should be written as a 2D array with \ishell{array_hdf5[:,0]=real(1d_array)} and \ishell{array_hdf5[:,1]=imag(1d_array)}. The functions \ishell{afqmctools.utils.misc.from_qmcpack_complex} and \ishell{afqmctools.utils.misc.to_qmcpack_complex} can be used to transform qmcpack format to complex valued numpy arrays of the appropriate shape and vice versa.

Finally, if using external tools to generate this file format, we provide a sanity checker script in \ishell{utils/afqmctools/bin/test_afqmc_input.py} which will raise errors if the format does not conform to what is being used internally.

\section{Advice/Useful Information}

AFQMC calculations are computationally expensive and require some care to obtain reasonable performance.
The following is a growing list of useful advice for new users, followed by a sample input for a large calculation.
\begin{itemize}
\item Generate Cholesky-decomposed integrals with external codes instead of the 2-electron integrals directly. The generation of the Cholesky factorization is faster and consumes less memory. 
\item Use the hybrid algorithm for walker propagation. Set steps/substeps to adequate values to reduce the number of energy evaluations. This is essential when using large multideterminant expansions.
\item Adjust cutoffs in the wavefunction and propagator bloxks until desired accuracy is reached. The cost of the calculation will depend on these cutoffs.
\item Adjust ncores/nWalkers to obtain better efficiency. Larger nWalkers will lead to more efficient linear algebra operations but will increase the time per step. Larger ncores will reduce the time per step but will reduce efficiency because of inefficiencies in the parallel implementation. For large calculations, values between 6--12 for both quantities should be reasonable, depending on architecture. 
\end{itemize}

\begin{lstlisting}[style=QMCPXML,caption=Example of sections of an AFQMC input file for a large calculation.]
...

  <Hamiltonian name="ham0" type="SparseGeneral" info="info0">
    <parameter name="filename">fcidump.h5</parameter>
    <parameter name="cutoff_1bar">1e-6</parameter>
    <parameter name="cutoff_decomposition">1e-5</parameter>
  </Hamiltonian>

  <Wavefunction name="wfn0" type="MSD" info="info0">
    <parameter name="filetype">ascii</parameter>
    <parameter name="filename">wfn.dat</parameter>
  </Wavefunction>

  <WalkerSet name="wset0">
    <parameter name="walker_type">closed</parameter>
  </WalkerSet>

  <Propagator name="prop0" info="info0">
    <parameter name="hybrid">yes</parameter>
  </Propagator>

  <execute wset="wset0" ham="ham0" wfn="wfn0" prop="prop0" info="info0">
    <parameter name="ncores">8</parameter>
    <parameter name="timestep">0.01</parameter>
    <parameter name="blocks">10000</parameter>
    <parameter name="steps">10</parameter>
    <parameter name="substeps">5</parameter>
    <parameter name="nWalkers">8</parameter>
    <parameter name="ortho">5</parameter>
  </execute>
\end{lstlisting}
